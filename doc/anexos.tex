\documentclass[a4paper,12pt,twoside]{memoir}

% Castellano
% Castellano
\usepackage[spanish,es-tabla]{babel}
\selectlanguage{spanish}
\usepackage[utf8]{inputenc}
\usepackage[T1]{fontenc}
\usepackage{lmodern} % Scalable font
\usepackage{microtype}
\usepackage{placeins}
\usepackage{csquotes}
\usepackage{lscape} 
\usepackage[table,xcdraw]{xcolor}
\usepackage{graphicx}
\usepackage{float} %required for the placement specifier H

\usepackage{makecell}

\renewcommand\theadalign{bc}
\renewcommand\theadfont{\bfseries}
\renewcommand\theadgape{\Gape[4pt]}
\renewcommand\cellgape{\Gape[4pt]}

% Landscape
\usepackage{pdflscape}

% Mathematic font
\usepackage{amsfonts}

\RequirePackage{booktabs}
\RequirePackage[table]{xcolor}
\RequirePackage{xtab}
\RequirePackage{multirow}

% Multi-page tables using
\nouppercaseheads
\usepackage{longtable}
\usepackage{tabularx}

% Cell with line break (e.g. \specialcell{Foo\\bar})
\newcommand{\specialcell}[2][c]{%
  \begin{tabular}[#1]{@{}l@{}}#2\end{tabular}}

% Links
\PassOptionsToPackage{hyphens}{url}\usepackage[colorlinks]{hyperref}
\hypersetup{
	allcolors = {blue}
}

% Ecuaciones
\usepackage{amsmath}

% Rutas de fichero / paquete
\newcommand{\ruta}[1]{{\sffamily #1}}

% Párrafos
\nonzeroparskip

% Huérfanas y viudas
\widowpenalty100000
\clubpenalty100000

% Imagenes
\usepackage{graphicx}
\newcommand{\imagen}[2]{
	\begin{figure}[!h]
		\centering
		\includegraphics[width=0.9\textwidth]{#1}
		\caption{#2}\label{fig:#1}
	\end{figure}
	\FloatBarrier
}

\newcommand{\imagenAncho}[3]{
	\begin{figure}[H]
		\centering
		\includegraphics[width=#3\textwidth]{#1}
		\caption{#2}\label{fig:#1}
	\end{figure}
	\FloatBarrier
}

\newcommand{\imagenflotante}[2]{
	\begin{figure}%[!h]
		\centering
		\includegraphics[width=0.9\textwidth]{#1}
		\caption{#2}\label{fig:#1}
	\end{figure}
}

\usepackage{listings}
\usepackage{xcolor}

\colorlet{punct}{red!60!black}
\definecolor{background}{HTML}{EEEEEE}
\definecolor{delim}{RGB}{20,105,176}
\colorlet{numb}{magenta!60!black}

\lstdefinelanguage{json}{
    basicstyle=\normalfont\ttfamily,
    numbers=left,
    numberstyle=\scriptsize,
    stepnumber=1,
    numbersep=8pt,
    showstringspaces=false,
    breaklines=true,
    frame=lines,
    backgroundcolor=\color{background},
    literate=
     *{0}{{{\color{numb}0}}}{1}
      {1}{{{\color{numb}1}}}{1}
      {2}{{{\color{numb}2}}}{1}
      {3}{{{\color{numb}3}}}{1}
      {4}{{{\color{numb}4}}}{1}
      {5}{{{\color{numb}5}}}{1}
      {6}{{{\color{numb}6}}}{1}
      {7}{{{\color{numb}7}}}{1}
      {8}{{{\color{numb}8}}}{1}
      {9}{{{\color{numb}9}}}{1}
      {:}{{{\color{punct}{:}}}}{1}
      {,}{{{\color{punct}{,}}}}{1}
      {\{}{{{\color{delim}{\{}}}}{1}
      {\}}{{{\color{delim}{\}}}}}{1}
      {[}{{{\color{delim}{[}}}}{1}
      {]}{{{\color{delim}{]}}}}{1},
}
\definecolor{maroon}{cmyk}{0, 0.87, 0.68, 0.32}
\definecolor{halfgray}{gray}{0.55}
\definecolor{ipython_frame}{RGB}{207, 207, 207}
\definecolor{ipython_bg}{RGB}{247, 247, 247}
\definecolor{ipython_red}{RGB}{186, 33, 33}
\definecolor{ipython_green}{RGB}{0, 128, 0}
\definecolor{ipython_cyan}{RGB}{64, 128, 128}
\definecolor{ipython_purple}{RGB}{170, 34, 255}

\lstdefinelanguage{python}{
    morekeywords={access,and,break,class,continue,def,del,elif,else,except,exec,finally,for,from,global,if,import,in,is,lambda,not,or,pass,print,raise,return,try,while},
    morekeywords=[2]{abs,all,any,basestring,bin,bool,bytearray,callable,chr,classmethod,cmp,compile,complex,delattr,dict,dir,divmod,enumerate,eval,execfile,file,filter,float,format,frozenset,getattr,globals,hasattr,hash,help,hex,id,input,int,isinstance,issubclass,iter,len,list,locals,long,map,max,memoryview,min,next,object,oct,open,ord,pow,property,range,raw_input,reduce,reload,repr,reversed,round,set,setattr,slice,sorted,staticmethod,str,sum,super,tuple,type,unichr,unicode,vars,xrange,zip,apply,buffer,coerce,intern},
    sensitive=true,
    morecomment=[l]\#,
    morestring=[b]',
    morestring=[b]",
    morestring=[s]{'''}{'''},
    morestring=[s]{"""}{"""},
    morestring=[s]{r'}{'},
    morestring=[s]{r"}{"},
    morestring=[s]{r'''}{'''},
    morestring=[s]{r"""}{"""},
    morestring=[s]{u'}{'},
    morestring=[s]{u"}{"},
    morestring=[s]{u'''}{'''},
    morestring=[s]{u"""}{"""},
    % {replace}{replacement}{lenght of replace}
    % *{-}{-}{1} will not replace in comments and so on
    literate=
    {á}{{\'a}}1 {é}{{\'e}}1 {í}{{\'i}}1 {ó}{{\'o}}1 {ú}{{\'u}}1
    {Á}{{\'A}}1 {É}{{\'E}}1 {Í}{{\'I}}1 {Ó}{{\'O}}1 {Ú}{{\'U}}1
    {à}{{\`a}}1 {è}{{\`e}}1 {ì}{{\`i}}1 {ò}{{\`o}}1 {ù}{{\`u}}1
    {À}{{\`A}}1 {È}{{\'E}}1 {Ì}{{\`I}}1 {Ò}{{\`O}}1 {Ù}{{\`U}}1
    {ä}{{\"a}}1 {ë}{{\"e}}1 {ï}{{\"i}}1 {ö}{{\"o}}1 {ü}{{\"u}}1
    {Ä}{{\"A}}1 {Ë}{{\"E}}1 {Ï}{{\"I}}1 {Ö}{{\"O}}1 {Ü}{{\"U}}1
    {â}{{\^a}}1 {ê}{{\^e}}1 {î}{{\^i}}1 {ô}{{\^o}}1 {û}{{\^u}}1
    {Â}{{\^A}}1 {Ê}{{\^E}}1 {Î}{{\^I}}1 {Ô}{{\^O}}1 {Û}{{\^U}}1
    {œ}{{\oe}}1 {Œ}{{\OE}}1 {æ}{{\ae}}1 {Æ}{{\AE}}1 {ß}{{\ss}}1
    {ç}{{\c c}}1 {Ç}{{\c C}}1 {ø}{{\o}}1 {å}{{\r a}}1 {Å}{{\r A}}1
    {€}{{\EUR}}1 {£}{{\pounds}}1
    %
    {^}{{{\color{ipython_purple}\^{}}}}1
    {=}{{{\color{ipython_purple}=}}}1
    %
    {+}{{{\color{ipython_purple}+}}}1
    {*}{{{\color{ipython_purple}$^\ast$}}}1
    {/}{{{\color{ipython_purple}/}}}1
    %
    {+=}{{{+=}}}1
    {-=}{{{-=}}}1
    {*=}{{{$^\ast$=}}}1
    {/=}{{{/=}}}1,
    literate=
    *{-}{{{\color{ipython_purple}-}}}1
     {?}{{{\color{ipython_purple}?}}}1,
    %
    identifierstyle=\color{black}\ttfamily,
    commentstyle=\color{ipython_cyan}\ttfamily,
    stringstyle=\color{ipython_red}\ttfamily,
    keepspaces=true,
    showspaces=false,
    showstringspaces=false,
    rulecolor=\color{ipython_frame},
    frame=single,
    frameround={t}{t}{t}{t},
    framexleftmargin=6mm,
    numbers=left,
    numberstyle=\tiny\color{halfgray},
    backgroundcolor=\color{ipython_bg},
    % extendedchars=true,
    basicstyle=\scriptsize,
    keywordstyle=\color{ipython_green}\ttfamily,
}

\definecolor{listinggray}{gray}{0.9}
\definecolor{lbcolor}{rgb}{0.9,0.9,0.9}
\lstdefinelanguage{ccc}{
    backgroundcolor=\color{lbcolor},
    tabsize=4,    
    language=[GNU]C++,
    basicstyle=\scriptsize,
    upquote=true,
    aboveskip={1.5\baselineskip},
    columns=fixed,
    showstringspaces=false,
    extendedchars=false,
    breaklines=true,
    prebreak = \raisebox{0ex}[0ex][0ex]{\ensuremath{\hookleftarrow}},
    frame=single,
    numbers=left,
    showtabs=false,
    showspaces=false,
    showstringspaces=false,
    identifierstyle=\ttfamily,
    keywordstyle=\color[rgb]{0,0,1},
    commentstyle=\color[rgb]{0.026,0.112,0.095},
    stringstyle=\color[rgb]{0.627,0.126,0.941},
    numberstyle=\color[rgb]{0.205, 0.142, 0.73},
}



\definecolor{dkgreen}{rgb}{0,0.6,0}
\definecolor{dred}{rgb}{0.545,0,0}
\definecolor{dblue}{rgb}{0,0,0.545}
\definecolor{lgrey}{rgb}{0.9,0.9,0.9}
\definecolor{gray}{rgb}{0.4,0.4,0.4}
\definecolor{darkblue}{rgb}{0.0,0.0,0.6}
\lstdefinelanguage{cpp}{
      backgroundcolor=\color{lgrey},  
      basicstyle=\footnotesize \ttfamily \color{black} \bfseries,   
      breakatwhitespace=false,       
      breaklines=true,               
      captionpos=b,                   
      commentstyle=\color{dkgreen},   
      deletekeywords={...},          
      escapeinside={\%*}{*)},                  
      frame=single,                  
      language=C++,                
      keywordstyle=\color{purple},  
      morekeywords={BRIEFDescriptorConfig,string,TiXmlNode,DetectorDescriptorConfigContainer,istringstream,cerr,exit}, 
      identifierstyle=\color{black},
      stringstyle=\color{blue},      
      numbers=right,                 
      numbersep=5pt,                  
      numberstyle=\tiny\color{black}, 
      rulecolor=\color{black},        
      showspaces=false,               
      showstringspaces=false,        
      showtabs=false,                
      stepnumber=1,                   
      tabsize=5,                     
      title=\lstname,                 
    }

%---------------------------------------------------------
%\usepackage{xcolor}

\definecolor{azulon}{rgb}{0,95,175}
\definecolor{dgreen}{RGB}{0, 153, 0}



\lstdefinelanguage{sh}
{morekeywords={locate,ls,man,cmp,diff,mkdir,cp,more,chgrp,mv,chmod,chown,rm,file,rmdir,find,tail,grep,umask,head,wc,info,whatis,less,whereis,date,halt,shutdown,reboot,break,case,cat,cd,continue,do,done,elif,else,exec,exit,export,expr,false,fi,for,function,if,in,kill,login,new,nohup,ps,read,readonly,return,then,true,type,wait,while,as,command,declare,help,history,jobs,logout,printf,pushd,popd,readarray,select,set,type,wait,sleep,path,pwd,out,systemctl,nano,Install,Service,Unit}, 
morecomment=[l]\#,
morestring=[d]",
morekeywords=[2]{sudo,bash,sh ,python3,\$,\{,\},\(,\),pip,pip3,gpio,stop,start,always,never,zero,enable,disable,default,install,echo}, %Lanzamientos
morekeywords=[3]{bool,int,str,string,'\n','\t',+,-,*,/,=,\%,\$path,mode,User,ExecStart,WorkingDirectory,WantedBy,Restart,RestartSec,StartLimitIntervalSec,Description,After,Type  }, %operaciones
%
backgroundcolor=\color{lgrey},
basicstyle=\scriptsize\ttfamily,
identifierstyle=\color{black}\ttfamily,
keywordstyle=\ttfamily \color{dgreen}, %dblue
keywordstyle={[2]\ttfamily\color{red}},
keywordstyle={[3]\ttfamily\color{olive}},
frame=tlb,% the frame is open on the right side
resetmargins=false,
rulesepcolor=\color{black}\ttfamily,
numbers=left,% % left
numberstyle=\tiny\ttfamily,
numbersep=5pt,
extendedchars=true, 
firstnumber=1,
stepnumber=1,
columns=fixed,% % to prevent inserting spaces
fontadjust=true,
keepspaces=true,
basewidth=0.5em,
captionpos=t,
abovecaptionskip=\smallskipamount\ttfamily,% same amount as default
belowcaptionskip=\smallskipamount\ttfamily,% in caption package
stringstyle=\color{black}\ttfamily,
commentstyle=\color{teal}\ttfamily
}[keywords,comments,strings]

\renewcommand{\lstlistingname}{Código}%





%---------------------------------------------------------


% El comando \figura nos permite insertar figuras comodamente, y utilizando
% siempre el mismo formato. Los parametros son:
% 1 -> Porcentaje del ancho de página que ocupará la figura (de 0 a 1)
% 2 --> Fichero de la imagen
% 3 --> Texto a pie de imagen
% 4 --> Etiqueta (label) para referencias
% 5 --> Opciones que queramos pasarle al \includegraphics
% 6 --> Opciones de posicionamiento a pasarle a \begin{figure}
\newcommand{\figuraConPosicion}[6]{%
  \setlength{\anchoFloat}{#1\textwidth}%
  \addtolength{\anchoFloat}{-4\fboxsep}%
  \setlength{\anchoFigura}{\anchoFloat}%
  \begin{figure}[#6]
    \begin{center}%
      \Ovalbox{%
        \begin{minipage}{\anchoFloat}%
          \begin{center}%
            \includegraphics[width=\anchoFigura,#5]{#2}%
            \caption{#3}%
            \label{#4}%
          \end{center}%
        \end{minipage}
      }%
    \end{center}%
  \end{figure}%
}

%
% Comando para incluir imágenes en formato apaisado (sin marco).
\newcommand{\figuraApaisadaSinMarco}[5]{%
  \begin{figure}%
    \begin{center}%
    \includegraphics[angle=90,height=#1\textheight,#5]{#2}%
    \caption{#3}%
    \label{#4}%
    \end{center}%
  \end{figure}%
}
% Para las tablas
\newcommand{\otoprule}{\midrule [\heavyrulewidth]}
%
% Nuevo comando para tablas pequeñas (menos de una página).
\newcommand{\tablaSmall}[5]{%
 \begin{table}[H]
  \begin{center}
   \rowcolors {2}{gray!35}{}
   \begin{tabular}{#2}
    \toprule
    #4
    \otoprule
    #5
    \bottomrule
   \end{tabular}
   \caption{#1}
   \label{tabla:#3}
  \end{center}
 \end{table}
}

%
%Para el float H de tablaSmallSinColores
\usepackage{float}

%
% Nuevo comando para tablas pequeñas (menos de una página).
\newcommand{\tablaSmallSinColores}[5]{%
 \begin{table}[H]
  \begin{center}
   \begin{tabular}{#2}
    \toprule
    #4
    \otoprule
    #5
    \bottomrule
   \end{tabular}
   \caption{#1}
   \label{tabla:#3}
  \end{center}
 \end{table}
}

\newcommand{\tablaApaisadaSmall}[5]{%
\begin{landscape}
  \begin{table}
   \begin{center}
    \rowcolors {2}{gray!35}{}
    \begin{tabular}{#2}
     \toprule
     #4
     \otoprule
     #5
     \bottomrule
    \end{tabular}
    \caption{#1}
    \label{tabla:#3}
   \end{center}
  \end{table}
\end{landscape}
}

%
% Nuevo comando para tablas grandes con cabecera y filas alternas coloreadas en gris.
\newcommand{\tabla}[6]{%
  \begin{center}
    \tablefirsthead{
      \toprule
      #5
      \otoprule
    }
    \tablehead{
      \multicolumn{#3}{l}{\small\sl continúa desde la página anterior}\\
      \toprule
      #5
      \otoprule
    }
    \tabletail{
      \hline
      \multicolumn{#3}{r}{\small\sl continúa en la página siguiente}\\
    }
    \tablelasttail{
      \hline
    }
    \bottomcaption{#1}
    \rowcolors {2}{gray!35}{}
    \begin{xtabular}{#2}
      #6
      \bottomrule
    \end{xtabular}
    \label{tabla:#4}
  \end{center}
}

%
% Nuevo comando para tablas grandes con cabecera.
\newcommand{\tablaSinColores}[6]{%
  \begin{center}
    \tablefirsthead{
      \toprule
      #5
      \otoprule
    }
    \tablehead{
      \multicolumn{#3}{l}{\small\sl continúa desde la página anterior}\\
      \toprule
      #5
      \otoprule
    }
    \tabletail{
      \hline
      \multicolumn{#3}{r}{\small\sl continúa en la página siguiente}\\
    }
    \tablelasttail{
      \hline
    }
    \bottomcaption{#1}
    \begin{xtabular}{#2}
      #6
      \bottomrule
    \end{xtabular}
    \label{tabla:#4}
  \end{center}
}

%
% Nuevo comando para tablas grandes sin cabecera.
\newcommand{\tablaSinCabecera}[5]{%
  \begin{center}
    \tablefirsthead{
      \toprule
    }
    \tablehead{
      \multicolumn{#3}{l}{\small\sl continúa desde la página anterior}\\
      \hline
    }
    \tabletail{
      \hline
      \multicolumn{#3}{r}{\small\sl continúa en la página siguiente}\\
    }
    \tablelasttail{
      \hline
    }
    \bottomcaption{#1}
  \begin{xtabular}{#2}
    #5
   \bottomrule
  \end{xtabular}
  \label{tabla:#4}
  \end{center}
}



\definecolor{cgoLight}{HTML}{EEEEEE}
\definecolor{cgoExtralight}{HTML}{FFFFFF}

%
% Nuevo comando para tablas grandes sin cabecera.
\newcommand{\tablaSinCabeceraConBandas}[5]{%
  \begin{center}
    \tablefirsthead{
      \toprule
    }
    \tablehead{
      \multicolumn{#3}{l}{\small\sl continúa desde la página anterior}\\
      \hline
    }
    \tabletail{
      \hline
      \multicolumn{#3}{r}{\small\sl continúa en la página siguiente}\\
    }
    \tablelasttail{
      \hline
    }
    \bottomcaption{#1}
    \rowcolors[]{1}{cgoExtralight}{cgoLight}

  \begin{xtabular}{#2}
    #5
   \bottomrule
  \end{xtabular}
  \label{tabla:#4}
  \end{center}
}




\graphicspath{ {./img/} }

% Capítulos
\chapterstyle{bianchi}
\newcommand{\capitulo}[2]{
	\setcounter{chapter}{#1}
	\setcounter{section}{0}
	\chapter*{#2}
	\addcontentsline{toc}{chapter}{#2}
	\markboth{#2}{#2}
}

% Apéndices
\renewcommand{\appendixname}{Apéndice}
\renewcommand*\cftappendixname{\appendixname}

\newcommand{\apendice}[1]{
	%\renewcommand{\thechapter}{A}
	\chapter{#1}
}

\renewcommand*\cftappendixname{\appendixname\ }

% Formato de portada
\makeatletter
\usepackage{xcolor}
\newcommand{\tutor}[1]{\def\@tutor{#1}}
\newcommand{\tutors}[1]{\def\@tutors{#1}}
\newcommand{\course}[1]{\def\@course{#1}}
\definecolor{cpardoBox}{HTML}{E6E6FF}
\def\maketitle{
  \null
  \thispagestyle{empty}
  % Cabecera ----------------
\noindent\includegraphics[width=\textwidth]{cabecera}\vspace{1cm}%
  \vfill
  % Título proyecto y escudo informática ----------------
  \colorbox{cpardoBox}{%
    \begin{minipage}{.8\textwidth}
      \vspace{.5cm}\Large
      \begin{center}
      \textbf{TFG del Grado en Ingeniería Informática}\vspace{.6cm}\\
      \textbf{\LARGE\@title{}}
      \end{center}
      \vspace{.2cm}
    \end{minipage}

  }%
  \hfill\begin{minipage}{.20\textwidth}
    \includegraphics[width=\textwidth]{escudoInfor}
  \end{minipage}
  
\begin{center}
\includegraphics[width=0.45\textwidth]{img/logoRBP.pdf}
\end{center}


  % Datos de alumno, curso y tutores ------------------
  \begin{center}%
  {%
    \noindent\LARGE
    Presentado por \@author{}\\ 
    en Universidad de Burgos --- \@date{}\\
    Tutor: \@tutor{}\\
    Tutor: \@tutors{}\\
  }%
  \end{center}%
  \null
  \cleardoublepage
  }
\makeatother


\newcommand{\nombre}{David Colmenero Guerra} %%% cambio de comando

% Datos de portada
\title{Sistema Domótico Inteligente \\Documentación Técnica}
\author{\nombre}
\tutor{Álvar Arnaiz-González}
\tutors{Alejandro Merino Gómez}
\date{\today}

\begin{document}

\maketitle



\cleardoublepage



%%%%%%%%%%%%%%%%%%%%%%%%%%%%%%%%%%%%%%%%%%%%%%%%%%%%%%%%%%%%%%%%%%%%%%%%%%%%%%%%%%%%%%%%



\frontmatter


\clearpage

% Indices
\tableofcontents

\clearpage

\listoffigures

\clearpage

\listoftables

\renewcommand{\lstlistlistingname}{Lista de códigos}
\lstlistoflistings

\clearpage

\mainmatter

\clearpage

\appendix

\apendice{Plan de Proyecto Software}

\section{Introducción}
En cualquier proyecto de este tipo es relevante incluir cierta información sobre la planificación, viabilidad y análisis de costes en el desarrollo del mismo con la finalidad de conseguir una visión de cómo ha ido evolucionando el proyecto y los ciclos de trabajo que se han llevado a cabo.

He de señalar que en la última tercera parte de los sprints, no siempre he contado con una línea de datos con la que poder replicar datos en GitHub, por lo que se han generado menos commits de los que se hubiera hecho si hubiera contado con medios a mi alcance. En cualquier caso, se ha intentado maximizar el número de puntos de salvado para que quedase constancia de la evolución del proyecto. Además, al redactar el proyecto en \LaTeX{} desde la plataforma de Overleaf no refleja el tiempo real invertido puesto que no se realizan cambios en el repositorio de forma automática, hasta que compilo, descargo y actualizo el repositorio manualmente.

En el primer apartado trataremos la~\textbf{planificación temporal}, donde podremos ver como han evolucionado los tiempos de trabajo durante las semanas en las que se ha trabajado en el proyecto. En el segundo apartado se reflejará un \textbf{estudio de viabilidad} sobre el proyecto, que a su vez incluye dos partes, la parte económica y la parte legal.

\section{Planificación temporal}
Al comenzar el proyecto se determinó que se utilizaría una metodología ágil para hacer el desarrollo del proyecto pero esta decisión ha terminado siendo un hándicap producido por la falta de documentación y la dudosa credibilidad de muchas páginas web y documentación que he encontrado, y se ha traducido en una gran inversión en tiempo.

Ante todo, se ha intentado seguir un mínimo de pautas~\textbf{Scrum}~\cite{manual:Scrum} pese a no existir un grupo de trabajo real con unas tareas diarias y roles definidos:
\begin{itemize}
    \item Las tareas fueron siempre semanales en forma de <<sprints>>.
    \item Al finalizar el sprint se hace la entrega del trabajo elaborado en la semana y se determinan las próximas tareas.
    \item Tras determinar las tareas se definen los <<milestones>> y los <<issues>>.
    \item Para hacer el seguimiento de las tareas se ha utilizado en tablero~\textbf{Kanban} de~\textbf{ZenHub}.
    \item Tras finalizar el sprint se puede comprobar el trabajo mediante los~\textit{gráficos burndown}.
\end{itemize}

Las reuniones de trabajo han sido consensuadas y planificadas para realizarlas los jueves de cada semana, con el siguiente resultado:

\begin{figure}
    \centering
    \includegraphics[width=0.9\textwidth]{img/BurnDown/1.PNG}
    \caption{Gráfico Burndown sprint1. } \label{BD1}
    \includegraphics[width=0.9\textwidth]{img/BurnDown/2.PNG}
    \caption{Gráfico Burndown sprint2. } \label{BD2}
\end{figure}

\subsection{Sprint 00 - 01/10/2020 - 08/10/2020}
En la primera reunión nos reunimos Álvar Arnaiz González y yo. En ella hice la propuesta de temática del TFG y, además expuse diferentes ideas y enfoques sobre el proyecto en la que se determinó a grandes rasgos la posible viabilidad del proyecto.
Los objetivos de este <<sprint>> introductorio fueron la búsqueda de repositorios y otros proyectos para tomar ideas o si era posible hacer algún fork. También, el estudio la estructura del proyecto y herramientas software e interfaces a utilizar.

\subsection{Sprint 01 - 09/10/2020 - 15/10/2020}
Tras determinar el enfoque final de proyecto, se consensuó que los objetivos de este sprint fueron la búsqueda de información relevante, otros proyectos y recursos que puedan servir de utilidad como pueden ser APIS. Se buscaron algunas páginas web para realizar pruebas de extracción de datos con <<web scraping>>~\footnote{Se trata brevemente en el punto 4 de la memoria} y se descompusieron las tareas en unidades pequeñas de trabajo para poder afrontarlas durante sprints.

En este sprint también generé el repositorio en GitHub para poder trabajar contra él y podemos ver las líneas de código utilizadas en el \href{https://github.com/davidelinformatico/TFG/issues/1}{primer \textit{issue}}.

\begin{figure}
    \centering
    \includegraphics[width=0.9\textwidth]{img/BurnDown/3.PNG}
    \caption{Gráfico Burndown sprint3. } \label{BD3}
    \includegraphics[width=0.9\textwidth]{img/BurnDown/4.PNG}
    \caption{Gráfico Burndown sprint4. } \label{BD4}
\end{figure}

\subsection{Sprint 02 - 16/10/2020 - 22/10/2020}
Los objetivos de este sprint fueron el buscar repositorios, APIS, servicios y tecnologías con las que darle forma al proyecto. Se decidió generar algún tipo de aplicación desde la que poder controlar la instalación haciéndola más amigable y con mejor capacidad de interacción. Se investigan opciones.

\item En el~\href{https://github.com/davidelinformatico/TFG/issues/2}{issue 2} se hizo una búsqueda por la web para buscar repositorios y proyectos sobre domótica
\item En el~\href{https://github.com/davidelinformatico/TFG/issues/3}{issue 3} buscamos algunas APIS para orientar el proyecto hacia el uso de APIS destinadas a ofrecer información que pueda servirnos y se realizaron algunas pruebas.
\item En el~\href{https://github.com/davidelinformatico/TFG/issues/4}{issue 4} se realizó un pequeño estudio de las tecnologías que podríamos utilizar, desde el procesador de textos, que finalmente se utilizó~\LaTeX{}, como la revisión de las normativas electrotécnicas que debía seguir en el proyecto y también se determinó la utilización de Telegram para interactuar con el Sistema Domótico.


\subsection{Sprint 03 - 23/10/2020 - 29/10/2020}
Los objetivos de este sprint fueron determinar los componentes hardware necesarios para completar el proyecto y se realiza la compra de material.
En todos los <<issues>> del <<\href{https://github.com/davidelinformatico/TFG/milestone/3?closed=1}{milestone 3}>> podemos ver múltiples enlaces, comparativas, justificaciones e imágenes en las que se ha basado toda la instalación física posterior.

\subsection{Sprint 04 - 30/10/2020 - 05/10/2020}
Se incorpora D.Alejandro Merino Gómez al que se le presenta el proyecto y se le concede acceso a los repositorios para que pueda co-tutorizar el proyecto.

Los objetivos de este <<milestone>> fueron todos de la parte física de la instalación:
\item Podemos ver en el \href{https://github.com/davidelinformatico/TFG/issues/15}{issue 15} que finalicé la tirada de cable junto a un diagrama explicativo del funcionamiento de un pulsador de 3 posiciones (Posición 1, Reposo y Posición 2. Teniendo en cuenta que las dos posiciones que dan continuidad al circuito tienen invertida la polaridad).
\item También vemos en el \href{https://github.com/davidelinformatico/TFG/issues/16}{issue 16} la presentación de la Raspberry Pi que utilizaremos, los conectores del tipo JST-XH que se han utilizado en el crimpado de los cables electrónicos, los cables finalizados con sus conectores y la disposición final de la Raspberry Pi en su ubicación final de la instalación.

\begin{figure}
    \centering
    \includegraphics[width=0.9\textwidth]{img/BurnDown/5.PNG}
    \caption{Gráfico Burndown sprint5. } \label{BD5}
    \includegraphics[width=0.9\textwidth]{img/BurnDown/6.PNG}
    \caption{Gráfico Burndown sprint6. } \label{BD6}
\end{figure}

\subsection{Sprint 05 - 06/11/2020 - 12/11/2020}
Se repasan los cambios propuestos y se incluyen otros nuevos sobre la parte física de la instalación. Se hace el seguimiento de la instalación y se comentan las fotos y el estado de la instalación.
Los objetivos de este sprint son todos aquellos que sean necesarios para la adecuación del software básico para comenzar el proyecto como la actualización del sistema operativo o instalación de software adicional. Además se deben valorar opciones para controlar los GPIO.

\item Vemos en los issues \href{https://github.com/davidelinformatico/TFG/issues/21}{21} y \href{https://github.com/davidelinformatico/TFG/issues/19}{19} las configuraciones básicas realizadas a nuestro Raspbian. Para explicar este proceso grabé dos vídeos, titulados \href{https://youtu.be/B8E6q1fLp7Q}{Primeros Pasos Raspberry Pi} y \href{https://youtu.be/Vz38sGYpcYQ}{Actualización de Raspberry Pi y configuración básica}, respectivamente.

\subsection{Sprint 06 - 13/11/2020 - 19/11/2020}
Se revisan los puntos anteriores y se compone un tablero de pruebas de forma que se puede controlar el encendido de una bombilla desde nuestra Raspberry Pi. También se graba un vídeo explicativo y se diseñan unos diagramas explicativos.

Podemos ver en el \href{https://github.com/davidelinformatico/TFG/issues/22}{issue 22} que se presenta un diagrama con la lista de los componentes que vamos a utilizar en el tablero de pruebas y un diagrama de como se compondrá el tablero. En este issue también grabé un vídeo para explicar el tablero de pruebas y decidí utilizar un efecto más conservador de cara al final del proyecto.

En el \href{https://github.com/davidelinformatico/TFG/issues/23}{issue 23} he explicado como se controlan los GPIO de nuestra Raspberry Pi desde su distribución Linux~\footnote{Raspbery pi (Raspbian OS).} desde Bash y desde Python.

\begin{figure}
    \centering
    \includegraphics[width=0.9\textwidth]{img/BurnDown/7.PNG}
    \caption{Gráfico Burndown sprint7. } \label{BD5}
    \includegraphics[width=0.9\textwidth]{img/BurnDown/8.PNG}
    \caption{Gráfico Burndown sprint8. } \label{BD6}
\end{figure}

\subsection{Sprint 07 - 20/11/2020 - 26/11/2020}
Se realiza una investigación sobre como se puede implantar el bot en nuestro proyecto, se realiza el primer código de pruebas y se presentan las primeras pruebas con el bot. Podemos ver en el \href{https://github.com/davidelinformatico/TFG/issues/13}{issue 13} un resumen de las pruebas que estuve realizando con los diferentes teclados de que dispone Telegram.

también, se proponen correcciones en la redacción.

\subsection{Sprint 08 - 27/11/2020 - 03/12/2020}
En el \href{https://github.com/davidelinformatico/TFG/milestone/8?closed=1}{milestone 8}, podemos ver que se incluyen varios issues:

\begin{itemize}
    \item Creación de scripts Bash para controlar el conjunto del Sistema Domótico inteligente.
    \item Obtención de los datos de geolocalización y cálculo de temperaturas.
    \item Generación del nuevo Cron tras integrar los scripts existentes.
\end{itemize}

De esta manera, se genera la primera automatización completa de la parte automática del proyecto utilizando CRON y lo pasamos del entorno de pruebas a un entorno de producción en fase experimental. En este momento únicamente tenemos control sobre la máquina de forma física o por VNC.
Por último, se presentan las primeras funcionalidades del bot, se proponen cambios y mejoras.


\subsection{Sprint 09 - 04/12/2020 - 10/12/2020}
En el \href{https://github.com/davidelinformatico/TFG/milestone/9?closed=1}{milestone 9} se proponen revisan las funcionalidades del Bot~\cite{misc:TelegramApi} de Telegram~\cite{misc:TelegramApp} que utilizaremos, se proponen cambios y mejoras en el código preexistente en fase de pruebas.

Finalmente se integra un bot de Telegram completamente funcional desde el que podremos lanzar órdenes a nuestro sistema domótico con los scripts existentes.

\begin{figure}
    \centering
    \includegraphics[width=0.9\textwidth]{img/BurnDown/9.PNG}
    \caption{Gráfico Burndown sprint9. } \label{BD9}
    \includegraphics[width=0.9\textwidth]{img/BurnDown/10.PNG}
    \caption{Gráfico Burndown sprint10. } \label{BD10}
\end{figure}

\subsection{Sprint 10 - 11/12/2020 - 17/12/2020}
En el \href{https://github.com/davidelinformatico/TFG/milestone/10?closed=1}{milestone 10} se ha aprovechado para optimizar las rutas de los archivos a los que se invoca desde el código en ejecución y se han modificado algunas salidas para que tengan un aspecto visual más atractivo, incluyendo tablas, emoticonos desde Unicode~\cite{misc:UnicodeWikipedia}.

Se comprueban las funcionalidades proponiendo cambios funcionales y de estilos de forma que se pueda interactuar contra el bot y se muestren los mensajes en un formato más atractivo. También se propone unificar las rutas de trabajo de los diferentes scripts.


\subsection{Sprint 11 - 18/12/2020 - 24/12/2020}
Se propone dividir el código por funcionalidades (obtención de información de la web, y por otro lado, modificación de archivos del sistema y se proponen cambios en la redacción a implementar en el sprint 12.

Finalmente, se ha conseguido generadr la división del código en 3 archivos principales:
\begin{enumerate}
    \item Toma de datos de la web y guardado en json.
    \item Lectura de parámetros personalizados.
    \item Grabado de CRON conforme a esas preferencias.
\end{enumerate}

De esta manera conseguimos minimizar las comunicaciones con el exterior y poder relanzar el código separado por funcionalidad lógica de forma que podamos relanzarlo de forma aislada en caso de necesitarlo. Un ejemplo puede ser cuando se modifica la hora de subida de las persianas, el módulo de generación de CRON leería el archivo de parámetros personalizados y generaría la configuración a partir de éstos.

\begin{figure}
    \centering
    \includegraphics[width=0.9\textwidth]{img/BurnDown/11.PNG}
    \caption{Gráfico Burndown sprint11. } \label{BD11}
\end{figure}

\subsection{Sprint 12 - 30/12/2020 - 06/01/2021}
Se finaliza la redacción de la memoria, antes de la revisión final del texto, y se comienza con los anexos. Además se produce depurado del código y se utiliza SonarCloud.

\subsection{Sprint 13 - 07/01/2020 - 13/01/2020}
Redacción del proyecto del TFG y grabado de vídeos.

\subsection{Sprint 14 - 14/01/2020 - 20/01/2020}
Edición de vídeos y entrega.


\section{Estudio de viabilidad}

\subsection{Viabilidad económica}

\subsection{Viabilidad legal}



\apendice{Especificación de Requisitos}
\section{Introducción}
En este punto se detallan los requisitos con que debe cumplir el proyecto y que determinarán restricciones, funcionalidades y comportamiento de éste. 

\section{Objetivos generales}
Los objetivos generales enmarcan la funcionalidad del proyecto, que se detallan a continuación:
\begin{itemize}
    \item Crear una plataforma que consiga aumentar confort, comodidad y seguridad dentro del domicilio.
    \item Generar un Sistema Domótico que nos permita automatizar parámetros de la vivienda así como poder ordenar acciones en tiempo real.
    \item Crear un Sistema Simulador de Presencia Domiciliaria de funcionamiento autónomo.
    \item Debe contar con una interfaz de comunicación multiplataforma para interactuar con el software.
\end{itemize}

\section{Catálogo de requisitos}
En este punto se desglosan los requisitos con que debe cumplir el proyecto, derivando de los anteriormente expuestos.

\subsection{\textbf{Requisitos funcionales}}

\begin{itemize}
    \item \textbf{RF-1 Obtención de datos:} El Sistema debe ser capaz de obtener datos necesarios para su correcto funcionamiento.
    \begin{itemize}
        \item \textbf{RF-1.1 Obtención de ubicación:} Debe ser capaz de determinar su ubicación geográfica.
        \item \textbf{RF-1.2 Obtención de posición solar:} Debe poder obtener información sobre la posición solar con respecto del planeta según su ubicación geográfica.
        \item \textbf{RF-1.3 Obtención de temperaturas:} Debe poder obtener información meteorológica de las próximas horas en la ubicación en la que se encuentra el Sistema.
        \item \textbf{RF-1.4 Datos bajo demanda:} El usuario debe poder solicitar la obtención de los datos bajo demanda.
    \end{itemize}

    \item \textbf{RF-2 Control de periféricos:} El Sistema debe poder controlar los periféricos que tiene conectados.
    \begin{itemize}
        \item \textbf{RF-2.1 Control automático:} El sistema debe poder controlar de forma autónoma los dispositivos conectados.
        \item \textbf{RF-2.2 Control instantáneo:} El usuario debe poder controlar los diferentes periféricos a placer.
        \item \textbf{RF-2.3 Control planificado:} El sistema debe poder planificar el control de los periféricos conectados.
        \item \textbf{RF-2.4 Parametrización:} El usuario debe poder parametrizar la automatización de los periféricos.
    \end{itemize}
    
    \item \textbf{RF-3 Información:} La información de que dispone el sistema siempre estará a disposición del usuario.
    \begin{itemize}
        \item \textbf{RF-3.1 Información de sistema:} El usuario podrá obtener información básica sobre el sistema.
        \item \textbf{RF-3.2. Información de planificación:} El usuario debe poder obtener información sobre la planificación de la automatización de los periféricos.
        \item \textbf{RF-3.3. Información parametrización:} El usuario debe poder obtener información sobre la parametrización de la automatización de los periféricos.
        \item \textbf{RF-3.4 Diagramas:} El usuario debe poder obtener un diagrama de temperaturas del día u otros días anteriores que estén archivados en la máquina a modo de registro.
    \end{itemize}
    
    \item \textbf{RF-4 Control máquina:} Se deben poder ejecutar algunas tareas básicas sobre la máquina que soporta el Sistema.
    \begin{itemize}
        \item \textbf{RF-4.1 Operaciones bajo demanda:} El usuario podrá apagar y reiniciar la máquina siempre que lo desee.
        \item \textbf{RF-4.2 Operaciones automáticas:} El sistema se reiniciará, al menos, una vez al día.
    \end{itemize}   
    
    \item \textbf{RF-5 Gestión de la máquina:} La máquina que soporta el sistema debe poder ser accesible por parte del usuario.
    \begin{itemize}
        \item \textbf{RF-5.1 Gestión local:} El usuario podrá gestionar el sistema Operativo de la máquina que soporta el Sistema de forma local.
        \item \textbf{RF-5.2 Gestión remota:} El usuario podrá gestionar el sistema Operativo de la máquina que soporta el Sistema de forma remota vía VNC.
    \end{itemize}
    
    \item \textbf{RF-6 Comunicación entre BackEnd y FrontEnd:} Debe existir un filtro en las comunicaciones.
    \begin{itemize}
        \item \textbf{RF-6.1 Selección de usuarios:} El propio usuario debe poder gestionar los usuarios que pueden interactuar con el Sistema.
        \item \textbf{RF-6.2 Rechazo de peticiones:} Los usuarios que no estén admitidos en la lista no podrán acceder a la comunicación con el Sistema.
        \item \textbf{RF-6.3 Interacción multiplataforma:} Debe poder interactuar mediante las aplicaciones móviles existentes de Telegram y desde cualquier navegador popularmente extendido como pueden ser Chrome, Edge o Firefox entre otros. 
    \end{itemize}
\end{itemize}

\subsection{\textbf{Requisitos no funcionales}}

\begin{itemize}
    \item \textbf{RNF-1 Escalabilidad:} Debe poder ampliarse fácilmente.
    \item \textbf{RNF-2 Eficiencia:} Debe minimizar la carga computacional para minimizar también el consumo energético del Sistema, además de permitir que la temperatura exterior incida en menos medida en las condiciones deseadas optimizando el consumo de recursos.
    \item \textbf{RNF-3 Rendimiento:} El sistema debe ser fluido y evitar cargas innecesarias.
    \item \textbf{RNF-4 Usabilidad:} El FrontEnd debe ser lo más usable posible, es decir, fácil de utilizar y aprender e intuitivo, y adaptado a las necesidades que pretende cubrir.
    \item \textbf{RNF-5 Disponibilidad:} El Sistema debe estar siempre en correcto funcionamiento.
    \item \textbf{RNF-6 Durabilidad:} El software debe poder funcionar correctamente durante un tiempo relativamente largo.
    \item \textbf{RNF-7 Capacidad:} Debe poder obtener la información necesaria y actuar conforme a lo que se espera de él.
    \item \textbf{RNF-8 Documentación:} Debe existir la suficiente documentación para poder implementar e interactuar con el Sistema.
    \item \textbf{RNF-9 Operabilidad:} Debe permitir manejar y controlar el Sistema según los requisitos funcionales.
    \item \textbf{RNF-10 Mantenibilidad:} Debe desarrollarse de tal manera que el mantenimiento sea lo más fácil y rápido posible.
    \item \textbf{RNF-11 Seguridad:} Todas las operaciones desde el FrontEnd deben ser seguras y estar cifradas.
    \item \textbf{RNF-12 Legibilidad:} El software debe ser fácilmente legible.
    \item \textbf{RNF-13 Extensibilidad:} El código debe ser fácilmente adaptable y reutilizable.
    \item \textbf{RNF-14 Liberación de código:} Debe disponer de un Sistema Operativo GNU y el código debe tener algún tipo de licencia GNU.
    \item \textbf{RNF-15 Respaldo documental:} Toda la instalación debe realizarse conforme a los estándares legales vigentes.
~\\~\\~\\~\\
\end{itemize}

\section{Especificación de requisitos}
La especificación de requisitos contempla los casos de uso contra nuestro código.
Además, también podemos ver el diagrama de los casos de uso, aunque lo he dividido en las figuras <<1/2>>~\ref{CDU1} y <<2/2>>~\ref{CDU2}

\begin{figure}[h]
\includegraphics[width=1.15\textwidth]{img/Diagramas/CasosDeUso.png}
\caption{Casos de uso 1/2.}\label{CDU1}
\includegraphics[width=0.85\textwidth]{img/Diagramas/CasosDeUsoBackEnd.png}
\caption{Casos de uso 2/2.}\label{CDU2}
\end{figure}

\subsection{Actores}
Los actores serán cada uno de los usuarios del Sistema que controlará el sistema desde el FrontEnd.

\footnotesize%%%%%%%%%%%  smaller font size %%%%%%%%
\begin{longtable}{>{\hspace{0pt}}m{0.182\linewidth}>{\hspace{0pt}}m{0.758\linewidth}}
\hline
\rowcolor[rgb]{0.937,0.937,0.937} \multicolumn{1}{|>{\hspace{0pt}}m{0.182\linewidth}|}{\textbf{CU-01}} & \multicolumn{1}{>{\hspace{0pt}}m{0.758\linewidth}|}{\textbf{Obtención de posición solar}} \endfirsthead 
\hline
\textbf{Versión} & 1.0 \\
\rowcolor[rgb]{0.937,0.937,0.937} \textbf{Actor} & Usuario \\
\textbf{Requisitos asociados} & RF-1, RF-1.1, RF-1.2, RF-1.3, RF-1.4 \\
\rowcolor[rgb]{0.937,0.937,0.937} \textbf{Descripción} & Permite al usuario obtener los datos necesarios para parametrizar el Sistema Domótico Inteligente. \\
\textbf{Precondición} & \begin{tabular}{@{\labelitemi\hspace{\dimexpr\labelsep+0.5\tabcolsep}}l}Al ser un proceso principalmente automático, únicamente se~\end{tabular}\par{}~ ~ requiere una línea de datos con acceso a Internet.\par\par{}\begin{tabular}{@{\labelitemi\hspace{\dimexpr\labelsep+0.5\tabcolsep}}l}En el caso de hacer la petición el usuario, también necesita~\end{tabular}\par{}~ ~ ser uno de los usuarios autorizados. \\
\rowcolor[rgb]{0.937,0.937,0.937} \textbf{Acciones} & \begin{tabular}{@{\labelitemi\hspace{\dimexpr\labelsep+0.5\tabcolsep}}>{\cellcolor[rgb]{0.937,0.937,0.937}}l}El usuario solicita la obtención inmediata de los datos~\end{tabular}\par{}necesarios para realizar la próxima programación.\par\par{}\begin{tabular}{@{\labelitemi\hspace{\dimexpr\labelsep+0.5\tabcolsep}}>{\cellcolor[rgb]{0.937,0.937,0.937}}l}El programa llama a las APIS de geolocalización,~\end{tabular}\par{}astrológicos y meteorológicos. \\
\textbf{Postcondición} & El bot lanza los scripts de recopilación de datos y almacena los datos. \\
\rowcolor[rgb]{0.937,0.937,0.937} \textbf{Excepciones} & Si no puede recopilar los datos envía mensaje al usuario. \\
\textbf{Importancia} & Alta \\\hline\\
\caption{CU-01 - Obtención de datos}\\ 
\end{longtable}


\begin{longtable}{>{\hspace{0pt}}m{0.278\linewidth}>{\hspace{0pt}}m{0.662\linewidth}}
\hline
\rowcolor[rgb]{0.937,0.937,0.937} \multicolumn{1}{|>{\hspace{0pt}}m{0.278\linewidth}|}{\textbf{CU-02}} & \multicolumn{1}{>{\hspace{0pt}}m{0.662\linewidth}|}{\textbf{Control de periféricos}} \endfirsthead 
\hline
\textbf{Versión} & 1.0 \\
\rowcolor[rgb]{0.937,0.937,0.937} \textbf{Actor} & Usuario \\
\textbf{Requisitos asociados} & RF-2, RF-2.1, RF-2.2, RF-2.3, RF-2.4 \\
\rowcolor[rgb]{0.937,0.937,0.937} \textbf{Descripción} & Permite controlar los periféricos \\
\textbf{Precondición} & \begin{tabular}{@{\labelitemi\hspace{\dimexpr\labelsep+0.5\tabcolsep}}l}Deben existir periféricos y estar configurados en el~\end{tabular}\par{}archivo al efecto.\par\par{}\begin{tabular}{@{\labelitemi\hspace{\dimexpr\labelsep+0.5\tabcolsep}}l}El usuario debe estar acreditado.\end{tabular} \\
\rowcolor[rgb]{0.937,0.937,0.937} \textbf{Acciones} & \begin{tabular}{@{\labelitemi\hspace{\dimexpr\labelsep+0.5\tabcolsep}}>{\cellcolor[rgb]{0.937,0.937,0.937}}l}El usuario puede ordenar controlar un elemento.\end{tabular} \\
\textbf{Postcondición} & El~periférico cambia de estado. \\
\rowcolor[rgb]{0.937,0.937,0.937} \textbf{Excepciones} & Si no puede realizar la acción envía mensaje al usuario. \\
\textbf{Importancia} & Media \\
\hline\\
\caption{CU-02 - Control de periféricos.}~\\~\\~\\~\\
\end{longtable}


\begin{longtable}{>{\hspace{0pt}}m{0.273\linewidth}>{\hspace{0pt}}m{0.668\linewidth}}\hline
\rowcolor[rgb]{0.937,0.937,0.937} \multicolumn{1}{|>{\hspace{0pt}}m{0.273\linewidth}|}{\textbf{CU-03}} & \multicolumn{1}{>{\hspace{0pt}}m{0.668\linewidth}|}{\textbf{Parametrización~de periféricos}} \endfirsthead 
\hline
\textbf{Versión} & 1.0 \\
\rowcolor[rgb]{0.937,0.937,0.937} \textbf{Actor} & Usuario \\
\textbf{Requisitos asociados} & RF-2, RF-2.4 \\
\rowcolor[rgb]{0.937,0.937,0.937} \textbf{Descripción} & Permite parametrizar el control automático \\
\textbf{Precondición} & \begin{tabular}{@{\labelitemi\hspace{\dimexpr\labelsep+0.5\tabcolsep}}l}Deben existir periféricos y estar configurados en el~\end{tabular}\par{}archivo al efecto.\par\par{}\begin{tabular}{@{\labelitemi\hspace{\dimexpr\labelsep+0.5\tabcolsep}}l}El usuario debe estar acreditado.\end{tabular}\par\par{}\begin{tabular}{@{\labelitemi\hspace{\dimexpr\labelsep+0.5\tabcolsep}}l}Deben introducirse los parámetros correctos.\end{tabular} \\
\rowcolor[rgb]{0.937,0.937,0.937} \textbf{Acciones} & El usuario parametriza la automatización de un periférico. \\
\textbf{Postcondición} & El~periférico cambiará de estado. \\
\rowcolor[rgb]{0.937,0.937,0.937} \textbf{Excepciones} & Si no puede realizar la acción envía mensaje al usuario. \\
\textbf{Importancia} & Alta \\
\hline
\\\caption{CU-03 - Parametrización de periféricos}
\end{longtable}

\begin{longtable}{>{\hspace{0pt}}m{0.207\linewidth}>{\hspace{0pt}}m{0.733\linewidth}}
\hline
\rowcolor[rgb]{0.937,0.937,0.937} \multicolumn{1}{|>{\hspace{0pt}}m{0.207\linewidth}|}{\textbf{CU-04}} & \multicolumn{1}{>{\hspace{0pt}}m{0.733\linewidth}|}{\textbf{Solicitar Información}} \endfirsthead 
\hline
\textbf{Versión} & 1.0 \\
\rowcolor[rgb]{0.937,0.937,0.937} \textbf{Actor} & Usuario \\
\textbf{Requisitos \mbox{asociados}} & RF-3, RF-3.1, RF-3.2, RF-3.3, RF-3.4 \\
\rowcolor[rgb]{0.937,0.937,0.937} \textbf{Descripción} & Permite obtener información del sistema, de automatización, de la parametrización. \\
\textbf{Precondición} & \begin{tabular}{@{\labelitemi\hspace{\dimexpr\labelsep+0.5\tabcolsep}}l}El usuario debe estar acreditado.\\Debe existir la información.\end{tabular} \\
\rowcolor[rgb]{0.937,0.937,0.937} \textbf{Acciones} & Lectura de información. \\
\textbf{Postcondición} & Se envía información del sistema. \\
\rowcolor[rgb]{0.937,0.937,0.937} \textbf{Excepciones} & Si no puede realizar la acción envía mensaje al usuario. \\
\textbf{Importancia} & Baja \\
\hline
\\ \caption{CU-04 Solicitar Información}\\ 
\end{longtable}

\begin{longtable}{>{\hspace{0pt}}m{0.278\linewidth}>{\hspace{0pt}}m{0.662\linewidth}}
\hline
\rowcolor[rgb]{0.937,0.937,0.937} \multicolumn{1}{|>{\hspace{0pt}}m{0.278\linewidth}|}{\textbf{CU-05}} & \multicolumn{1}{>{\hspace{0pt}}m{0.662\linewidth}|}{\textbf{Solicitar Información de sistema}} \endfirsthead 
\hline
\textbf{Versión} & 1.0 \\
\rowcolor[rgb]{0.937,0.937,0.937} \textbf{Actor} & Usuario \\
\textbf{Requisitos asociados} & RF-3, RF-3.1 \\
\rowcolor[rgb]{0.937,0.937,0.937} \textbf{Descripción} & Permite obtener información del sistema \\
\textbf{Precondición} & \begin{tabular}{@{\labelitemi\hspace{\dimexpr\labelsep+0.5\tabcolsep}}l}El usuario debe estar acreditado.\\Debe existir el diagrama.\end{tabular} \\
\rowcolor[rgb]{0.937,0.937,0.937} \textbf{Acciones} & Se envía información del sistema. \\
\textbf{Postcondición} & Se enviará la información solicitada. \\
\rowcolor[rgb]{0.937,0.937,0.937} \textbf{Excepciones} & Si no puede realizar la acción envía mensaje al usuario. \\
\textbf{Importancia} & Baja \\
\hline
\\\caption{CU-05 Solicitar Información de sistema} 
\end{longtable}

\begin{longtable}{>{\hspace{0pt}}m{0.2\linewidth}>{\hspace{0pt}}m{0.741\linewidth}}
\hline
\rowcolor[rgb]{0.937,0.937,0.937} \multicolumn{1}{|>{\hspace{0pt}}m{0.2\linewidth}|}{\textbf{CU-06}} & \multicolumn{1}{>{\hspace{0pt}}m{0.741\linewidth}|}{\textbf{Solicitar Información de planificación}} \endfirsthead 
\hline
\textbf{Versión} & 1.0 \\
\rowcolor[rgb]{0.937,0.937,0.937} \textbf{Actor} & Usuario \\
\textbf{Requisitos \mbox{asociados}} & RF-3, RF-3.1 \\
\rowcolor[rgb]{0.937,0.937,0.937} \textbf{Descripción} & Permite obtener información sobre la planificación del funcionamiento de los periféricos. \\
\textbf{Precondición} & \begin{tabular}{@{\labelitemi\hspace{\dimexpr\labelsep+0.5\tabcolsep}}l}El usuario debe estar acreditado.\\Debe existir el diagrama.\end{tabular} \\
\rowcolor[rgb]{0.937,0.937,0.937} \textbf{Acciones} & Lectura de información. \\
\textbf{Postcondición} & Se envía información del sistema \\
\rowcolor[rgb]{0.937,0.937,0.937} \textbf{Excepciones} & Si no puede realizar la acción envía mensaje al usuario. \\
\textbf{Importancia} & Baja \\
\hline
\caption{CU-06 Solicitar Información de planificación}\\ 
\end{longtable}


\begin{longtable}{>{\hspace{0pt}}m{0.19\linewidth}>{\hspace{0pt}}m{0.75\linewidth}}
\hline
\rowcolor[rgb]{0.937,0.937,0.937} \multicolumn{1}{|>{\hspace{0pt}}m{0.19\linewidth}|}{\textbf{CU-07}} & \multicolumn{1}{>{\hspace{0pt}}m{0.75\linewidth}|}{\textbf{Solicitar Información de parametrización}} \endfirsthead 
\hline
\textbf{Versión} & 1.0 \\
\rowcolor[rgb]{0.937,0.937,0.937} \textbf{Actor} & Usuario \\
\textbf{Requisitos \mbox{asociados}} & RF-3, RF-3.1 \\
\rowcolor[rgb]{0.937,0.937,0.937} \textbf{Descripción} & Permite obtener información sobre los parámetros personalizados introducidos por el usuario. \\
\textbf{Precondición} & \begin{tabular}{@{\labelitemi\hspace{\dimexpr\labelsep+0.5\tabcolsep}}l}El usuario debe estar acreditado.\\Debe existir el diagrama.\end{tabular} \\
\rowcolor[rgb]{0.937,0.937,0.937} \textbf{Acciones} & Lectura de información. \\
\textbf{Postcondición} & Se envia información de parametrización. \\
\rowcolor[rgb]{0.937,0.937,0.937} \textbf{Excepciones} & Si no puede realizar la acción envía mensaje al usuario. \\
\textbf{Importancia} & Baja \\
\hline
\caption{CU-07 Solicitar Información de parametrización}\\ 
\end{longtable}


\begin{longtable}{>{\hspace{0pt}}m{0.19\linewidth}>{\hspace{0pt}}m{0.75\linewidth}}
\hline
\rowcolor[rgb]{0.937,0.937,0.937} \multicolumn{1}{|>{\hspace{0pt}}m{0.19\linewidth}|}{\textbf{CU-08}} & \multicolumn{1}{>{\hspace{0pt}}m{0.75\linewidth}|}{\textbf{Solicitar Diagramas informativos}} \endfirsthead 
\hline
\textbf{Versión} & 1.0 \\
\rowcolor[rgb]{0.937,0.937,0.937} \textbf{Actor} & Usuario \\
\textbf{Requisitos \mbox{asociados}} & RF-3, RF-3.1 \\
\rowcolor[rgb]{0.937,0.937,0.937} \textbf{Descripción} & Permite obtener información sobre los parámetros personalizados introducidos por el usuario. \\
\textbf{Precondición} & \begin{tabular}{@{\labelitemi\hspace{\dimexpr\labelsep+0.5\tabcolsep}}l}El usuario debe estar acreditado.\\Debe existir el diagrama.\end{tabular} \\
\rowcolor[rgb]{0.937,0.937,0.937} \textbf{Acciones} & Busca el diagrama elegido. \\
\textbf{Postcondición} & Envía el diagrama. \\
\rowcolor[rgb]{0.937,0.937,0.937} \textbf{Excepciones} & Si no puede realizar la acción envía mensaje al usuario. \\
\textbf{Importancia} & Baja \\
\hline
\caption{CU-08 Solicitar Diagramas informativos}\\ 
\end{longtable}


\begin{longtable}{>{\hspace{0pt}}m{0.267\linewidth}>{\hspace{0pt}}m{0.674\linewidth}}
\hline
\rowcolor[rgb]{0.937,0.937,0.937} \multicolumn{1}{|>{\hspace{0pt}}m{0.267\linewidth}|}{\textbf{CU-09}} & \multicolumn{1}{>{\hspace{0pt}}m{0.674\linewidth}|}{\textbf{Operaciones bajo demanda}} \endfirsthead 
\hline
\textbf{Versión} & 1.0 \\
\rowcolor[rgb]{0.937,0.937,0.937} \textbf{Actor} & Usuario \\
\textbf{Requisitos asociados} & RF-4, RF-4.1 \\
\rowcolor[rgb]{0.937,0.937,0.937} \textbf{Descripción} & Permite ejecutar apagado y reinicio de la máquina a placer. \\
\textbf{Precondición} & El usuario debe estar acreditado. \\
\rowcolor[rgb]{0.937,0.937,0.937} \textbf{Acciones} & Apaga o Reinicia el Sistema. \\
\textbf{Postcondición} & La máquina ejecuta la órden enviada. \\
\rowcolor[rgb]{0.937,0.937,0.937} \textbf{Excepciones} & Si no puede realizar la acción envía mensaje al usuario. \\
\textbf{Importancia} & Baja \\
\hline
\\\caption{CU-09 Operaciones bajo demanda}\\ 
\end{longtable}

\begin{longtable}{>{\hspace{0pt}}m{0.24\linewidth}>{\hspace{0pt}}m{0.701\linewidth}}
\hline
\rowcolor[rgb]{0.937,0.937,0.937} \multicolumn{1}{|>{\hspace{0pt}}m{0.24\linewidth}|}{\textbf{CU-10}} & \multicolumn{1}{>{\hspace{0pt}}m{0.701\linewidth}|}{Gestión local de la máquina} \endfirsthead 
\hline
\textbf{Versión} & 1.0 \\
\rowcolor[rgb]{0.937,0.937,0.937} \textbf{Actor} & Usuario \\
\textbf{Requisitos \mbox{asociados}} & RF-5, RF-5.1 \\
\rowcolor[rgb]{0.937,0.937,0.937} \textbf{Descripción} & Permite administrar la máquina que soporta el sistema de forma local \\
\textbf{Precondición} & El usuario existir en el sistema. \\
\rowcolor[rgb]{0.937,0.937,0.937} \textbf{Acciones} & Administrar el sistema. \\
\textbf{Postcondición} & Si conoce las credenciales podrá administrar el sistema.  \\
\rowcolor[rgb]{0.937,0.937,0.937} \textbf{Excepciones} & Si no puede realizar la acción envía mensaje al usuario. \\
\textbf{Importancia} & Alta \\
\hline
\\\caption{CU-10 Gestión local de la máquina}\\ 
\end{longtable}

\begin{longtable}{>{\hspace{0pt}}m{0.232\linewidth}>{\hspace{0pt}}m{0.708\linewidth}}
\hline
\rowcolor[rgb]{0.937,0.937,0.937} \multicolumn{1}{|>{\hspace{0pt}}m{0.232\linewidth}|}{\textbf{CU-11}} & \multicolumn{1}{>{\hspace{0pt}}m{0.708\linewidth}|}{Gestión remota de la máquina} \endfirsthead 
\hline
\textbf{Versión} & 1.0 \\
\rowcolor[rgb]{0.937,0.937,0.937} \textbf{Actor} & Usuario \\
\textbf{Requisitos \mbox{asociados}} & RF-5, RF-5.2 \\
\rowcolor[rgb]{0.937,0.937,0.937} \textbf{Descripción} & Permite administrar la máquina que soporta el sistema de forma remota~ \\
\textbf{Precondición} & El usuario debe existir en el sistema, VNC debe estar funcionando y conocer las credenciales. \\
\rowcolor[rgb]{0.937,0.937,0.937} \textbf{Acciones} & Administrar el sistema. \\
\textbf{Postcondición} & Si conoce las credenciales podrá administrar el sistema. \\
\rowcolor[rgb]{0.937,0.937,0.937} \textbf{Excepciones} & Si no puede realizar la acción envía mensaje al usuario. \\
\textbf{Importancia} & Alta \\
\hline
\\\caption{CU-11 Gestión remota de la máquina}\\ 
\end{longtable}

\begin{longtable}{>{\hspace{0pt}}m{0.179\linewidth}>{\hspace{0pt}}m{0.762\linewidth}}
\hline
\rowcolor[rgb]{0.937,0.937,0.937} \multicolumn{1}{|>{\hspace{0pt}}m{0.179\linewidth}|}{\textbf{CU-12}} & \multicolumn{1}{>{\hspace{0pt}}m{0.762\linewidth}|}{Selección de usuarios} \endfirsthead 
\hline
\textbf{Versión} & 1.0 \\
\rowcolor[rgb]{0.937,0.937,0.937} \textbf{Actor} & Usuario \\
\textbf{Requisitos asociados} & RF-6, RF-6.1, RF-6.2 \\
\rowcolor[rgb]{0.937,0.937,0.937} \textbf{Descripción} & Se deben poder escoger los usuarios que podrán interactuar con el bot, el resto quedan descartados. \\
\textbf{Precondición} & Los usuarios seleccionados deben existir en el archivo creado para ello. \\
\rowcolor[rgb]{0.937,0.937,0.937} \textbf{Acciones} & Permite interactuar con el bot o rechaza las peticiones. \\
\textbf{Postcondición} & Si está acreditado podrá interactuar con el sistema. \\
\rowcolor[rgb]{0.937,0.937,0.937} \textbf{Excepciones} & Si no puede realizar la acción envía mensaje al usuario. \\
\textbf{Importancia} & Alta \\
\hline
\\\caption{CU-12 Selección de usuarios}\\ 
\end{longtable}


\begin{longtable}{>{\hspace{0pt}}m{0.213\linewidth}>{\hspace{0pt}}m{0.727\linewidth}}
\hline
\rowcolor[rgb]{0.937,0.937,0.937} \multicolumn{1}{|>{\hspace{0pt}}m{0.213\linewidth}|}{\textbf{CU-13}} & \multicolumn{1}{>{\hspace{0pt}}m{0.727\linewidth}|}{Interacción Multiplaforma} \endfirsthead 
\hline
\textbf{Versión} & 1.0 \\
\rowcolor[rgb]{0.937,0.937,0.937} \textbf{Actor} & Usuario \\
\textbf{Requisitos \mbox{asociados}} & RF-6, RF-6.1, RF-6.2, RF-6.3 \\
\rowcolor[rgb]{0.937,0.937,0.937} \textbf{Descripción} & Debe poder interactuar desde un usuario existente y desde cualquier plataforma. \\
\textbf{Precondición} & El usuario debe existir y estar acreditado. \\
\rowcolor[rgb]{0.937,0.937,0.937} \textbf{Acciones} & Permite interactuar con el bot o rechaza las peticiones. \\
\textbf{Postcondición} & Podrá interactuar con el sistema. \\
\rowcolor[rgb]{0.937,0.937,0.937} \textbf{Excepciones} & Si no puede realizar la acción envía mensaje al usuario. \\
\textbf{Importancia} & Alta \\
\hline
\\\caption{CU-13 Interacción Multiplataforma}\\ 
\end{longtable}



\normalsize

























\apendice{Especificación de diseño}

\section{Introducción}

\section{Diseño de datos}

\section{Diseño procedimental}

\section{Diseño arquitectónico}

\begin{comment}
    como se han repartido los archivos
\end{comment}


\apendice{Documentación técnica de programación}

\section{Introducción}

\section{Estructura de directorios}
Hay que conocer las diferentes partes del proyecto y tener en cuenta que son completamente diferentes aunque en este proyecto se han integrado para poder hacer uso del sistema domótico. Podemos ver una primera distribución general en la imagen~\ref{Img:PartesProyecto}. En ella podemos diferenciar tres partes principales:

Tuve un problema generalizado en el código puesto que obtenía las rutas a los archivos de forma unitaria desde el archivo en ejecución. Este es el código utilizado para obtener la ruta en Python:
\begin{lstlisting}[language=json,firstnumber=0]
{
    #!/usr/bin/env python3
    #Calculamos ruta
    ruta=os.getcwd().split('/')
}
\end{lstlisting}

Éste es el código utilizado para obtener la ruta en Bash:
\begin{lstlisting}[language=json,firstnumber=0]
{
    #!/bin/bash
    path=$(pwd)
}
\end{lstlisting}

\begin{itemize}
    \item La parte de toma de datos.
    \item La parte de mensajería.
    \item La parte física de nuestra instalación domótica, incluyendo la Raspberrry Pi.
\end{itemize}

\begin{figure}
    \centering
    \includegraphics[width=0.7\textwidth]{img/PartesProyecto.png}
    \caption{Partes del proyecto. } \label{Img:PartesProyecto}
\end{figure}

Por ello, el código se ha estructurado en tres partes:
\begin{itemize}
    \item La carpeta \textbf{auto}, que permite obtener la información de las APIs seleccionadas de Internet.
    \item La carpeta \textbf{bot}, donde almacenamos la configuración de nuestro bot.
    \item La carpeta \textbf{control}, donde almacenamos los scripts de control de la instalación domótica.
\end{itemize}

En todas ellas, los scripts, tanto de Bash como de Python llevan la cabecera específica para el tipo de código contenido en el archivo:

\begin{lstlisting}[language=json,firstnumber=0]
{
    #!/usr/bin/env python3
}
\end{lstlisting}

Éste es el código utilizado para obtener la ruta en Bash:
\begin{lstlisting}[language=json,firstnumber=0]
{
    #!/bin/bash
}
\end{lstlisting}


Para los archivos Python, la sentencia <<#!/usr/bin/env python3>> y para los archivos Bash, la sentencia <<#!/bin/bash>>. He introducido esta cabecera porque he tenido problemas al lanzar las instrucciones desde otro script.

\subsection{Carpeta auto}
Al comienzo del proyecto esperé encontrar alguna web desde la que poder recoger la información haciendo webscraping pero tras valorarlo detenidamente en las tutorías prefería a buscar una API (ver concepto~\ref{concepto:API}). Uno de los motivos es que una web es más susceptible de sufrir cambios que una API que se ha predispuesto para una función.
Finalmente, dispuse el proyecto para poder funcionar con dos APIs públicas y gratuitas para dotar a nuestro sistema domótico de autonomía.
Las siguientes fases de la parte de toma de datos, se albergan en la carpeta auto ya que es un proceso automatizado que permite al sistema obtener los datos de forma desatendida y continua.
El proceso de automatización comprende varias fases:
\begin{enumerate}
    \item La fase de recopilación de datos.
    \item La fase de lectura local de parámetros.
    \item El cocinado de los datos.
    \item El grabado de datos en el sistema local.
\end{enumerate}
La fase de recopilación de datos llama a la API de geolocalización y nos devuelve, entre otros, los parámetros de nuestra ubicación en formato de latitud y longitud. Posteriormente, estos datos de ubicación son necesarios para incorporarlos en la consulta de la siguiente API~\ref{4:API_Tiempo} y obtener la información de la salida y puesta del sol así como la información de las temperaturas del día siguiente en dicha ubicación. Con esta recopilación de datos tenemos la base para poder determinar el comportamiento autónomo que necesitamos por lo que almacenaremos esta información en un archivo de datos.

El siguiente paso es leer las preferencias del usuario, que se almacenan en el archivo de condicionantes. Y, una vez tengamos estos datos, podemos pasar a generar el Cron para controlar de forma automática toda la instalación.

Estas fases se han confeccionado al final del proyecto puesto que al principio se desarrolló de forma unitaria pero surgió el problema de que, cada vez que queríamos hacer una modificación en el sistema domótico de cualquier índole, teníamos que hacer una petición nueva a las APIs y generar de nuevo un archivo Cron con los parámetros del usuario <<al vuelo>> para conseguir regenerar la automatización con los parámetros nuevos. Este sistema funcionaba correctamente pero era poco eficiente puesto que que se multiplican las operaciones en función de las veces que se quiera modificar algún parámetro.
Únicamente he permitido realizar dos salidas "al vuelo" tras obtener los datos, estos son: el grabado de los datos recién obtenidos y la gráfica de las temperaturas conforme a dichos datos.

He tenido algunos problemas al realizar varias peticiones a las APIS ya que, por falta de permisos, no podía sobreescribir los archivos ni las imágenes. Para subsanarlo, incluí una restricción para modificar los permisos de los archivos de forma que únicamente pueden acceder el propietario y el grupo.

\subsection{Carpeta control}
La carpeta control alberga aquellos scripts bash que controlan los elementos domóticos. Este punto únicamente me dio problemas a la hora de implantarlo desde el bot, que se subsanó con la librería os.

\subsection{Carpeta bot}
La carpeta bot contiene varios archivos, desde el archivo de control del bot hasta otros que contienen la mayoría de funcionalidades de éste. En primera instancia, se dispuso todo el código dentro del archivo de control del bot, pero por legibilidad y facilidad de mantenimiento se extrajeron las funciones.
En este punto, lo que más problemas me ha dado han sido los teclados. Como comportamiento de los teclados es diferente entre ellos tendría que dividir las opciones del bot entre dos tipos de teclados diferentes. Finalmente, opté por no incluir estos teclados ya que no incluye funcionalidad alguna y disminuirían la legibilidad y facilidad de uso del bot. Además, para aumentar la usabilidad del bot, se ha dispuesto un menú que podemos ver al introducir el carácter "/" además de la posibilidad de utilizar la primera letra del comando siempre que no tengamos que acompañarlo de parámetros.

\subsection{Configuración del bot como servicio}
El bot se puede lanzar como si fuera un script más pero en este caso es necesario incorporarlo como servicio. Esta decisión tiene beneficios como la posibilidad de que se inicie con el sistema o que podamos lanzarlo o pararlo desde cualquier ubicación del SO sin tener que recordar la ubicación con la sentencia al efecto de <<\textbf{sudo service bot start/stop}>>. Pero, una vez que el demonio está corriendo, tenemos que tener en cuenta de que ya está levantada la instancia del bot, por lo que tendremos que detenerla a la hora de hacer algún tipo de modificación así como recordar que debemos levantarlo una vez terminemos.



\section{Manual del programador}

\section{Compilación, instalación y ejecución del proyecto}

\section{Pruebas del sistema}



\apendice{Documentación de usuario}

\section{Introducción}
El usuario es aquella persona destinada a utilizar el Sistema generado.

\section{Requisitos de la instalación física}
Se incluye la instalación en este punto porque se entiende que cualquiera que disponga el código puede correrlo en su propia instalación realizando la instalación física.

\subsection{Instalación física}
Para poder realizar la instalación física de nuestra vivienda necesitamos el material detallado en la tabla~\ref{tab:CosteHW}, ya que en este punto se cuenta con que la instalación física está completa para que el sistema pueda cumplir su cometido. Ahora utilizaremos los siguientes materiales:
\begin{itemize}
    \item Bobina de Cable UTP5e.
    \item Placa Board.
    \item Relés con las características: [ 10A 250VAC 10A 125VAC ]~[ 10A 30VDC 10A 28VDC ].
    \item Cables de electrónica.
    \item Conectores JST-XH.
\end{itemize}

Además requeriremos de crimpadora JST-XH y destornilladores de estrella y plano para trabajar con las clemas eléctricas (elementos de interconexión de conductores) del domicilio.

También necesitamos otros elementos de uso no específico para este proyecto pero que resultan necesarios para que el software funcione correctamente~\footnote{El software puede funcionar correctamente también vía Ethernet.} como son el Dongle WiFi y una línea de datos con salida a Internet.

\subsection{Tirada de cable}
Para realizar la tirada de cable necesitaremos hacer algunos cálculos y tener claras algunas consideraciones:

\subsubsection{Consideraciones}
Debemos tener claras las características eléctricas que soporta cada uno de los pines GPIO de nuestra Raspberry Pi. Cada uno de los pines, soporta 3,3VDC~\footnote{VDC:Voltage Direct Current.} y 16 mA, excepto algunos pines de alimentación, que cuentan con hasta 5VDC.
Hay que comprobar que cada uno de los hilos del cable UTP5e que vamos a instalar soporta la señal que vamos a transmitir por el cable aunque según la norma TIA/EIA568, en el punto A.3 especifica que deben pasar pruebas de hasta 500VDC y un mínimo entre dos cables de 100M\si{\ohm}. Además, también se especifica que debe contar con un AWG~\cite{wiki:DefAWG} de, entre 22 y 24, lo que se traduce en que soporta 0.577A-0.92A~\cite{wiki:TablaAWG}, respectivamente.

\subsubsection{Cálculos necesarios}
Debemos calcular si podemos introducir nuestros cables por los tubos con seguridad. Esto viene recogido en el REBT, más concretamente en apartado ITC\_BT\_21 de éste.

Sabiendo que nuestro cable UTP tiene unos 5mm de diámetro de sección, que la cara interna del tubo es de 17,8mm y que cada cable eléctrico tiene un área de 1,5mm$^{2}$ podemos calcular el área total ocupada y el tubo por el que debería hacerse la instalación con la seguridad requerida por norma:
\begin{equation}
\centering
\pi \cdot D^2/4 >= FactorCorrector \cdot (\sum(NumeroDeCabes \cdot \pi \cdot DiametroCable/4))
\end{equation}\label{E1}

Si realizamos el cálculo, observamos que la suma del área que ocupan los tres cables de 1,5mm$^{2}$ con el área del UTP (19,63mm$^{2}$) hace una ocupación total de 24,13mm$^{2}$.

Por otro lado, tenemos que el tubo que alberga los cables es de 20mm de diámetro de sección por la parte exterior del tubo y 17,80mm por la parte interna. Si calculamos el área interna obtenemos 248,85mm$^{2}$.
Según norma debemos ocupar la tercera parte del tubo. Para comprobar que el tubo puede contener los cables con seguridad, debemos calcular si la multiplicación del factor de corrección (en este caso de \textbf{3} por estar en una canalización empotrada de una vivienda), por la suma de las áreas que ocupan cada uno de los cables: 
\begin{equation}
\centering
3 \cdot [(3 \cdot 1,5)+(1 \cdot 19,63)]=72,40
\end{equation}\label{E2}

Ahora, debemos hacer el mismo cálculo del área de la cara interna del tubo:
\begin{equation}
\centering
\pi \cdot (17,8)/4
\end{equation}\label{E3}
Finalmente, vemos que~\ref{E3} es mayor que~\ref{E2} de forma que se cumple que el área ocupada (multiplicada por el factor de corrección) es inferior al área total, de modo que podemos albergar nuestro UTP en los tubos sin problema.

Por facilitar futuras instalaciones, figuro el cálculo para el mismo tubo también con cables de 2,5mm$^{2}$ de área en la imagen~\ref{Img:Calculo}:

\begin{figure}[h]
    \centering
    \includegraphics[width=0.9\textwidth]{img/Diagramas/calculo cables.jpeg}
    \caption{Cálculos para introducir cableado en tubo.} \label{Img:Calculo}
\end{figure}

Tras finalizar los cálculos, se hacen las tiradas de cableado y se crimpan los extremos. Podemos ver en la imagen~\ref{Img:CajaDerivacion} una caja de derivación donde ya tenemos albergada la placa de relés.~\\~\\

\begin{figure}[h]
    \centering
    \includegraphics[width=0.9\textwidth]{img/fotos/caja-persiana.jpeg}
    \caption{Caja de derivación secundaria instalada.} \label{Img:CajaDerivacion}
\end{figure}

En la imagen~\ref{Img:CajaDerivacionPrincipal} vemos la caja de derivación principal con la instalación terminada indicando algunos elementos.

\begin{figure}[h]
    \centering
    \includegraphics[width=0.9\textwidth]{img/fotos/explicación caja instalada.jpg}
    \caption{Caja de derivación principal instalada.} \label{Img:CajaDerivacionPrincipal}
\end{figure}

El primer extremo del cable, podemos ver que termina en el relé dentro de la caja de derivación pero el extremo que conectará a la Raspberry Pi no se conectará directamente sino que terminará primero en una protoboard como podemos ver en la imagen~\ref{Img:ProtoboardConecada}, donde podemos diferenciar que existen dos tipos de conectores. Los redondos ya vienen conectorizados de fábrica pero los cuadrados del tipo~\href{https://ae01.alicdn.com/kf/H4205e9c4ec4c4be4864e44b6925a22bdf/10-juegos-de-conector-de-Cable-de-2-54mm-XH2-54-conector-XH-macho-y-hembra.jpg_Q90.jpg_.webp}{JST-XH} los he fabricado según las necesidades.~\\~\\

\begin{figure}[h]
    \centering
    \includegraphics[width=0.7\textwidth]{img/fotos/protoboard-instalada.jpeg}
    \caption{Protoboard conectada.} \label{Img:ProtoboardConecada}
\end{figure}

En la imagen~\ref{Img:InstalacionDomotica} podemos ver la situación final de la tirada de cable y conectorizado.~\\~\\

\begin{figure}[h]
    \centering
    \includegraphics[width=0.9\textwidth]{img/fotos/RBP-instalada.jpeg}
    \caption{Instalación domótica y caja de derivación principal.} \label{Img:InstalacionDomotica}
\end{figure}


\begin{figure}[h]
    \centering
    \includegraphics[width=0.9\textwidth]{img/RbP_CompletaLineas.png}
    \caption{Conectorizado final de las líneas GPIO.} \label{Img:RbP_CompletaLineas}
\end{figure}

El conectorizado final de las salidas de los GPIO es el que se puede ver en la imagen~\ref{Img:RbP_CompletaLineas}

Podemos interactuar con los puertos GPIO desde Bash y desde Python~\cite{misc:Python}. Se ha decidido utilizar lenguaje Bash para interactuar con los GPIO, dejando la potencia de Python para la obtención y procesado de datos. En ocasiones se producen errores al lanzar desde Cron scripts Python, por lo que se lanzará todo desde scripts bash.


\subsection{Diagramas eléctricos}
Para mayor claridad he generado unos planos eléctricos de cómo se ha realizado la instalación.

\begin{figure}[h]
    \centering
    \includegraphics[width=0.9\textwidth]{img/Diagramas/rele-pulsador-motor.png}
    \caption{Raspberry Pi, relé y pulsador.} \label{Img:Relé+Pulsador+Rbp_Fritzing}
\end{figure}

En la imagen~\ref{Img:Relé+Pulsador+Rbp_Fritzing} podemos ver que tenemos conectada una placa con dos relés y un pulsador en paralelo, con el motor de la persiana. La protoboard realmente no existe en este punto pero la he incluido para clarificar el circuito.

En la imagen~\ref{Img:Relé+Rbp_Fritzing} podemos ver como se conecta, desde la Raspberry, el relé que conecta con una persiana identificando cada uno de los pines.

\begin{figure}[h]
    \centering
    \includegraphics[width=0.5\textwidth, angle =90]{img/Diagramas/relé-RBP.png}
    \caption{Raspberry y relé conectados.} \label{Img:Relé+Rbp_Fritzing}
\end{figure}

\begin{figure}[h]
    \centering
    \includegraphics[width=0.6\textwidth]{img/Diagramas/PulsadorInterno.png}
    \caption{Funcionamiento interno de un pulsador para persianas.} \label{img:PulsadorInterno}
\end{figure}


Para explicar el funcionamiento de un pulsador de 3 posiciones para persianas podemos ver la imagen~\ref{img:PulsadorInterno}. En ella podemos ver que en la posición central no tiene ninguna salida eléctrica y que las salidas 1 y 2 tienen la polaridad invertida. La lógica del cambio de orden en la salida de cables es porque según el sentido de la polaridad el motor girará en un sentido o en otro. Esto se respalda mediante la~\href{https://fisica.laguia2000.com/dinamica-clasica/fuerzas/ley-de-laplace-fuerza-ejercida-sobre-un-conductor}{segunda ley de Laplace} para cada espira del bobinado del motor quedando de la forma~\href{http://www.uco.es/grupos/giie/cirweb/teoria/tema_11/tema_11_01.pdf}{$M=N*(I*B*l+sen\theta)$}.

Siguiendo esta lógica, podemos ver que en la imagen~\ref{Img:Relé+Pulsador+Rbp_Fritzing} tenemos dos relés de forma que cada uno de ellos deja pasar la corriente en un sentido. En la realidad, el motor no tiene por qué tener dos entradas positivas y dos negativas pero en el dibujo queda mejor ilustrado que son entradas diferentes.

Resumiendo, la instalación mecánica y nuestra instalación mediante relés se instalarán en paralelo para poder activar una opción o la otra según la ocasión.

\section{Requisitos de usuarios}
Para poder ejecutar el código, el usuario necesitará este material, que también podemos ver en la imagen~\ref{material}:
\begin{itemize}
    \item Un terminal que soporte mensajería mediante Telegram. Cabe recordar que Telegram es multiplataforma, de modo que serviría un ordenador, un smartphone o un smartwatch entre otros.
    \item Una Raspberry Pi conectada a Internet.
    \item Una tarjeta uSD que haga de disco duro de nuestra Raspberry Pi.
    \item Alimentación de 2A para la Raspberry Pi.
    \item Otro equipo desde el que descargar y montar el Sistema Operativo de Raspberry Pi.
    \item Necesita generar cuentas en Climacell y WeatherAPI y un bot en BotFather.
\end{itemize}

\begin{figure}[!h]
\centering
\includegraphics[width=0.7\textwidth]{img/fotos/RbP+SD+pen.jpeg}
\caption{Material básico para ejecutar el código.}\label{material}
\end{figure}

\section{Obtención de tokens}
En este punto se indica como obtener un token para poder utilizar las APIs.

\subsection{Climacell}
La primera API es Climacell. Para obtener el token debemos realizar los siguientes pasos:
\begin{itemize}
    \item Accedemos a la web oficial:~\url{https://www.climacell.co/pricing/}.
    \item Seleccionamos la opción gratuita.
    \item Rellenamos los datos que nos piden.
    \item Obtenemos el token de la pantalla que nos aparece.
\end{itemize}
Este token debemos guardarle muy bien puesto que es uno de los que utilizaremos para que el sistema funcione correctamente.

\subsection{WeatherApi}
La segunda API es WeatherApi. Para obtener el token debemos realizar los siguientes pasos:
\begin{itemize}
    \item Accedemos a la web oficial:~\url{https://openweathermap.org/price}.
    \item Seleccionamos la opción gratuita.
    \item Rellenamos los datos que nos piden.
    \item Accedemos a la pestaña de API Keys.
    \item Obtenemos el token de la pantalla que nos aparece.
\end{itemize}
Este token también debemos guardarlo muy bien puesto que es uno de los que utilizaremos para que el sistema funcione correctamente.

\subsection{BotFather}
Los pasos para crear nuestro bot y obtener su token, son:
\begin{itemize}
    \item Desde una instancia logada de Telegram debemos buscar un chat llamado BotFather.
    \item Le enviamos el mensaje~\texttt{/start}
    \item Le enviamos el mensaje~\texttt{/newbot}
    \item Ahora, debemos enviarle el nombre que queremos ponerle a nuestro bot.
    \item La respuesta del BotFather es un mensaje donde nos indica el nombre de nuesto bot y el token que debemos guardar (en rojo).
\end{itemize}
Éste es el último token que debemos guardar para completar el archivo \texttt{config2.bot}.

\section{Instalación}

\subsection{Sistema Operativo}
El primer paso es descargar el Sistema Operativo de la web oficial y montarlo en la tarjeta SD con la ayuda de otro equipo informático y el adaptador USB para uSD que vemos en la imagen~\ref{material}. 

\begin{enumerate}
    \item Accedemos a: \url{https://www.raspberrypi.org/software/}
    \item Descargamos el software para montar la imagen en nuestra uSD \url{https://downloads.raspberrypi.org/imager/imager_1.5.exe}
    \item Introducimos el adaptador USB con la uSD en el equipo y corremos el software <<imagen\_1.5.exe>> para instalarlo, en este caso.
    \item En la ventana de instalación, seleccionamos <<install>> como figura en la imagen~\ref{instalacionRaspbian} para instalar el programa <<imager>> de Raspberry Pi Foundation.
\end{enumerate}

\begin{figure}[h]
\centering
\includegraphics[width=0.7\textwidth]{img/fotos/instalacionRaspbian.PNG}
\caption{Ventana instalación <<imager>>, de Raspberry Foundation.}\label{instalacionRaspbian}
\end{figure}

\begin{enumerate}
\setcounter{enumi}{4}
    \item Seleccionamos las opciones de <<Sistema Operativo>> y <<SD Card>> como figura en la imagen~\ref{flasheadoSDRbP}.
\end{enumerate}

\begin{figure}[h]
\centering
\includegraphics[width=0.8\textwidth]{img/fotos/flasheadoSDRbP.PNG}
\caption{Ventana selección imagen Raspbian.}\label{flasheadoSDRbP}
\end{figure}

\begin{enumerate}
\setcounter{enumi}{5}
    \item Una vez termine la grabación de la tarjeta, procederemos a introducirla en la Raspberry Pi por la ranura destinada a ello en la parte inferior de la placa y podremos acceder a Sistema Operativo con la ayuda de una pantalla, un teclado y un ratón.
    \item Debemos completar las siguientes ventanas de configuración:
\end{enumerate}

\begin{figure}[h]
\centering
\includegraphics[width=0.8\textwidth]{img/Instalación/1.png}
\caption{Ventana de configuración Raspbian 1.}\label{Instala1}
\end{figure}

\begin{figure}[h]
\centering
\includegraphics[width=0.8\textwidth]{img/Instalación/2.png}
\caption{Ventana de configuración Raspbian 2.}\label{Instala2}
\end{figure}

\begin{figure}[h]
\centering
\includegraphics[width=0.8\textwidth]{img/Instalación/3.png}
\caption{Ventana de configuración Raspbian 3.}\label{Instala3}
\end{figure}

\begin{figure}[h]
\centering
\includegraphics[width=0.8\textwidth]{img/Instalación/4.png}
\caption{Ventana de configuración Raspbian 4.}\label{Instala4}
\end{figure}

\begin{figure}[h]
\centering
\includegraphics[width=0.8\textwidth]{img/Instalación/5.png}
\caption{Ventana de configuración Raspbian 5.}\label{Instala5}
\end{figure}

\newpage{\ }\thispagestyle{empty}
\newpage{\ }\thispagestyle{empty}
\newpage{\ }\thispagestyle{empty}
Una vez realizada la configuración básica vamos a habilitar VNC para poder conectarnos desde otros equipos de forma sencilla accediendo al menú de la imagen~\ref{VNC1}. Este punto es opcional pero recomendable para evitar desplazarnos a la ubicación de la Raspberry Pi para hacer cualquier operación en local.

\begin{figure}[h]
\centering
\includegraphics[width=0.8\textwidth]{img/Instalación/tempsnip.png}
\caption{Configura VNC 1.}\label{VNC1}
\end{figure}

Cuando abra la ventana, debemos acceder a la pestaña <<interfaces>> y activar VNC como podemos ver en la imagen~\ref{VNC2}:

\begin{figure}[h]
\centering
\includegraphics[width=0.8\textwidth]{img/Instalación/vnc2.PNG}
\caption{Configura VNC 2.}\label{VNC2}
\end{figure}

De esta forma podemos acceder desde VNC Viewer a nuestra máquina sabiendo la dirección IP, usuario y contraseña de ésta.

\subsection{Descarga e instalación del software SDI}
Para descargarnos e instalar el software SDI debemos seguir los siguientes pasos:

\begin{enumerate}
    \item Descargar la última release que exista en el repositorio de GitHub en el formato que se prefiera:~\url{https://github.com/davidelinformatico/TFG/releases}
    \item Para extraer del archivo .tar utilizar:~\\ \texttt{tar -xvf SistemaDomoticoInteligente\_v1.0.tar}.
    \item Para extraer del archivo .tar.gz utilizar:~\\ \texttt{tar xzvf SistemaDomoticoInteligente\_v1.0.tar.gz}.
\end{enumerate}
En este punto ya tendremos todos los archivos extraídos.

El siguiente paso es rellenar el archivo \texttt{config2.bot} con los tokens obtenidos anteriormente y la información de estancias donde se sitúan los periféricos y sus pines de control.

Ahora instalaremos el software. Para ello accederemos a \textbf{\textasciitilde/source/scripts/auto/} y corremos el archivo \texttt{setup.sh} con la siguiente sentencia:
\begin{lstlisting}[language=sh,firstnumber=0]
sh setup.sh
\end{lstlisting}

En este momento se descargarán todos los elementos necesarios de internet y se instalarán.

\subsection{Automatizado y puesta a producción}
Uno de los puntos de la automatización es la generación de un demonio para nuestro bot de forma que se ejecutará con el inicio del sistema y podremos trabajar con el una sencilla sentencia:

\begin{lstlisting}[language=sh,firstnumber=0]
sudo systemctl bot stop
\end{lstlisting}
\begin{lstlisting}[language=sh]
sudo systemctl bot start 
\end{lstlisting}
Para conseguirlo, he realizado los siguientes pasos:

Se debe generar el archivo de nuestro demonio en \texttt{lib/systemd/system/} con la extensión <<.service>>:
\begin{lstlisting}[language=sh, firstnumber=0]
sudo nano /lib/systemd/system/bot.service
\end{lstlisting}
El contenido del archivo es:
\begin{lstlisting}[language=sh, caption={Modificaciones en el archivo /lib/systemd/system/bot.service.}, firstnumber=0]
[Unit]
Description=Lanza el bot de control domotico
After=network.target
StartLimitIntervalSec=0

[Service]
Type=simple
Restart=always
RestartSec=1
User=pi
WorkingDirectory=/home/pi/source/TFG/scripts/bot/
ExecStart=/usr/bin/env python3 /home/pi/source/TFG/scripts/bot/bot.py

[Install]
WantedBy=multi-user.target
\end{lstlisting}
Después debemos actualizar los demonios con: 
\begin{lstlisting}[language=sh, firstnumber=0]
systemctl daemon-reload
\end{lstlisting}
Iniciar el demonio: 
\begin{lstlisting}[language=sh, firstnumber=0]
sudo systemctl start bot
\end{lstlisting}
Parar el demonio: 
\begin{lstlisting}[language=sh, firstnumber=0]
sudo systemctl stop bot
\end{lstlisting}
Estado del demonio, con el que podremos conocer el pid~\footnote{Identificador del proceso}, que siempre es útil: 
\begin{lstlisting}[language=sh, firstnumber=0]
sudo systemctl status bot
\end{lstlisting}

Para incluirlo en el inicio de la máquina habría que moverlo a /etc/init.d y luego ejecutar: 
\begin{lstlisting}[language=sh, firstnumber=0]    
sudo update-rc.d bot defaults
sudo systemctl daemon-reload
sudo systemctl enable bot
sudo systemctl start bot
\end{lstlisting}  

Con esto tendríamos el software completamente instalado.

\section{Inclusión de usuarios en el bot}
El primer paso es buscar el bot por su nombre. En mi caso se llama \textbf{SDI\_Domo\_bot}.
Una vez localizado, accedemos a él y le enviamos un mensaje. A lo que nuestro bot nos responderá con un mensaje con un número largo. Éste número largo es el que debemos incluir en el apartado de usuarios dentro del archivo \texttt{config2.bot}.

AAG: Reestructurar


\bibliography{bibliografiaAnexos}
\bibliographystyle{plain}

\clearpage

\newenvironment{bottompar}{\par\vspace*{\fill}}{\clearpage}

\begin{bottompar}
\begin{figure}[h]
    \centering
    \includegraphics[width=0.3\textwidth]{img/Diagramas/ccbysa.png}
\end{figure}

\begin{center}
Esta obra está bajo una licencia 

Creative Commons Reconocimiento 3.0 Internacional

(\href{https://creativecommons.org/licenses/by-sa/3.0/es/}{CC-BY-SA-3.0}).
\end{center}
\end{bottompar}

\end{document}

