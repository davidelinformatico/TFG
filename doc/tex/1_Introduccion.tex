\capitulo{1}{Introducción}

El concepto de domótica se acuñó para poder denominar a aquellos sistemas que disponen de la capacidad de automatizar elementos de una vivienda aportando confort, seguridad, mejoras energéticas, etc.

El término procede de la unión de dos palabras:

\begin{itemize}
    \item Domo, procedente del griego <<domus>>, que significa casa, vivienda.
    \item Por otra parte, <<tica>> procede de automática, cuyo significado es que dispone de la capacidad para realizar tareas por sí solo.
\end{itemize}

Formando una palabra cuyo significado aúna los términos de casa y automático.

Pese a conformarse el término de domótica en el año 1984, ésta aún es una gran desconocida, aunque se van introduciendo pequeños elementos automatizables como pueden ser las famosas bombillas que podemos encender o apagar desde diferentes plataformas.

Al carecer de movilidad desde la llegada de la pandemia de la COVID19 nos vemos en la necesidad de que nuestras viviendas dispongan de algún elemento de seguridad a un precio razonable, como puede ser un sistema que simule nuestra presencia en la vivienda para intentar evitar posibles percances, aumentando la sensación de confort.

Hay quien opta por opciones tradicionales de seguridad como el blindaje del domicilio para impedir el acceso o contratar a una empresa externa para que monitorice el domicilio. Los sistemas domóticos que desarrollaremos pretenden ser elementos complementarios.

Nuestro simulador de presencia funcionará de forma autónoma interactuando con persianas y luces desde una máquina Raspberry Pi mediante relés. De esta forma la vivienda parece estar ocupada de forma que ahuyentamos a potenciales delincuentes. También dispondremos de un estudio diario de la temperatura con la que podremos programar la calefacción. Además, este sistema domótico es fácilmente escalable con sistemas de acceso a la vivienda, telefonía IP, música, calefacción, telefonillo IP, etc.

Por otro lado, el que las persianas estén automatizadas generará un evidente ahorro energético, tanto en invierno como en verano, al hacer de pantalla térmica exterior.

En resumen, el proyecto se sitúa en un campo que cubre un conjunto de necesidades generales dentro de los domicilios, a un bajo coste y con relativa sencillez a la hora de implantarlo, lo cual hace que pueda llegar a un gran número de hogares.
