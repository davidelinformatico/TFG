\capitulo{2}{Objetivos del proyecto}

Con este proyecto se pretende crear un sistema domótico automatizado que nos permita aumentar la seguridad y la sensación de confort y bienestar dentro de nuestros domicilios.

Para ello debemos alcanzar algunos objetivos funcionales mínimos.

\section{Objetivos funcionales}

\begin{itemize}
    \item El sistema debe ser capaz de extraer información de Internet ya sea vía API o web scrapping.
    \item Debe poderse conectar a distintas instalaciones para poder tomar decisiones sobre éstas de una forma parametrizada.
    \item El usuario podrá interactuar con la máquina cuando lo desee.
    \item La instalación se podrá realizar con material corriente de fácil acceso.
\end{itemize}

\section{Objetivos técnicos}
\begin{itemize}
    \item El sistema domótico funcionará de forma autónoma para que no interfiera en la vida diaria del inquilino y consiga facilitarle el día a día.
    \item Poder controlar elementos eléctricos desde la interfaz GPIO\footnote{General Purpose Input/Output, que significa <<Entrada/Salida de Propósito General>>}.
    \item Debe ser un proyecto de bajo coste y asequible para que pueda llegar al mayor número de viviendas posible aumentando el beneficio social.
    \item Debe disponer de un Sistema Operativo GNU como puede ser una distribución de Linux. En nuestro caso utilizaremos Raspbian.
    \item Posibilidad de interacción multiplataforma para aumentar la flexibilidad a la hora de interactuar con los equipos.
    \item También, debe ser fácilmente escalable para permitir la posibilidad de continuar con el desarrollo.
    \item Se pretende conseguir también un notable ahorro energético real que repercuta en el bolsillo de quien instale el sistema como ayudar a combatir el cambio climático consumiendo de una manera autónoma y responsable conforme a los parámetros del domicilio haciendo de éste un entorno más eficiente. Por ello, podremos controlar el encendido de la calefacción.
    \item Utilizar Git y Github como sistema de control de versiones, de Zenhub como herramienta de gestión de proyectos y Atom para trabajar con el control de versiones.
    \item Utilización de Frizking para realizar los diseños electrónicos de la instalación.
    \item Utilización de www.draw.io para realizar los diseños de la instalación.
    \item Tratamiento de datos utilizando lenguaje Python.
    \item Utilización de diferentes APIs para obtener datos.
    \item Correcto conectorizado electrónico de todos los elementos según RETB.
\end{itemize}


\section{Objetivos personales}
\begin{itemize}
    \item Comprender la composición de un sistema domótico y aplicarlo.
    \item Desarrollar un sistema domótico con cierta complejidad y autonomía más allá de subir o bajar persianas o encender y apagar luces de forma programada.
    \item Obtener conocimientos sobre trabajo con json desde Python.
    \item Obtener conocimientos de web scraping para extraer datos correctamente de la web.
    \item Aprender a controlar elementos eléctricos desde una Raspberry Pi.
    \item Aprender a utilizar una Raspberry Pi para fines domóticos utilizando GPIO.
    \item Poner en práctica conocimientos de cableado estructurado y REBT.
    \item Profundizar mis conocimientos sobre Linux.
    \item Poder aportar un dispositivo útil a la sociedad.
\end{itemize}
