\apendice{Especificación de Requisitos}

\section{Introducción}

\section{Objetivos generales}

\section{Funcionamiento autónomo}
Se pretende que tras pocas configuraciones el sistema sea capaz de funcionar extrayendo información de Internet y decidiendo que acción realizar en cada caso a partir de ese momento.

\section{Escalabilidad}
Como cualquier proyecto de calidad debe ser escalable, esto es, debe existir la posibilidad de que sea ampliado fácilmente.

\section{Acceso multiplataforma}
En este caso dispondremos de un acceso GUI(del inglés <<Graphical User Interface>> o <<Interfaz Gráfica de Usuario>>) o CLI\footnote{Traducción del inglés: 'Command Line Interface'} (Línea de comandos) a nuestro equipo.

\section{Ahorro energético}
El ahorro energético se produce al permitir que la temperatura exterior incida, o no, en la vivienda para conseguir las condiciones deseadas optimizando el consumo de recursos.

COMPROBAR UBICACIÓN
            \section{Objetivos técnicos}
            \begin{itemize}
                
                \item Debe disponer de un Sistema Operativo GNU como puede ser una distribución de Linux. En nuestro caso utilizaremos Raspbian.
                \item Posibilidad de interacción multiplataforma para aumentar la flexibilidad a la hora de interactuar con los equipos.
                \item También, debe ser fácilmente escalable para permitir la posibilidad de continuar con el desarrollo.
                
                \item Utilizar Git y Github como sistema de control de versiones, de Zenhub como herramienta de gestión de proyectos y Atom para trabajar con el control de versiones.
                \item Utilización de Frizking para realizar los diseños electrónicos de la instalación.
                \item Utilización de www.draw.io para realizar los diseños de la instalación.
                \item Tratamiento de datos utilizando lenguaje Python.
                \item Utilización de diferentes APIs para obtener datos.
                \item Correcto conectorizado electrónico de todos los elementos según RETB.
            \end{itemize}





\section{Catálogo de requisitos}

\section{Especificación de requisitos}


