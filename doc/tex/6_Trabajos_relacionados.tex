\capitulo{6}{Trabajos relacionados}
\begin{comment}
Este apartado sería parecido a un estado del arte de una tesis o tesina. En un trabajo final grado no parece obligada su presencia, aunque se puede dejar a juicio del tutor el incluir un pequeño resumen comentado de los trabajos y proyectos ya realizados en el campo del proyecto en curso. 
\end{comment}

Como he comentado anteriormente, el tiempo invertido en la búsqueda de información válida ha sido abundante por lo que creo interesante reflejar la parte que me ha servido de ayuda para darle forma al proyecto.

%https://github.com/ubutfgm/plantillaLatex/tree/master/tex

\subsection{Artículos}
\subsubsection{Utilidades para Raspberry Pi}
En este repositorio de Github encontramos como se conecta una pantalla LCD mediante los puertos GPIO.
\begin{itemize}
    \item Url del artículo: \url{https://github.com/tidus747/Utilidades_RaspberryPi}
\end{itemize}

\subsubsection{Tutoriales básicos de programación en Raspberry Pi}
En este repositorio de Github se publican algunos ejercicios parte de un tutorial básico con Raspberry Pi.
\begin{itemize}
    \item Url del artículo: \url{https://github.com/tidus747/Tutoriales_RaspberryPi}
\end{itemize}

\subsubsection{Control eléctrico mediante Raspberry Pi}
En el proyecto controlan la salida de fluido eléctrico mediante relés y una interfaz HTML+CSS.
\begin{itemize}
    \item Url del artículo: \url{https://chemise23.wordpress.com/2013/09/19/domotica-con-raspberry/}
\end{itemize}

\subsubsection{Control de LEDs desde una interfaz web HTML}
En el proyecto se genera una página web en Flask desde la que se controlan unos LEDs.
\begin{itemize}
    \item Url del artículo: \url{https://github.com/WilliBobadilla/Domotica}
\end{itemize}

\subsubsection{Control de LEDs desde una apk}
El proyecto genera una aplicación android desde la que controlar un led.
\begin{itemize}
    \item Url del artículo: \url{https://github.com/rpi-jefer/domotica_raspberry_pi/tree/master/domotica_app/Domotica}
\end{itemize}

\subsubsection{Controlando las persianas con Shelly 2.5 y Home Assistant}
En el artículo explica como controlar persianas cun Shelly desde Home assistant. Este artículo me ayudó a valorar otras opciones de comuniacción para subir y bajar persianas.
\begin{itemize}
    \item Url del artículo: \url{https://www.bujarra.com/controlando-las-persianas-con-shelly-2-5-y-home-assistant/}
\end{itemize}

\subsubsection{Crear Menú interactivo en Telegram}
Muestra como generar un menú mediante botones en un bot de Telegram.
\begin{itemize}
    \item Url del artículo: \url{https://domology.es/crear-menu-interactivo-en-telegram/}
\end{itemize}

\subsubsection{Bots: An introduction for developers}
Introducción a la creación de bots en Telegram. Ofrece información para comenzar a crear tu propio bot.
\begin{itemize}
    \item Url del artículo: \url{https://domology.es/crear-menu-interactivo-en-telegram/}
\end{itemize}

\subsubsection{Introducing Bot API 2.0}
Introducción a la API del BOT de Telegram v2.0. Ofrece inforamción de como generar algunas opciones en el bot, como pueden ser botones.
\begin{itemize}
    \item Url del artículo: \url{https://domology.es/crear-menu-interactivo-en-telegram/}
\end{itemize}


\subsection{Proyectos}

\subsubsection{Repositorio con plantilla Latex}
En este repositorio se incluye la plantilla que la parte docente facilita para realizar la redacción del proyecto.
\begin{itemize}
    \item Url del artículo: \url{https://github.com/ubutfgm/plantillaLatex}
\end{itemize}

\subsubsection{Diseño e implementación de un sistema domótico basado en Raspberry Pi}
Incluye el control domótico asistencial según parámetros internos de una vivienda. La idea es muy buena pero en mi caso se pretende implantar físicamente la instalación de forma que necesitaría hacer pequeños desperfectos en la vivienda para acoplar los sensores.
\begin{itemize}
    \item Url del artículo: \url{https://e-archivo.uc3m.es/bitstream/handle/10016/26313/TFG_Hector_Santos_Senra.pdf?sequence=1&isAllowed=y}
\end{itemize}

\section{Fortalezas y debilidades del proyecto}


[[FALTA]]


