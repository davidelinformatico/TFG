\begin{comment}
\capitulo{6}{Trabajos relacionados}
Este apartado sería parecido a un estado del arte de una tesis o tesina. En un trabajo final grado no parece obligada su presencia, aunque se puede dejar a juicio del tutor el incluir un pequeño resumen comentado de los trabajos y proyectos ya realizados en el campo del proyecto en curso. 
\end{comment}

\capitulo{6}{Trabajos relacionados}
Desde la aparición de elementos electrónicos accesibles y asequibles se ha intentado, con mayor o menor fortuna, generar sistemas que nos ayuden en nuestro día a día. A continuación se exponen proyectos con objetivos similares.

\section{Comparativa con otros proyectos}
En este apartado mostraré otras opciones de acometer un proyecto similar y compararé los principales puntos de éstos con el presente proyecto para poder tener una visión global entre estas opciones.

\subsection{Interacción remota para elementos de un hogar mediante una red de sensores actuadores}\label{Proy_Domo_1}
Se trata de un  se trata de un TFG/TFM desarrollado en la Universidad de Salamanca que trata sobre el control domótico basado en un conjunto de sensores instalados en el domicilio utilizando nodos intermedios para realizar el control de los dispositivos finales y la comunicación con los sensores.

Podemos ver su referencia en la bibliografía en la referencia~\cite{misc:TFG_Salamanca}.

En la columna `Domo\_1' de la tabla~\ref{tabla:comparativa-proyectos} podemos ver algunas características de este proyecto.

\subsection{Diseño e implementación de un sistema domótico basado en Raspberry Pi}\label{Proy_Domo_2}
Éste es un TFG/TFM desarrollado en la Universidad Carlos III, en Madrid que trata sobre el control domótico de un domicilio basándose únicamente en un conjunto de sensores instalados en el domicilio.

\begin{itemize}
    \item Url del trabajo: \url{https://e-archivo.uc3m.es/bitstream/handle/10016/26313/TFG_Hector_Santos_Senra.pdf?sequence=1&isAllowed=y}
\end{itemize}

En la columna `Domo\_2' de la tabla~\ref{tabla:comparativa-proyectos} podemos ver algunas características de este proyecto.

\section{Fortalezas y debilidades este proyecto}

Las principales fortalezas del proyecto que se presenta en esta memoria son:

\begin{itemize}
\item
    Éste es un proyecto que cualquiera puede implementar fácilmente en su domicilio haciendo una copia local del repositorio y haciendo unas sencillas configuraciones en el archivo de configuración.
    
\item
    En este proyecto se ha tenido en cuenta la eficiencia de recursos y de consumo de la propia máquina Raspberry Pi~\cite{misc:RbPWeb}, por lo que no se han incluido servicios extraordinarios como puede ser un servidor de bases de datos o web, como sucede en otros proyectos. Se ha optado por incluir un bot que `escucha' las órdenes y actúa conforme a un pequeño archivo de configuración y actúa contra la máquina utilizando los lenguajes nativos de la máquina. De esta manera, el gasto en cómputo es menor y también el utilizado en mantener los servicios en memoria, quedando recursos suficientes para continuar haciendo un uso normal de la máquina para cualquier otra tarea.
    Además, al no tener servidores levantados y con puertos de comunicaciones abiertos, aumentamos la seguridad de nuestra instalación.

\item
    Se ha procurado utilizar los lenguajes integrados de forma nativa en el Sistema Operativo para evitar posibles fallos de integración entre ellos y aumentando la fiabilidad del sistema. Estos son Python~\cite{misc:Python} y Bash~\cite{misc:Linux}. 
    
\item
    El sistema automatizado de nuestro proyecto se nutre de la información contrastable obtenida de API's reconocidas.
    
\begin{table}
\centering
\begin{tabular}{lccc}
\toprule
Características & SDI & Domo\_1 & Domo\_2  \\
\midrule
Proyecto libre                          & \cellcolor{green!25} {\checkmark} & \cellcolor{green!25} {\checkmark} & \cellcolor{red!25} {\xmark} \\
No precisa montar servicios             & \cellcolor{green!25} {\checkmark} & \cellcolor{red!25} {\xmark} & \cellcolor{red!25} {\xmark} \\
No requiere lenguajes no nativos en el SO  & \cellcolor{green!25} {\checkmark} & \cellcolor{red!25} {\xmark} & \cellcolor{red!25} {\xmark} \\
Obtiene información externa contrastada & \cellcolor{green!25} {\checkmark} & \cellcolor{red!25} {\xmark} & \cellcolor{red!25} {\xmark} \\
Interacción multiplataforma             & \cellcolor{green!25} {\checkmark} & \cellcolor{green!25} {\checkmark} & \cellcolor{green!25} {\checkmark} \\
No necesita nodos intermedios           & \cellcolor{green!25} {\checkmark} & \cellcolor{red!25} {\xmark} & \cellcolor{green!25} {\checkmark} \\
Cableado entre elementos                & \cellcolor{green!25} {\checkmark} & \cellcolor{red!25} {\xmark} & \cellcolor{green!25} {\checkmark} \\
WiFi entre elementos                    & \cellcolor{red!25} {\xmark} & \cellcolor{green!25} {\checkmark} & \cellcolor{red!25} {\xmark} \\
\bottomrule
\end{tabular}
\caption{Comparativa de las características de los proyectos.}
\label{tabla:comparativa-proyectos}
\end{table}

    
\item
    Se puede interactuar con el sistema domótico de varias formas. La más cómoda es mediante el bot de Telegram creado en exclusiva para controlar las funcionalidades principales de nuestro sistema domótico, aunque también se puede contectar de forma remota desde otra máquina mediante VNC~\footnote{VNC: <<Virtual Network Computing>>, programa de control remoto de equipos informáticos basado en una estructura <<cliente-servidor>>} siempre que se esté en la misma red. Se puede interactuar con el bot mediante aplicaciones Android o IOs, o desde navegadores de uso extendido como Chrome, Firefox o Edge, entre otros.
    
\item
    No se necesitan elementos intermedios de ningún tipo, únicamente tenemos nuestra Raspberry Pi~\cite{misc:RbPWeb} y los actuadores o relés~\ref{Img:Rele1}. De esta manera evitamos posibles fallos de los elementos intermedios. Podemos comprobarlo en la imagen~\ref{Img:diagramaBasico}.

    
\item
    Se ha preferido cablear la instalación para evitar que los elementos receptores puedan quedar en un estado no deseado así como evitar reconfigurar elementos ante cualquier posible cambio de alguno de los elementos como pueden ser el \textit{dongle} WiFi o el router, ver secciones~\ref{concepto:WIFI} y~\ref{4:Router} respectivamente.
    También, al dejar libres los canales WiFi, evitamos contribuir a la actual saturación de frecuencias en los espectros de 2.4GHz y 5GHz~\cite{manual:IEEE802.11}.

\end{itemize}


Podemos ver en la tabla~\ref{tabla:comparativa-proyectos} de comparativa de proyectos lo anteriormente expuesto.



Las principales debilidades del proyecto son:

\begin{itemize}
\tightlist
\item
    Hace falta cierta soltura a la hora de trabajar con el cableado UTP, ver sección~\ref{4:guiaPasacables}.
\item
    Es necesaria una línea de datos para que se actualicen los parámetros de los scripts.

\end{itemize}
