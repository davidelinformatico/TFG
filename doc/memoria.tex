\documentclass[a4paper,12pt,twoside]{memoir}

% Castellano
\usepackage[spanish,es-tabla]{babel}
\selectlanguage{spanish}
\usepackage[utf8]{inputenc}
\usepackage[T1]{fontenc}
\usepackage{lmodern} % Scalable font
\usepackage{microtype}
\usepackage{placeins}
\usepackage{csquotes}
\usepackage{lscape} 
\usepackage[table,xcdraw]{xcolor}
\usepackage{minted}
\usepackage{xcolor}
\usepackage{amssymb}
\usepackage{pifont}

\usepackage{makecell}

\renewcommand\theadalign{bc}
\renewcommand\theadfont{\bfseries}
\renewcommand\theadgape{\Gape[4pt]}
\renewcommand\cellgape{\Gape[4pt]}

\newcommand{\cmark}{\ding{51}}%
\newcommand{\xmark}{\ding{55}}%

% Mathematic font
\usepackage{amsfonts}

\RequirePackage{booktabs}
\RequirePackage[table]{xcolor}
\RequirePackage{xtab}
\RequirePackage{multirow}

% Multi-page tables using
\usepackage{longtable}
\usepackage{tabularx}

% Cell with line break (e.g. \specialcell{Foo\\bar})
\newcommand{\specialcell}[2][c]{%
  \begin{tabular}[#1]{@{}l@{}}#2\end{tabular}}

% Links
\PassOptionsToPackage{hyphens}{url}\usepackage[colorlinks]{hyperref}
\hypersetup{
	allcolors = {blue}
}

% Ecuaciones
\usepackage{amsmath}

% Rutas de fichero / paquete
\newcommand{\ruta}[1]{{\sffamily #1}}

% Párrafos
\nonzeroparskip

% Huérfanas y viudas
\widowpenalty100000
\clubpenalty100000

% Imagenes
\usepackage{graphicx}
\newcommand{\imagen}[2]{
	\begin{figure}[!h]
		\centering
		\includegraphics[width=0.9\textwidth]{#1}
		\caption{#2}\label{fig:#1}
	\end{figure}
	\FloatBarrier
}

\newcommand{\imagenflotante}[2]{
	\begin{figure}%[!h]
		\centering
		\includegraphics[width=0.9\textwidth]{#1}
		\caption{#2}\label{fig:#1}
	\end{figure}
}

\usepackage{listings}
\usepackage{xcolor}

\colorlet{punct}{red!60!black}
\definecolor{background}{HTML}{EEEEEE}
\definecolor{delim}{RGB}{20,105,176}
\colorlet{numb}{magenta!60!black}

\lstdefinelanguage{json}{
    basicstyle=\tiny\ttfamily,
    numbers=left,
    numberstyle=\scriptsize,
    stepnumber=1,
    numbersep=8pt,
    showstringspaces=false,
    breaklines=true,
    frame=lines,
    backgroundcolor=\color{background},
    literate=
     *{0}{{{\color{numb}0}}}{1}
      {1}{{{\color{numb}1}}}{1}
      {2}{{{\color{numb}2}}}{1}
      {3}{{{\color{numb}3}}}{1}
      {4}{{{\color{numb}4}}}{1}
      {5}{{{\color{numb}5}}}{1}
      {6}{{{\color{numb}6}}}{1}
      {7}{{{\color{numb}7}}}{1}
      {8}{{{\color{numb}8}}}{1}
      {9}{{{\color{numb}9}}}{1}
      {:}{{{\color{punct}{:}}}}{1}
      {,}{{{\color{punct}{,}}}}{1}
      {\{}{{{\color{delim}{\{}}}}{1}
      {\}}{{{\color{delim}{\}}}}}{1}
      {[}{{{\color{delim}{[}}}}{1}
      {]}{{{\color{delim}{]}}}}{1},
}
\definecolor{maroon}{cmyk}{0, 0.87, 0.68, 0.32}
\definecolor{halfgray}{gray}{0.55}
\definecolor{ipython_frame}{RGB}{207, 207, 207}
\definecolor{ipython_bg}{RGB}{247, 247, 247}
\definecolor{ipython_red}{RGB}{186, 33, 33}
\definecolor{ipython_green}{RGB}{0, 128, 0}
\definecolor{ipython_cyan}{RGB}{64, 128, 128}
\definecolor{ipython_purple}{RGB}{170, 34, 255}

\lstdefinelanguage{python}{
    morekeywords={access,and,break,class,continue,def,del,elif,else,except,exec,finally,for,from,global,if,import,in,is,lambda,not,or,pass,print,raise,return,try,while},
    morekeywords=[2]{abs,all,any,basestring,bin,bool,bytearray,callable,chr,classmethod,cmp,compile,complex,delattr,dict,dir,divmod,enumerate,eval,execfile,file,filter,float,format,frozenset,getattr,globals,hasattr,hash,help,hex,id,input,int,isinstance,issubclass,iter,len,list,locals,long,map,max,memoryview,min,next,object,oct,open,ord,pow,property,range,raw_input,reduce,reload,repr,reversed,round,set,setattr,slice,sorted,staticmethod,str,sum,super,tuple,type,unichr,unicode,vars,xrange,zip,apply,buffer,coerce,intern},
    sensitive=true,
    morecomment=[l]\#,
    morestring=[b]',
    morestring=[b]",
    morestring=[s]{'''}{'''},
    morestring=[s]{"""}{"""},
    morestring=[s]{r'}{'},
    morestring=[s]{r"}{"},
    morestring=[s]{r'''}{'''},
    morestring=[s]{r"""}{"""},
    morestring=[s]{u'}{'},
    morestring=[s]{u"}{"},
    morestring=[s]{u'''}{'''},
    morestring=[s]{u"""}{"""},
    % {replace}{replacement}{lenght of replace}
    % *{-}{-}{1} will not replace in comments and so on
    literate=
    {á}{{\'a}}1 {é}{{\'e}}1 {í}{{\'i}}1 {ó}{{\'o}}1 {ú}{{\'u}}1
    {Á}{{\'A}}1 {É}{{\'E}}1 {Í}{{\'I}}1 {Ó}{{\'O}}1 {Ú}{{\'U}}1
    {à}{{\`a}}1 {è}{{\`e}}1 {ì}{{\`i}}1 {ò}{{\`o}}1 {ù}{{\`u}}1
    {À}{{\`A}}1 {È}{{\'E}}1 {Ì}{{\`I}}1 {Ò}{{\`O}}1 {Ù}{{\`U}}1
    {ä}{{\"a}}1 {ë}{{\"e}}1 {ï}{{\"i}}1 {ö}{{\"o}}1 {ü}{{\"u}}1
    {Ä}{{\"A}}1 {Ë}{{\"E}}1 {Ï}{{\"I}}1 {Ö}{{\"O}}1 {Ü}{{\"U}}1
    {â}{{\^a}}1 {ê}{{\^e}}1 {î}{{\^i}}1 {ô}{{\^o}}1 {û}{{\^u}}1
    {Â}{{\^A}}1 {Ê}{{\^E}}1 {Î}{{\^I}}1 {Ô}{{\^O}}1 {Û}{{\^U}}1
    {œ}{{\oe}}1 {Œ}{{\OE}}1 {æ}{{\ae}}1 {Æ}{{\AE}}1 {ß}{{\ss}}1
    {ç}{{\c c}}1 {Ç}{{\c C}}1 {ø}{{\o}}1 {å}{{\r a}}1 {Å}{{\r A}}1
    {€}{{\EUR}}1 {£}{{\pounds}}1
    %
    {^}{{{\color{ipython_purple}\^{}}}}1
    {=}{{{\color{ipython_purple}=}}}1
    %
    {+}{{{\color{ipython_purple}+}}}1
    {*}{{{\color{ipython_purple}$^\ast$}}}1
    {/}{{{\color{ipython_purple}/}}}1
    %
    {+=}{{{+=}}}1
    {-=}{{{-=}}}1
    {*=}{{{$^\ast$=}}}1
    {/=}{{{/=}}}1,
    literate=
    *{-}{{{\color{ipython_purple}-}}}1
     {?}{{{\color{ipython_purple}?}}}1,
    %
    identifierstyle=\color{black}\ttfamily,
    commentstyle=\color{ipython_cyan}\ttfamily,
    stringstyle=\color{ipython_red}\ttfamily,
    keepspaces=true,
    showspaces=false,
    showstringspaces=false,
    rulecolor=\color{ipython_frame},
    frame=single,
    frameround={t}{t}{t}{t},
    framexleftmargin=6mm,
    numbers=left,
    numberstyle=\tiny\color{halfgray},
    backgroundcolor=\color{ipython_bg},
    % extendedchars=true,
    basicstyle=\scriptsize,
    keywordstyle=\color{ipython_green}\ttfamily,
}

\definecolor{listinggray}{gray}{0.9}
\definecolor{lbcolor}{rgb}{0.9,0.9,0.9}
\lstdefinelanguage{ccc}{
    backgroundcolor=\color{lbcolor},
    tabsize=4,    
    language=[GNU]C++,
    basicstyle=\scriptsize,
    upquote=true,
    aboveskip={1.5\baselineskip},
    columns=fixed,
    showstringspaces=false,
    extendedchars=false,
    breaklines=true,
    prebreak = \raisebox{0ex}[0ex][0ex]{\ensuremath{\hookleftarrow}},
    frame=single,
    numbers=left,
    showtabs=false,
    showspaces=false,
    showstringspaces=false,
    identifierstyle=\ttfamily,
    keywordstyle=\color[rgb]{0,0,1},
    commentstyle=\color[rgb]{0.026,0.112,0.095},
    stringstyle=\color[rgb]{0.627,0.126,0.941},
    numberstyle=\color[rgb]{0.205, 0.142, 0.73},
}



\definecolor{dkgreen}{rgb}{0,0.6,0}
\definecolor{dred}{rgb}{0.545,0,0}
\definecolor{dblue}{rgb}{0,0,0.545}
\definecolor{lgrey}{rgb}{0.9,0.9,0.9}
\definecolor{gray}{rgb}{0.4,0.4,0.4}
\definecolor{darkblue}{rgb}{0.0,0.0,0.6}
\lstdefinelanguage{cpp}{
      backgroundcolor=\color{lgrey},  
      basicstyle=\footnotesize \ttfamily \color{black} \bfseries,   
      breakatwhitespace=false,       
      breaklines=true,               
      captionpos=b,                   
      commentstyle=\color{dkgreen},   
      deletekeywords={...},          
      escapeinside={\%*}{*)},                  
      frame=single,                  
      language=C++,                
      keywordstyle=\color{purple},  
      morekeywords={BRIEFDescriptorConfig,string,TiXmlNode,DetectorDescriptorConfigContainer,istringstream,cerr,exit}, 
      identifierstyle=\color{black},
      stringstyle=\color{blue},      
      numbers=right,                 
      numbersep=5pt,                  
      numberstyle=\tiny\color{black}, 
      rulecolor=\color{black},        
      showspaces=false,               
      showstringspaces=false,        
      showtabs=false,                
      stepnumber=1,                   
      tabsize=5,                     
      title=\lstname,                 
    }

%---------------------------------------------------------
%\usepackage{xcolor}

\definecolor{azulon}{rgb}{0,95,175}
\definecolor{dgreen}{RGB}{0, 153, 0}



\lstdefinelanguage{sh}
{morekeywords={locate,ls,man,cmp,diff,mkdir,cp,more,chgrp,mv,chmod,chown,rm,file,rmdir,find,tail,grep,umask,head,wc,info,whatis,less,whereis,date,halt,shutdown,reboot,break,case,cat,cd,continue,do,done,echo,elif,else,exec,exit,export,expr,false,fi,for,function,if,in,kill,login,new,nohup,ps,read,readonly,return,then,true,type,wait,while,as,command,declare,help,history,jobs,logout,printf,pushd,popd,readarray,select,set,type,wait,sleep,path,pwd,out,systemctl,nano,Install,Service,Unit,}, 
morecomment=[l]\#,
morestring=[d]",
morekeywords=[2]{sudo,bash,sh,python3,\$,\{,\},\(,\), gpio,mode,stop,start,always,never,zero,enable,disable,default}, %Lanzamientos
morekeywords=[3]{bool,int,str,string,'\n','\t',+,-,*,/,=,\%,\$path,User,ExecStart,WorkingDirectory,WantedBy,Restart,RestartSec,StartLimitIntervalSec,Description,After,Type  }, %operaciones
%
backgroundcolor=\color{lgrey},
basicstyle=\scriptsize\ttfamily,
identifierstyle=\color{black},
keywordstyle=\color{dgreen}, %dblue
keywordstyle={[2]\color{red}},
keywordstyle={[3]\color{olive}},
frame=tlb,% the frame is open on the right side
resetmargins=false,
rulesepcolor=\color{black},
numbers=left,% % left
numberstyle=\tiny,
numbersep=5pt,
extendedchars=true, 
firstnumber=1,
stepnumber=1,
columns=fixed,% % to prevent inserting spaces
fontadjust=true,
keepspaces=true,
basewidth=0.5em,
captionpos=t,
abovecaptionskip=\smallskipamount,% same amount as default
belowcaptionskip=\smallskipamount,% in caption package
stringstyle=\color{black},
commentstyle=\color{teal}
}[keywords,comments,strings]






%---------------------------------------------------------



% El comando \figura nos permite insertar figuras comodamente, y utilizando
% siempre el mismo formato. Los parametros son:
% 1 -> Porcentaje del ancho de página que ocupará la figura (de 0 a 1)
% 2 --> Fichero de la imagen
% 3 --> Texto a pie de imagen
% 4 --> Etiqueta (label) para referencias
% 5 --> Opciones que queramos pasarle al \includegraphics
% 6 --> Opciones de posicionamiento a pasarle a \begin{figure}
\newcommand{\figuraConPosicion}[6]{%
  \setlength{\anchoFloat}{#1\textwidth}%
  \addtolength{\anchoFloat}{-4\fboxsep}%
  \setlength{\anchoFigura}{\anchoFloat}%
  \begin{figure}[#6]
    \begin{center}%
      \Ovalbox{%
        \begin{minipage}{\anchoFloat}%
          \begin{center}%
            \includegraphics[width=\anchoFigura,#5]{#2}%
            \caption{#3}%
            \label{#4}%
          \end{center}%
        \end{minipage}
      }%
    \end{center}%
  \end{figure}%
}

%
% Comando para incluir imágenes en formato apaisado (sin marco).
\newcommand{\figuraApaisadaSinMarco}[5]{%
  \begin{figure}%
    \begin{center}%
    \includegraphics[angle=90,height=#1\textheight,#5]{#2}%
    \caption{#3}%
    \label{#4}%
    \end{center}%
  \end{figure}%
}
% Para las tablas
\newcommand{\otoprule}{\midrule [\heavyrulewidth]}
%
% Nuevo comando para tablas pequeñas (menos de una página).
\newcommand{\tablaSmall}[5]{%
 \begin{table}
  \begin{center}
   \rowcolors {2}{gray!35}{}
   \begin{tabular}{#2}
    \toprule
    #4
    \otoprule
    #5
    \bottomrule
   \end{tabular}
   \caption{#1}
   \label{tabla:#3}
  \end{center}
 \end{table}
}

%
% Nuevo comando para tablas pequeñas (menos de una página).
\newcommand{\tablaSmallSinColores}[5]{%
 \begin{table}[H]
  \begin{center}
   \begin{tabular}{#2}
    \toprule
    #4
    \otoprule
    #5
    \bottomrule
   \end{tabular}
   \caption{#1}
   \label{tabla:#3}
  \end{center}
 \end{table}
}

\newcommand{\tablaApaisadaSmall}[5]{%
\begin{landscape}
  \begin{table}
   \begin{center}
    \rowcolors {2}{gray!35}{}
    \begin{tabular}{#2}
     \toprule
     #4
     \otoprule
     #5
     \bottomrule
    \end{tabular}
    \caption{#1}
    \label{tabla:#3}
   \end{center}
  \end{table}
\end{landscape}
}

%
% Nuevo comando para tablas grandes con cabecera y filas alternas coloreadas en gris.
\newcommand{\tabla}[6]{%
  \begin{center}
    \tablefirsthead{
      \toprule
      #5
      \otoprule
    }
    \tablehead{
      \multicolumn{#3}{l}{\small\sl continúa desde la página anterior}\\
      \toprule
      #5
      \otoprule
    }
    \tabletail{
      \hline
      \multicolumn{#3}{r}{\small\sl continúa en la página siguiente}\\
    }
    \tablelasttail{
      \hline
    }
    \bottomcaption{#1}
    \rowcolors {2}{gray!35}{}
    \begin{xtabular}{#2}
      #6
      \bottomrule
    \end{xtabular}
    \label{tabla:#4}
  \end{center}
}

%
% Nuevo comando para tablas grandes con cabecera.
\newcommand{\tablaSinColores}[6]{%
  \begin{center}
    \tablefirsthead{
      \toprule
      #5
      \otoprule
    }
    \tablehead{
      \multicolumn{#3}{l}{\small\sl continúa desde la página anterior}\\
      \toprule
      #5
      \otoprule
    }
    \tabletail{
      \hline
      \multicolumn{#3}{r}{\small\sl continúa en la página siguiente}\\
    }
    \tablelasttail{
      \hline
    }
    \bottomcaption{#1}
    \begin{xtabular}{#2}
      #6
      \bottomrule
    \end{xtabular}
    \label{tabla:#4}
  \end{center}
}

%
% Nuevo comando para tablas grandes sin cabecera.
\newcommand{\tablaSinCabecera}[5]{%
  \begin{center}
    \tablefirsthead{
      \toprule
    }
    \tablehead{
      \multicolumn{#3}{l}{\small\sl continúa desde la página anterior}\\
      \hline
    }
    \tabletail{
      \hline
      \multicolumn{#3}{r}{\small\sl continúa en la página siguiente}\\
    }
    \tablelasttail{
      \hline
    }
    \bottomcaption{#1}
  \begin{xtabular}{#2}
    #5
   \bottomrule
  \end{xtabular}
  \label{tabla:#4}
  \end{center}
}



\definecolor{cgoLight}{HTML}{EEEEEE}
\definecolor{cgoExtralight}{HTML}{FFFFFF}

%
% Nuevo comando para tablas grandes sin cabecera.
\newcommand{\tablaSinCabeceraConBandas}[5]{%
  \begin{center}
    \tablefirsthead{
      \toprule
    }
    \tablehead{
      \multicolumn{#3}{l}{\small\sl continúa desde la página anterior}\\
      \hline
    }
    \tabletail{
      \hline
      \multicolumn{#3}{r}{\small\sl continúa en la página siguiente}\\
    }
    \tablelasttail{
      \hline
    }
    \bottomcaption{#1}
    \rowcolors[]{1}{cgoExtralight}{cgoLight}

  \begin{xtabular}{#2}
    #5
   \bottomrule
  \end{xtabular}
  \label{tabla:#4}
  \end{center}
}


















\graphicspath{ {./img/} }

% Capítulos
\chapterstyle{bianchi}
\newcommand{\capitulo}[2]{
	\setcounter{chapter}{#1}
	\setcounter{section}{0}
	\chapter*{#2}
	\addcontentsline{toc}{chapter}{#2}
	\markboth{#2}{#2}
}

% Apéndices
\renewcommand{\appendixname}{Apéndice}
\renewcommand*\cftappendixname{\appendixname}

\newcommand{\apendice}[1]{
	%\renewcommand{\thechapter}{A}
	\chapter{#1}
}

\renewcommand*\cftappendixname{\appendixname\ }

% Formato de portada
\makeatletter
\usepackage{xcolor}
\newcommand{\tutor}[1]{\def\@tutor{#1}}
\newcommand{\tutors}[1]{\def\@tutors{#1}}
\newcommand{\course}[1]{\def\@course{#1}}
\definecolor{cpardoBox}{HTML}{E6E6FF}
\def\maketitle{
  \null
  \thispagestyle{empty}
  % Cabecera ----------------
\noindent\includegraphics[width=\textwidth]{cabecera}\vspace{1cm}%
  \vfill
  % Título proyecto y escudo informática ----------------
  \colorbox{cpardoBox}{%
    \begin{minipage}{.8\textwidth}
      \vspace{.5cm}\Large
      \begin{center}
      \textbf{TFG del Grado en Ingeniería Informática}\vspace{.6cm}\\
      \textbf{\LARGE\@title{}}
      \end{center}
      \vspace{.2cm}
    \end{minipage}

  }%
  \hfill\begin{minipage}{.20\textwidth}
    \includegraphics[width=\textwidth]{escudoInfor}
  \end{minipage}
  
\begin{center}
\includegraphics[width=0.45\textwidth]{img/logoRBP.pdf}
\end{center}

  \vfill
  % Datos de alumno, curso y tutores ------------------
  \begin{center}%
  
  {%
    \noindent\LARGE
    Presentado por \@author{}\\ 
    en Universidad de Burgos --- \@date{}\\
    Tutor: \@tutor{}\\
    Tutor: \@tutors{}\\
  }%
  
  \end{center}%
  \null
  \cleardoublepage
  }
\makeatother

\newcommand{\nombre}{David Colmenero Guerra} %%% cambio de comando

% Datos de portada
\title{Sistema Domótico Inteligente}
\author{\nombre}
\tutor{Álvar Arnaiz González}
\tutors{Alejandro Merino Gómez}
\date{\today}

\begin{document}

\maketitle


\newpage\null\thispagestyle{empty}\newpage


%%%%%%%%%%%%%%%%%%%%%%%%%%%%%%%%%%%%%%%%%%%%%%%%%%%%%%%%%%%%%%%%%%%%%%%%%%%%%%%%%%%%%%%%
\thispagestyle{empty}


\noindent\includegraphics[width=\textwidth]{cabecera}\vspace{1cm}

\noindent D.Álvar Arnaiz-González, profesor del departamento de Ingeniería Informática del área de Lenguajes y Sistemas Informáticos y D. Alejandro Merino Gómez, profesor del departamento de Ingeniería Electromecánica del área de Ingeniería de Sistemas y Automática,

\noindent Exponen:

\noindent Que el alumno D. \nombre, con DNI 02287122W, ha realizado el Trabajo final de Grado en Ingeniería Informática titulado título de TFG. 

\noindent Y que dicho trabajo ha sido realizado por el alumno bajo la dirección de quienes suscriben, en virtud de lo cual se autoriza su presentación y defensa.

\begin{center} %\large
En Burgos, {\large \today}
\end{center}

\vfill\vfill\vfill

% Author and supervisor
\begin{minipage}{0.45\textwidth}
\begin{flushleft} %\large
Vº. Bº. del Tutor:\\[2cm]
D. Arnaiz-González, Álvar
\end{flushleft}
\end{minipage}
\hfill
\begin{minipage}{0.45\textwidth}
\begin{flushleft} %\large
Vº. Bº. del co-tutor:\\[2cm]
D. Merino Gómez, Alejandro
\end{flushleft}
\end{minipage}
\hfill

\vfill

% para casos con solo un tutor comentar lo anterior
% y descomentar lo siguiente
%Vº. Bº. del Tutor:\\[2cm]
%D. nombre tutor


\newpage\null\thispagestyle{empty}\newpage




\frontmatter

% Abstract en castellano
\renewcommand*\abstractname{Resumen}
\begin{abstract}
El proyecto pretende integrar diversas tecnologías para confeccionar una solución domótica generalista y de bajo costo. Es decir, se pretende publicar un producto final que pueda ser utilizado por quien lo desee para confeccionar su propio sistema domótico básico dentro de su domicilio y, para ello, se ha desarrollado un software así como vídeos explicativos para su correcta comprensión.

El sistema utiliza una placa Raspberry Pi con un Sistema Operativo propio con licencia GNU, para obtener información de diferentes APIS y páginas web, como pueden ser la hora a la que amanece, anochece, la temperatura por horas, velocidad del viento, etc; permitiendo al sistema tomar decisiones conforme a estos datos y controlar diferentes dispositivos eléctricos mediante la activación de relés. Para añadir usabilidad y facilitar la capacidad de transmisión de información del sistema, se desarrollará un Bot en Telegram con el que poder interactuar.

En el proyecto se incluye la documentación que se debe consultar antes de realizar cualquier tipo de instalación de telecomunicaciones o eléctrica en el ámbito doméstico pese a que, finalmente, el proyecto se basa en el REBT\footnote{Reglamento Electrotécnico de Baja Tensión}. Esta documentación es la necesaria para implementar este proyecto de forma profesional por una empresa. 
\end{abstract}

\renewcommand*\abstractname{Descriptores}
\begin{abstract}
Android, Raspberry Pi, GPIO, REBT, ICT, Autónomo, Sistema Domótico, Linux, Bot, Telegram, energético, Python, relé, Bash scripting\ldots


\end{abstract}

\clearpage

% Abstract en inglés
\renewcommand*\abstractname{Abstract}
\begin{abstract}

The project aims to integrate various technologies to make a generalist and low-cost home automation solution. That is to say, it is intended to publish a final product that can be used by whoever wishes to make their own basic home automation system within their home and, for this, a fully commented code has been developed as well as explanatory videos for their correct understanding.

The system uses a Raspberry Pi board with its own operating system with a GNU license, to obtain information from different APIs and web pages, such as the time of sunrise, sunset, hourly temperature, wind speed, etc; allowing the system to make decisions based on this data and control different electrical devices by activating relays. To add usability and facilitate the information transmission capacity of the system, a Telegram Bot will be developed with which to interact.

The project includes the documentation that must be consulted before carrying out any type of telecommunications or electrical installation in the domestic sphere, despite the fact that, finally, the project is based on the REBT \footnote {Low Voltage Electrotechnical Regulation}. This documentation is necessary to implement this project professionally by a company.
\end{abstract}

\renewcommand*\abstractname{Descriptors}
\begin{abstract}
Android, Raspberry Pi, GPIO, REBT, ICT, Standalone, Domotic System, Linux, Bot, Telegram, energetic, Python, relay, Bash scripting \ldots
\end{abstract}

\clearpage

% Indices
\tableofcontents

\clearpage

\listoffigures

\clearpage

\listoftables
\clearpage

\mainmatter
\capitulo{1}{Introducción}

El concepto de domótica se acuñó para poder denominar a aquellos sistemas que disponen de la capacidad de automatizar elementos de una vivienda aportando confort, seguridad, mejoras energéticas, etc.

El término procede de la unión de dos palabras:

\begin{itemize}
    \item Domo, procedente del griego <<domus>>, que significa casa, vivienda.
    \item Por otra parte, <<tica>> procede de automática, cuyo significado es que dispone de la capacidad para realizar tareas por sí solo.
\end{itemize}

Formando una palabra cuyo significado aúna los términos de casa y automático.

Pese a conformarse el término de domótica en el año 1984, ésta aún es una gran desconocida, aunque se van introduciendo pequeños elementos automatizables como pueden ser las famosas bombillas que podemos encender o apagar desde diferentes plataformas.

Al carecer de movilidad desde la llegada de la pandemia de la COVID19 nos vemos en la necesidad de que nuestras viviendas dispongan de algún elemento de seguridad a un precio razonable, como puede ser un sistema que simule nuestra presencia en la vivienda para intentar evitar posibles percances, aumentando la sensación de confort.

Hay quien opta por opciones tradicionales de seguridad como el blindaje del domicilio para impedir el acceso o contratar a una empresa externa para que monitorice el domicilio. Los sistemas domóticos que desarrollaremos pretenden ser elementos complementarios.

Nuestro simulador de presencia funcionará de forma autónoma interactuando con persianas y luces desde una máquina Raspberry Pi mediante relés. De esta forma la vivienda parece estar ocupada de forma que ahuyentamos a potenciales delincuentes. También dispondremos de un estudio diario de la temperatura con la que podremos programar la calefacción. Además, este sistema domótico es fácilmente escalable con sistemas de acceso a la vivienda, telefonía IP, música, calefacción, telefonillo IP, etc.

Por otro lado, el que las persianas estén automatizadas generará un evidente ahorro energético, tanto en invierno como en verano, al hacer de pantalla térmica exterior.

En resumen, el proyecto se sitúa en un campo que cubre un conjunto de necesidades generales dentro de los domicilios, a un bajo coste y con relativa sencillez a la hora de implantarlo, lo cual hace que pueda llegar a un gran número de hogares.

\capitulo{2}{Objetivos del proyecto}

Con este proyecto se pretende crear un sistema domótico automatizado que nos permita aumentar la seguridad y la sensación de confort y bienestar dentro de nuestros domicilios.

Para ello debemos alcanzar algunos objetivos funcionales mínimos.

\section{Objetivos generales}

\begin{itemize}
    \item El sistema domótico funcionará de forma autónoma para que no interfiera en la vida diaria del inquilino y consiga facilitarle el día a día.
    \item El sistema domótico debe ser capaz de extraer información de Internet ya sea vía API o web scrapping.
    \item El sistema domótico debe poder conectarse a distintas instalaciones para actuar sobre éstas de una forma parametrizada.
    \item El usuario podrá interactuar con la máquina cuando lo desee.
    \item La instalación se podrá realizar con materiales de fácil acceso.
    \item Poder controlar elementos eléctricos desde la interfaz GPIO (General Purpose Input/Output, que significa Entrada/Salida de Propósito General).
    \item Debe ser un proyecto de bajo coste y asequible para que pueda llegar al mayor número de viviendas posible aumentando el beneficio social.
    \item Se pretende conseguir también un notable ahorro energético real que repercuta en el bolsillo de quien instale el sistema, además de ayudar a combatir el cambio climático consumiendo de una manera autónoma y responsable conforme a los parámetros del domicilio haciendo de éste un entorno más eficiente. Por ello, podremos controlar el encendido de la calefacción.
\end{itemize}

\section{Objetivos personales}
\begin{itemize}
    \item Poder aportar un dispositivo útil a la sociedad.
    \item Comprender la composición de un sistema domótico y aplicarlo.
    \item Desarrollar un sistema domótico con cierta complejidad y autonomía más allá de subir o bajar persianas o encender y apagar luces de forma programada o manual.
    \item Obtener conocimientos sobre trabajo con json desde Python.
    \item Aprender a controlar elementos eléctricos desde una Raspberry Pi.
    \item Aprender a utilizar una Raspberry Pi para fines domóticos utilizando GPIO.
    \item Poner en práctica conocimientos de cableado estructurado y REBT.
    \item Profundizar mis conocimientos sobre Linux.
    \item Aprender a utilizar el procesador de textos \LaTeX.
\end{itemize}

\capitulo{3}{Conceptos teóricos}

Este punto nace ante la necesidad de enmarcar el proyecto dentro de las tecnologías y elementos que utilizaremos durante todo el proyecto y que no tienen por qué conocerse.

El término ‘domótica’ es el pilar principal del proyecto y, por ello, comenzaré explicando lo que es y como lo enfocaremos:

\section{Domótica}
La domótica podemos definirla como aquel conjunto de elementos capaces de automatizar una vivienda aportando un beneficio.
En nuestro caso, nuestro sistema domótico deberá controlar luces, persianas y calefacción permitiendo un aumento del confort y la seguridad, además de permitir un consumo eficiente de recursos a la hora de climatizar la vivienda.

\section{RaspberryPi}
En nuestro proyecto tendremos el control de la instalación domótica desde una Raspberry Pi. 
Para dar un enfoque muy general, podemos decir que las placas RaspberryPi son microordenadores que disponen de poca potencia si las comparamos con equipos usuales pero, disponen de suficiente potencia para llevar a cabo este tipo de proyectos.

Se diseñaron en su origen por la RaspBerry Pi Foundation en el Reino Unido para dotar de equipos informáticos a los centros de estudios a un bajo coste, pero el proyecto ha evolucionado para poder desarrollar, además, otras muchas tareas como puede ser nuestro caso, que la utilizaremos como ‘núcleo’ de toda nuestra instalación domótica y, será donde configuremos todo el entorno domótico de la vivienda.
Estas placas pueden ejecutar con agilidad distribuciones Linux y, desde sus distribuciones podemos interactuar con sus famosos “GPIO”.
  \begin{center}
  \includegraphics[width=400]{RBP2B}
  \end{center}

\section{GPIO}
Éstos son unos puertos de entrada y salida conformados en forma de pines que están albergados en la placa, con los que enviaremos órdenes a otros dispositivos externos para que realicen las tareas que les designemos. En nuestro caso, servirán para controlar unos relés para conseguir una acción final.


\section{Relé}
Es un dispositivo electromagnético que desempeña la misma función de un interruptor, es decir, con nuestros relés, dejaremos pasar la energía eléctrica, o no, a nuestros dispositivos.

\section{Distribución Linux (Raspbian)}
Como he comentado anteriormente, pretendo correr una distribución Linux en nuestro microPc. Las placas Raspberry Pi disponen de unas distribuciones de Linux desarrolladas expresamente para su hardware desde la Raspberry Pi Foundation. De esta manera conseguimos que el entorno esté diseñado para el hardware donde será ejecutado incluyendo, además, utilidades preinstaladas para explotarlas más fácil y eficientemente.

\section{Funcionamiento autónomo}
Se pretende que tras pocas configuraciones el sistema sea capaz de funcionar extrayendo información de Internet y decidiendo que acción realizar en cada caso a partir de ese momento.

\section{Escalabilidad}
Como cualquier proyecto de calidad debe ser escalable, esto es, debe existir la posibilidad de que sea ampliado fácilmente.

\section{Acceso multiplataforma}
En este caso dispondremos de un acceso GUI\footnote{Traducción del inglés: 'Graphical User Interface'} (Interfaz gráfica) o CLI\footnote{Traducción del inglés: 'Command Line Interface'} (Línea de comandos) a nuestro equipo.

\section{Ahorro energético}
El ahorro energético se produce al permitir que la temperatura exterior incida, o no, en la vivienda para conseguir las condiciones deseadas optimizando el consumo de recursos.

\section{Web Scraping}
Es una técnica utilizada para extraer información de una página web utilizando las etiquetas de que dispone para movernos por los elementos de ésta.
En nuestro caso, podremos utilizarlo desde Python con beautifulsoup.

\section{API}
Es el acrónimo de ‘Application Programming Interfaces’ que, traducido al castellano significa ‘Interfaz de Programación de Aplicaciones’. Estas interfaces nos sirven información que podremos utilizar en un desarrollo.
Por ejemplo, para obtener la ubicación de la máquina Raspberry Pi accedemos a una API pública que nos devolverá información que podremos procesar a nuestro gusto.
Podemos acceder a esta \href{http://ip-api.com/json/?fields=country,regionName,city,lat,lon,isp,query}{http://ip-api.com/json/?fields=country,regionName,city,lat,lon,isp,query}.
Y obtendremos los valores de país, región, ciudad, latitud, longitud, ISP y dirección IP.
Estos valores, podremos procesarlos desde Python como json.

\section{json}
Es el acrónimo en inglés de ‘JavaScript Object Notation’ y sirve para almacenar información de forma estructurada mediante etiquetas.

\section{RETB}
Es el acrónimo de Reglamento electrotécnico para baja tensión y en él se recoge la normativa eléctrica aplicable en domicilios.
Esta norma acaba de ser actualizada y podemos disponer de la información en páginas oficiales como puede ser el  \href{https://www.boe.es/biblioteca_juridica/codigos/abrir_pdf.php?fich=326_Reglamento_electrotecnico_para_baja_tension_e_ITC.pdf}{BOE}.
Del documento \href{http://eschoform.educarex.es/useruploads/r/c/886/scorm_imported/88234455166233572664/media/guia_bt_21.pdf}{ICT-BT-21} podemos extraer información para realizar las instalaciones eléctricas de nuestro sistema domótico como el número máximo de cables a introducir por un tubo eléctrico.


\section{Normativa de ICT}
Como figura en el BOE 143, de 16 de junio de 2011: “El Reglamento regulador de las infraestructuras comunes de telecomunicaciones para el acceso a los servicios de telecomunicación en el interior de las edificaciones, aprobado por el Real Decreto 346/2011, de 11 de marzo.
Debemos regirnos por esta normativa a la hora de hacer cualquier instalación de comunicaciones nueva dentro de domicilios.
Podemos informarnos y ampliar información en la publicación del \href{https://www.boe.es/buscar/pdf/2011/BOE-A-2011-10457-consolidado.pdf}{BOE}.

Por otro lado, disponemos de guías para instaladores con dibujos y tablas que facilitan la comprensión, como puede ser la documentación que publica Televés:
\href{https://docs.televes.com/web/Legislacion/m_ict2_3ed_reglamento_0.pdf}{https://docs.televes.com/web/Legislacion/m_ict2_3ed_reglamento_0.pdf}.

De este punto, obtendremos la norma para introducir cableado ICT conforme a norma.

Tras hacer un estudio en mi domicilio, no necesitaré utilizar la normativa de ICT porque toda la instalación se realizará mediante canales eléctricos, pero está bien conocer la norma para, en caso de necesitarla poder hacer uso de ella correctamente.

\section{Cableado estructurado}
El establecimiento de un sistema de cableado estructurado consiste en la organización de los cables en un recinto conforme a una norma y constituye el nivel básico de cualquier red de comunicaciones.
Al contar y cumplir con este estándar nos damos cuenta de que tendremos instalaciones limpias, uniformes, seguras y escalables, facilitando la supervisión, el mantenimiento y posibles migraciones de tecnologías.
Un sistema de cableado genérico dispone de tres subsistemas, Troncal, de Edificio y Horizontal. En nuestro proyecto únicamente trataremos con el subsistema horizontal.

\emph{En este proyecto no contaremos con un gran número de cables, pero no está de más realizar una instalación lo más correctamente posible con unas normas de referencia}

\section{WiFi}
Es una tecnología de comunicaciones de forma inalámbrica o “Wireless”. WiFi\footnote{Traducción del inglés: 'Wireless Fidelity'} es el acrónimo traducido de “Fidelidad Inalámbrica”.
Estas tecnologías inalámbricas se rigen por la norma \underline{IEEE 802.11}.

Podemos revisar la normativa vigente al respecto en el propio IEEE:
\url{https://standards.ieee.org/standard/802_11-2016.html}{https://standards.ieee.org/standard/802_11-2016.html}
En este enlace podremos comprobar qué estándares dentro del 802.11 están vigentes y cuáles no.

\section{UTP}
Es un tipo de cableado de datos que se compone de 4 pares de cables sin apantallar que están albergados dentro de una camisa de PVC.

  \begin{center}
  \includegraphics[width=300]{img/bobina_UTP.pdf}\footnote{Licencia Creative Commons by https://solarmat.es}
  \end{center}

Existen diferentes tipos de cables de datos: UTP, STP, FTP:
\begin{itemize}
\item Los cables \textbf{UTP} no disponen de protección ante interferencias electromagnéticas.
  \begin{center}
  \includegraphics[width=200]{img/UTP.pdf}
  \end{center}

\item Los cables \textbf{FTP} disponen de una pantalla global contra interferencias electromagnéticas dentro de la camisa de PVC que recoge los 4 pares destinados a transmisión de datos.
  \begin{center}
  \includegraphics[width=200]{img/FTP.pdf}
  \end{center}

\item Los cables \textbf{STP} disponen de una pantalla contra interferencias electromagnéticas por cada par de cables pero, además, también cuentan con una malla metálica exterior.
  \begin{center}
  \includegraphics[width=200]{img/STP.pdf}
  \end{center}

\emphEn nuestro caso utilizaremos UTP puesto que no necesitamos un apantallamiento ya que no transmitiremos datos y tampoco tendremos un alto grado de interferencias.

\section{Placa de Pruebas o ProtoBoard}
Es un tablero electrónico para realizar pruebas. Protoboard es la agrupación de los términos ingleses “prototype board”.
Esta protoboard la he instalado para poder hacer fácilmente el interconexionado entre los cables que llegan de los relés y los que van a la Raspberry Pi, evitando posibles tirones y movimiento de cables a la hora de hacer alguna manipulación.
Éstas, disponen de tres zonas diferenciadas:

\begin{itemize}
    \item \textbf{Canal Central}: Está situada en el medio de la placa y es donde se colocan los circuitos.
    \item \textbf{Buses}: Se sitúan en los extremos de la placa y disponen de dos líneas:
    \item \textbf{Línea roja}: Bus positivo o de voltaje.
    \item \textbf{Línea azul}: Bus negativo o de tierra.
    \item \textbf{Pistas}: Se sitúan en la zona central de la placa y, conducen en sentido contrario de las líneas rojas y azul.
\end{itemize}

\begin{center}
\includegraphics[width=400]{img/protoboard.pdf}
\end{center}

\section{Router}
Es un dispositivo que nos permite interconectar diferentes redes de datos. En mi caso dispongo de un router con WiFi integrado para poder dotar a la Raspberry Pi de salida a Internet.
Al principio, estuve valorando la opción de ‘tirar’ un cable de datos desde la primera caja de derivación después del Cuadro General de Mando y Protección o Cuadro eléctrico, hasta el RTR (Registro Terminación de Red) que es donde tengo la línea de datos más cercana, pero en único tubo que comunica es de naturaleza eléctrica y no está bien mezclar tipos de instalaciones.

Adjunto un pequeño dibujo informativo para que se pueda entender la instalación:
\begin{center}
\includegraphics[width=400]{img/Diagrama Básico.pdf}
\end{center}


\end{itemize}


\capitulo{4}{Técnicas, herramientas y componentes}

Esta parte de la memoria tiene como objetivo presentar las técnicas metodológicas y las herramientas de desarrollo que se han utilizado para llevar a cabo el proyecto. Si se han estudiado diferentes alternativas de metodologías, herramientas, bibliotecas se puede hacer un resumen de los aspectos más destacados de cada alternativa, incluyendo comparativas entre las distintas opciones y una justificación de las elecciones realizadas. 
No se pretende que este apartado se convierta en un capítulo de un libro dedicado a cada una de las alternativas, sino comentar los aspectos más destacados de cada opción, con un repaso somero a los fundamentos esenciales y referencias bibliográficas para que el lector pueda ampliar su conocimiento sobre el tema.

\section{Web Scraping}
Es una técnica utilizada para extraer información de una página web utilizando las etiquetas de que dispone el propio lenguaje interpretado de HTML (del inglés <<HyperText Markup Language>> o lenguaje de marcas de hipertexto) para organizar elementos dentro de una página web, de forma que se introduce dentro de una etiqueta y subetiquetas hasta llegar al contenido del elemento requerido. Podemos entenderlo como si fueran contenedores lógicos configurables.
En nuestro caso, podremos utilizarlo desde Python sirviéndonos de la librería <<beautifulsoup>> siempre que necesitemos obtener información de una página web.

\section{Tirada de cable con guía pasacables}
El procedimiento a seguir es el siguiente:
\begin{enumerate}
        \item Se abren las tapas de dos cajas de derivación próximas.
        \item Se introduce una guía pasacables (herramienta plástica con la forma de cuerda para introducir cables por canalizaciones) por el extremo de uno de los tubos dentro de la caja hasta llegar al otro extremo.
        \item Se asegura el cable a uno de los extremos de la guía pasacables.
        \item Se tira del otro extremo de la guía pasacables hasta conseguir sacar el cable por éste.
\end{enumerate}

\section{RaspberryPi}
En nuestro proyecto tendremos el control de la instalación domótica desde una Raspberry Pi. 
Para dar un enfoque muy general, podemos decir que las placas RaspberryPi son microordenadores que disponen de poca potencia si las comparamos con equipos usuales, pero disponen de suficiente potencia para llevar a cabo este tipo de proyectos.

Se diseñaron en su origen por la RaspBerry Pi Foundation en el Reino Unido para dotar de equipos informáticos a los centros de estudios a un bajo coste, pero el proyecto ha evolucionado para poder desarrollar, además, otras muchas tareas como puede ser nuestro caso, que la utilizaremos como ‘núcleo’ de toda nuestra instalación domótica y, será donde configuremos todo el entorno domótico de la vivienda.
Estas placas pueden ejecutar con agilidad distribuciones Linux y, desde sus distribuciones podemos interactuar con sus famosos “GPIO”, ver imagen \ref{Img:Especificaciones RBP2B}

\begin{figure}
    \centering
    \includegraphics[width=\textwidth]{img/RBP2B.pdf}
    \caption{Especificaciones de Raspberry Pi 2B. Imagen de \url{https://raspberryparatorpes.net} modificada por mí\cite{wiki:Creative}. }\label{Img:Especificaciones RBP2B}
\end{figure}

\section{Relé}
Es un dispositivo electromagnético que desempeña la misma función de un interruptor, es decir, con nuestros relés, dejaremos pasar la energía eléctrica, o no, a nuestros dispositivos. Los relés se activan mediante impulsos eléctricos que abren o cierran el circuito según se predisponga. Podemos verlo en la imagen \ref{Img:Rele1}.
\begin{figure}
    \centering
    \includegraphics[width=\textwidth]{img/Rele_1.jpg}
    \caption{Estructura interna de un relé. Imagen de \url{https://https://commons.wikimedia.org/}\cite{manual:GNU}}. \label{Img:Rele1}
\end{figure}

\section{Distribución Linux (Raspbian)}
Como he comentado anteriormente, pretendo correr una distribución Linux en nuestro microPc. Las placas Raspberry Pi disponen de unas distribuciones de Linux desarrolladas expresamente para su hardware desde la Raspberry Pi Foundation. De esta manera conseguimos que el entorno esté diseñado para el hardware donde será ejecutado incluyendo, además, utilidades preinstaladas para explotarlas más fácil y eficientemente. Una de estas distribuciones optimizadas y orientadas a estas placas es Raspbian\cite{misc:RbPWeb} o Raspberry Pi OS, que incluye software orientado a la educación, programación y otras de uso general. Algunas de estas aplicaciones son Python\cite{misc:Python} (Lenguaje de programación que pretende que se desarrolle cógigo de una forma sencilla, rápida, poco costosa y legible), Scratch\cite{misc:Scratch}(Simulador amigable para aprender programación.) o Java\cite{misc:Java}(Lenguaje de programación multiplataforma que utiliza una máquina virtual transparente para el usuario para ejecutarse), entre otros.

\section{Placa de Pruebas o ProtoBoard}
Es un tablero electrónico para realizar pruebas. Protoboard es la agrupación de los términos ingleses “prototype board”.
Esta protoboard la he instalado para poder hacer fácilmente el interconexionado entre los cables que llegan de los relés y los que van a la Raspberry Pi, evitando posibles tirones y movimiento de cables a la hora de hacer alguna manipulación.

Éstas, disponen de tres zonas diferenciadas(Ver imagen \ref{Img:Protoboard}):

\begin{itemize}
    \item \textbf{Canal Central}: Está situada en el medio de la placa y es donde se colocan los circuitos.
    \item \textbf{Buses}: Se sitúan en los extremos de la placa y disponen de dos líneas:
    \item \textbf{Línea roja}: Bus positivo o de voltaje.
    \item \textbf{Línea azul}: Bus negativo o de tierra.
    \item \textbf{Pistas}: Se sitúan en la zona central de la placa y, conducen en sentido contrario de las líneas rojas y azul.
\end{itemize}

\begin{figure}
    \centering
    \includegraphics[width=0.9\textwidth]{img/protoboard.pdf}
    \caption{Imagen de una placa <<protoboard>>. } \label{Img:Protoboard}
\end{figure}

\section{Router}
Es un dispositivo que nos permite interconectar diferentes redes de datos. En mi caso dispongo de un router con WiFi integrado para poder dotar a la Raspberry Pi de salida a Internet.



\section{Imágenes}

Se pueden incluir imágenes con los comandos standard de \LaTeX, pero esta plantilla dispone de comandos propios como por ejemplo el siguiente:

\imagen{escudoInfor}{Autómata para una expresión vacía}



\section{Listas de items}

Existen tres posibilidades:

\begin{itemize}
	\item primer item.
	\item segundo item.
\end{itemize}

\begin{enumerate}
	\item primer item.
	\item segundo item.
\end{enumerate}

\begin{description}
	\item[Primer item] más información sobre el primer item.
	\item[Segundo item] más información sobre el segundo item.
\end{description}
	
\begin{itemize}
\item 
\end{itemize}

\section{Tablas}

Igualmente se pueden usar los comandos específicos de \LaTeX o bien usar alguno de los comandos de la plantilla.

\tablaSmall{Herramientas y tecnologías utilizadas en cada parte del proyecto}{l c c c c}{herramientasportipodeuso}
{ \multicolumn{1}{l}{Herramientas} & App AngularJS & API REST & BD & Memoria \\}{ 
HTML5 & X & & &\\
CSS3 & X & & &\\
BOOTSTRAP & X & & &\\
JavaScript & X & & &\\
AngularJS & X & & &\\
Bower & X & & &\\
PHP & & X & &\\
Karma + Jasmine & X & & &\\
Slim framework & & X & &\\
Idiorm & & X & &\\
Composer & & X & &\\
JSON & X & X & &\\
PhpStorm & X & X & &\\
MySQL & & & X &\\
PhpMyAdmin & & & X &\\
Git + BitBucket & X & X & X & X\\
Mik\TeX{} & & & & X\\
\TeX{}Maker & & & & X\\
Astah & & & & X\\
Balsamiq Mockups & X & & &\\
VersionOne & X & X & X & X\\
} 
\capitulo{5}{Aspectos relevantes del desarrollo del proyecto}
%%%%%%%%%%%%%%%%%%%%%%%%%%%%%%%%%%%%%%%%%% Borrar
\begin{comment}
Este apartado pretende recoger los aspectos más interesantes del desarrollo del proyecto, comentados por los autores del mismo.
Debe incluir desde la exposición del ciclo de vida utilizado, hasta los detalles de mayor relevancia de las fases de análisis, diseño e implementación.
Se busca que no sea una mera operación de copiar y pegar diagramas y extractos del código fuente, sino que realmente se justifiquen los caminos de solución que se han tomado, especialmente aquellos que no sean triviales.
Puede ser el lugar más adecuado para documentar los aspectos más interesantes del diseño y de la implementación, con un mayor hincapié en aspectos tales como el tipo de arquitectura elegido, los índices de las tablas de la base de datos, normalización y desnormalización, distribución en ficheros3, reglas de negocio dentro de las bases de datos (EDVHV GH GDWRV DFWLYDV), aspectos de desarrollo relacionados con el WWW...
Este apartado, debe convertirse en el resumen de la experiencia práctica del proyecto, y por sí mismo justifica que la memoria se convierta en un documento útil, fuente de referencia para los autores, los tutores y futuros alumnos.

\end{comment}
%%%%%%%%%%%%%%%%%%%%%%%%%%%%%%%%%%%%%%%%%% COMIENZO

En este punto se recogerán los aspectos más relevantes del desarrollo comentando en cada caso las decisiones tomadas para llegar a nuestros objetivos haciendo un resumen de la experiencia práctica del proyecto, de cómo se solucionaron los problemas encontrados en cada caso y la relevancia que tuvieron en el alcance total del proyecto.

\section{Motivación del proyecto}
El proyecto se me ocurrió hace años cuando me emancipé a una vivienda con las persianas motorizadas que me parecieron que podrían hacer una función mayor si dispusieran de cierto equipamiento, y cogió fuerza con la llegada de la pandemia de la COVID19 cuando el tema surgió en distintas conversaciones en las que que había quien no podía acercarse a sus segundas viviendas y estaban preocupados por una posible ocupación. Tras darle vueltas a esta situación surgió la idea de crear un sistema domótico para poder ayudar a quien lo precise con este proyecto.

\section{Formación necesaria}
El proyecto requirió muchas horas de búsqueda de ideas por la web, ya que no existe información fiable de cómo realizar una instalación de estas características sino que existen muchos pequeños proyectos amateur que se centran en cubrir una pequeña necesidad; siendo proyectos que no disponen de un respaldo documental detrás. Abundan las soluciones de profesionales de otros campos que quieren probar a hacer sus propias soluciones de carácter amateur.

Por ello, para poder desarrollar el proyecto me vi en la necesidad de visitar numerosas páginas web de distinta índole para hacerme a la idea de cómo podría enfocar el proyecto. Aunque, el mayor hándicap a la hora de realizar este proyecto es la desinformación, por lo que me apoyé sobre todo en el REBT~\ref{concepto:RETB}~\cite{manual:REBT} y en mis conocimientos básicos de motores fruto de formación pasada.

La información de cómo funcionan los GPIO la obtuve tras realizar el curso de: “Control de GPIO con Python en Raspberry Pi” de Programo Ergo Sum~\cite{misc:programoergosum}.
Aunque, en parte la información de la web de \url{bujarra.com}~\cite{misc:BujarraGPIO} está desactualizada, esta publicación me ayudó a comprender cuál era el funcionamiento real de un relé y como conectarlo a los GPIO.

Otro punto importante a la hora de enfocar correctamente el proyecto fue el estudio del REBT~\cite{manual:REBT}, de su apartado BT-21~\cite{manual:ICT-BT-21}, del reglamento de ICT~\cite{manual:ICT} y los estándares de comunicaciones como son el IEEE802.11~\cite{manual:IEEE802.11} y el TIA568~\cite{manual:568.1}~\cite{manual:568.2}. Creo necesario confrontar estas normativas para poder realizar un proyecto a nivel profesional y documentalmente respaldado.

\section{Metodología}
Se eligió Scrum como la metodología global sobre la que realizar el proyecto de forma iterativa y ágil. Buscamos generar un proyecto lo más cercano a cómo sería un proyecto de este tipo en el ámbito laboral, pero salvando las distancias con un grupo de trabajo real con el que poder interactuar. Aunque he tenido reuniones semanales con los tutores no se ha realizado un seguimiento diario pero sí se cumplieron, en cualquier caso, unas reglas generales mínimas:

\begin{itemize}
    \item Los trabajos se desarrollaron en forma de <<sprints>> semanales.
    \item Tras cada <<sprint>> semanal se realiza una entrega de trabajo de forma incremental.
    \item En cada revisión del <<sprint>> se disponen los trabajos a realizar durante la semana.
    \item Realizando una revisión semanal, el proyecto goza de flexibilidad al integrar cambios sobre trabajos pequeños de forma que el proyecto está en constante mejora.
    \item Sobre el Kanban se realiza la estimación de tiempos por tareas según dificultad.
    \item Se priorizan las tareas del proyecto según mayor valor de negocio.
    \item Se cambia el estado de los <<issues>>, mediante un Kanban, según evoluciona el trabajo.
    \item Comprobamos como avanza el proyecto con los gráficos BurnDown de que dispone Zenhub.
\end{itemize}

Reseñar que en las fases de trabajos físicos no se ha aplicado Scrum pero se ha realizado con un criterio lo más cercano posible. Por ello, se predispuso realizar la tirada de cable y el conexionado la misma semana para que pudiera integrarse lo mejor posible en la metodología.

\section{Desarrollo del proyecto}
A medida que el proyecto ha ido tomando forma también se han ido implementando cambios: La idea original del proyecto era parametrizar los elementos según horas pero posteriormente se valoró e implantó la idea de tomar datos externos para poder controlar los elementos del sistema domótico. El proyecto se ha creado de forma cronológica siguiendo los pasos que se desarrollan a continuación.

El proyecto se comenzó por los scripts de extracción de datos para comprobar que se podía realizar el software que se había propuesto antes de continuar ya que es más fácil presentar cambios al principio que cuando el proyecto está avanzado. Aunque cabe destacar que se ha invertido una gran cantidad de tiempo en la búsqueda de información y normativas que desconocía, así como en la extracción y el tratamiento de datos.

Por tanto, se comenzó extrayendo datos desde Python~\cite{misc:Python} mediante web scraping~\ref{4:WebScraping} pero nos dimos cuenta de que una web es más susceptible de sufrir cambios. Por ello, recurrimos a APIS de terceros sin coste para poder realizar dicha extracción de datos, y tratándolos como json~\cite{misc:Json} extraer los datos que nos interesen.

\subsection{Extracción de datos}
Los algoritmos de extracción y tratamiento de datos también han ido cambiando ya que se han ido modificando las APIS para obtener más datos y de carácter más fiable.

Al comenzar a extraer datos de la API de geolocalización de~\url{www.ifconfig.me/ip} me obtenía la ubicación de la provincia, lo cual no es nada exacto ya que tendremos temperaturas diferentes según ubicación. Por ello, me vi obligado a hacer una búsqueda más a fondo para encontrar una API que se ajustase a las necesidades del proyecto. Tardé algunos días hasta que encontré \url{http://ip-api.com} que me entregaba más información y más fiable. Además, en el mismo intervalo, pasamos de utilizar web scraping a tratar los datos directamente como json~\cite{misc:Json}, que también es un cambio importante.

\begin{figure}
    \centering
    \includegraphics[width=\textwidth]{img/Cron1.PNG}
    \caption{Proceso de obtención de datos, programación de tareas y ejecución.} \label{Img:Cron1}
\end{figure}

En la segunda API, en el caso de~\url{www.weatherapi.com} podía obtener únicamente los valores de salida y puesta de sol para realizar el control de las persianas pero también era insuficiente, por lo que tuve que volver a hacer una búsqueda más amplia hasta que encontré~\url{www.climacell.co}, que me entrega las temperaturas de las próximas horas además de otros muchos parámetros que pueden llegar a servirnos como, por ejemplo, la velocidad del viento o si será un día nublado.

Tras obtener la ubicación de la instalación y extraer los parámetros necesarios de las APIS pasamos a conformar el archivo de datos para generar la configuración de Cron (Cron, es el programador de tareas de Linux~\cite{misc:Linux}). Esta fase dispone de varios pasos:

\begin{enumerate}
    \item Desde <<Cron>> se llama al script de toma de datos.
    \item Conformamos el nuevo archivo de <<Cron>>.
    \item El nuevo <<Cron>> lanza los scripts según las horas predispuestas.
\end{enumerate}

De esta manera el control de los dispositivos se hará automáticamente desde <<Cron>> con los datos actualizados diariamente. Podemos ver un diagrama explicativo de este proceso en la imagen~\ref{Img:Cron1}. Además, podemos ver a continuación un ejemplo de archivo de recopilación de datos:

\begin{minipage}{\linewidth}
\begin{lstlisting}[language=json, basicstyle=\small, caption={Ejemplo archivo de recopilado de datos.}]
{
"Planeta":
	{
	"Amanecer": "08:37:00",
	"Anochecer": "17:57:00"
	},

"temperaturas":
	{
	"0": 5.26,
	"1": 5.78,
	"2": 5.93,
	"3": 5.95,
	"4": 5.97,
	"5": 5.76,
	"6": 4.72,
	"7": 3.79,
	"8": 3.38,
	"9": 3.98,
	"10": 5.3,
	"11": 6.58,
	"12": 7.46,
	"13": 8.47,
	"14": 8.78,
	"15": 8.89,
	"16": 7.91,
	"17": 6.58,
	"18": 5.66,
	"19": 5.05,
	"20": 4.43,
	"21": 3.85,
	"22": 3.37,
	"23": 2.91
	}
}
\end{lstlisting}
\end{minipage}

Este archivo de datos es leído siempre que queremos modificar alguno de los parámetros de automatización desde el bot, como son la temperatura o la hora más temprana a la que pueden bajar las persianas, además de cuando se regenera el Cron.

\subsection{Scripts}
Utilizaremos dos tipos de scripts, Bash~\cite{misc:Linux} y Python~\cite{misc:Python}.

Una vez determinados los datos que vamos a necesitar en un notebook de Jupyter debemos realizar la extracción de estas líneas de código a un script Python.
En este punto tuve un problema bastante sencillo de resolver: el script necesita importar librerías para hacerlo correr y necesitaba instalarlas. En principio, esto no debería ser un problema ya que sabemos que se puede instalar con Pip. El problema surge al tener instalado Python v2 y Python v3, ya que si instalas las librerías utilizando Pip no funcionarán con Python3. Así que, hay que instalarlas con Pip3 para poder utilizarlas desde Python3:
\begin{lstlisting}[language=cpp,firstnumber=0]
pip install paquete
\end{lstlisting}
\begin{lstlisting}[language=cpp,firstnumber=1]
pip3 install paquete
\end{lstlisting}

Controlaremos la Raspberry Pi y sus periféricos mediante comandos Bash y realizaremos el cocinado de datos con Python. Se ha tomado esta determinación para conseguir la mayor velocidad de procesamiento posible así como para evitar posibles problemas. Por ello, la automatización que haremos desde CRON que lanzará otros scripts será mediante bash. 

Podemos ver el resultado final del Cron que genera la máquina diariamente:


\begin{lstlisting}[language=sh]
# Edit this file to introduce tasks to be run by cron.
# 
##   Entry              Description     Equivalent To
##   @yearly    Run once a year at midnight in the morning of January 1 0 0 1 1 *
##   @monthly   Run once a month at midnight in the morning of the first of the month 0 0 1 * *
##   @weekly    Run once a week at midnight in the morning of Sunday 0 0 * * 0
##   @daily     Run once a day at midnight 0 0 * * *
##   @hourly    Run once an hour at the beginning of the hour 0 * * * *
##   @reboot    Run at startup  @reboot
##   
# For more information see the manual pages of crontab(5) and cron(8)
# 
#minute (0-59),
#| hour (0-23),
#| | day of the month (1-31),
#| | | month of the year (1-12),
#| | | | day of the week (0-6 with 0=Sunday).
#| | | | |   commands

#--------------------------------------------------------------------------
#Este CRON ha sido generado en el instante Sun Dec 27 13:13:27 2020
#--------------------------------------------------------------------------
#Codigo de control Automatico de Persianas 
#--------------------------------------------------------------------------
#Subimos las persianas Modo mañana de Lunes a Domingo
37 8 * * * sh /home/pi/source/TFG/scripts/control/GPIO_off.sh
38 8 * * * sh /home/pi/source/TFG/scripts/control/Subir.sh
44 8 * * * sh /home/pi/source/TFG/scripts/control/GPIO_off.sh

#Encendemos las luces
7 18 * * * sh /home/pi/source/TFG/scripts/LucesOn.sh

#Bajamos TODAS
12 18 * * * sh /home/pi/source/TFG/scripts/control/GPIO_off.sh
13 18 * * * sh /home/pi/source/TFG/scripts/control/Bajar.sh
19 18 * * * sh /home/pi/source/TFG/scripts/control/GPIO_off.sh

#Apagamos las luces
19 18 * * * sh /home/pi/source/TFG/scripts/LucesOn.sh

#Todos los días se  reinicia la maquina par que el demonio siempre 
#este correcto por la mañana
05 04 * * * sudo reboot

#Lanzamos el script de toma de horas
05 00 * * * cd /home/pi/source/TFG/scripts/auto/ && sudo sh LanzaTodoElProceso.sh

#--------------------------------------------------------------------------
#Codigo de control Automatico de Calefaccion 
#--------------------------------------------------------------------------
 
0 0 * * 1-7 sh /home/pi/source/TFG/scripts/control/EncenderCaldera.sh
0 1 * * 1-7 sh /home/pi/source/TFG/scripts/control/EncenderCaldera.sh
0 2 * * 1-7 sh /home/pi/source/TFG/scripts/control/EncenderCaldera.sh
0 3 * * 1-7 sh /home/pi/source/TFG/scripts/control/EncenderCaldera.sh
0 4 * * 1-7 sh /home/pi/source/TFG/scripts/control/EncenderCaldera.sh
0 5 * * 1-7 sh /home/pi/source/TFG/scripts/control/EncenderCaldera.sh
0 6 * * 1-7 sh /home/pi/source/TFG/scripts/control/EncenderCaldera.sh
0 7 * * 1-7 sh /home/pi/source/TFG/scripts/control/EncenderCaldera.sh
0 8 * * 1-7 sh /home/pi/source/TFG/scripts/control/EncenderCaldera.sh
0 9 * * 1-7 sh /home/pi/source/TFG/scripts/control/EncenderCaldera.sh
0 10 * * 1-7 sh /home/pi/source/TFG/scripts/control/EncenderCaldera.sh
0 11 * * 1-7 sh /home/pi/source/TFG/scripts/control/EncenderCaldera.sh
0 12 * * 1-7 sh /home/pi/source/TFG/scripts/control/EncenderCaldera.sh
0 13 * * 1-7 sh /home/pi/source/TFG/scripts/control/EncenderCaldera.sh
0 14 * * 1-7 sh /home/pi/source/TFG/scripts/control/EncenderCaldera.sh
0 15 * * 1-7 sh /home/pi/source/TFG/scripts/control/EncenderCaldera.sh
0 16 * * 1-7 sh /home/pi/source/TFG/scripts/control/EncenderCaldera.sh
0 17 * * 1-7 sh /home/pi/source/TFG/scripts/control/EncenderCaldera.sh
0 18 * * 1-7 sh /home/pi/source/TFG/scripts/control/EncenderCaldera.sh
0 19 * * 1-7 sh /home/pi/source/TFG/scripts/control/EncenderCaldera.sh
0 20 * * 1-7 sh /home/pi/source/TFG/scripts/control/EncenderCaldera.sh
0 21 * * 1-7 sh /home/pi/source/TFG/scripts/control/EncenderCaldera.sh
0 22 * * 1-7 sh /home/pi/source/TFG/scripts/control/EncenderCaldera.sh
0 23 * * 1-7 sh /home/pi/source/TFG/scripts/control/EncenderCaldera.sh

#Instante de grabado de datos: 2020-12-27 - 13:13:27

\end{lstlisting}

~\\~\\~\\El script que lanza todo el código contiene las siguientes líneas:

\begin{lstlisting}[language=sh, caption={Script que lanza el proceso automático completo.}]
#!/bin/bash
path=$(pwd)
echo $path

# Este script se ha desarrollado para lanzar todo el proceso desde CRON.
python3 ${path%}/1_recabaInfo.py
python3 ${path%}/3_cocinado.py
sh ${path%}/4_reescribeCron.sh
\end{lstlisting}

Este archivo es un archivo Bash porque desde Cron no he conseguido lanzar los scripts Python, sin embargo no he tenido problemas al lanzarlo desde este script Bash siempre que especifique nuevamente que se lance con python3.

Por otro lado, para controlar los relés desde los GPIO, debemos configurarlos como OUT y con valor 1 para activarlos y, con valor 0 y configuración IN para desactivarlos. Podemos ver un ejemplo a continuación:
\begin{lstlisting}[language=sh, caption={Ejemplo activación y desactivación de relé.}]
gpio -g mode 17 out # Activa el relé conectado al pin 17
sleep 1             # Espera 1 segundo
gpio -g mode 17 in  # Desactiva el relé conectado al pin 17
\end{lstlisting}

Este código lo lanzaremos tanto desde el bot directamente para controlar las persianas <<en vivo>>, como desde el Cron para el control de persianas y 

\subsection{Instalación física}

Quizás, la parte más compleja del proyecto ha sido la instalación física dado mi grado de desconocimiento así como el encontrar las piezas idóneas para el domicilio.

Al comienzo del proyecto, hubo una gran parte de investigación e instalación del sistema ya que quería realizar el proyecto de la forma más profesional posible. Por ello, me dispuse a bucar en el REBT~\cite{manual:REBT} y en la normativa de ICT~\cite{manual:ICT-BT-21} para no saltarme ningún paso y que la instalación fuera totalmente legal.

Fue complejo leer tanto y comprender cómo realizar los cálculos de los cables dentro de los tubos y conseguir discernir entre la asignación real entre los diferentes canales de la vivienda. Una vez comprobado que utilizaría la norma actualizada de 2020 del REBT, realicé los cálculos para comprobar que podía `tirar' el cable UTP por estos canales con seguridad. 

\begin{figure}[h]
    \centering
    \includegraphics[width=1.0\textwidth]{img/Diagrama Básico.pdf}
    \caption{Diagrama físico. } \label{Img:diagramaBasico}
\end{figure}

Estuve valorando la opción de ‘tirar’ un cable de datos desde la primera caja de derivación después del Cuadro General de Mando y Protección o Cuadro eléctrico, hasta el RTR (Registro Terminación de Red) que es donde tengo la línea de datos más cercana, pero en único canal que comunica es de naturaleza eléctrica y no está bien mezclar tipos de instalaciones.

Buscando por Internet he podido ver que es más sencillo utilizar medios inalámbricos ya que no tendríamos que hacer tirada de cable, pero no me ofrecen la misma garantía ante un posible fallo de comunicación o interferencias, y con el cable eliminamos la posibilidad de sufrir este tipo de problemas. Sobre la categoría del cable, me he decantado por UTP cat5e (Ver imagen~\ref{Img:CablesDatos}) porque es un cable suficientemente capaz de transmitir los 3.3V en pico de nuestra Raspberry Pi y el precio no es muy alto.

Me he documentado sobre las características eléctricas de cada uno de los hilos del cable UPT cat5e y he visto que, según la norma TIA/EIA 568, el cableado debe soportar 500VDC~\footnote{VDC:Voltage Direct Current}, como podemos comprobar que figura en el punto A3 <<Insulation resistance>>, conforme a los test de voltaje que deben pasar de acuerdo con IEC60512-2 que figuran en la página 55 del documento de la norma~\cite{manual:568.2}. En cualquier caso, como las comunicaciones de nuestros gpio tienen un máximo de 3,3VDC y además dispone de dos salidas de 5VDC para alimentación de periféricos, necesitaremos soporte hasta 5V y nuestro cableado lo soporta sin problema.


El siguiente paso fue determinar la situación de la Raspberry Pi para que fuera lo más accesible posible ahorrando cable y que la instalación quedase lo más limpia posible, por lo que lo determiné para instalarla cerca de la primera caja de derivación eléctrica después del cuadro eléctrico general. Podemos ver un diagrama simplificado en la imagen~\ref{Img:diagramaBasico}. En ella podemos diferenciar varios elementos: Por un lado, tenemos el cuadro eléctrico y la primera caja de derivación, que ya vienen instaladas en el domicilio en la misma disposición que podemos ver en el diagrama (la caja de derivación encima del cuadro eléctrico), y en medio, me dispuesto la Raspberry Pi, para que puedan llegar los cables de datos fácilmente, ya que los hemos dispuesto por los canales eléctricos hasta los diferentes elementos a controlar (persianas, luz y calefacción).

\begin{figure}[h]
    \centering
    \includegraphics[width=0.9\textwidth]{img/CABLES.png}
    \caption{Disposición interna cables UTP, FTP y STP.} \label{Img:CablesDatos}
\end{figure}

Sobre el medio de comunicación entre dispositivos electrónicos, ha sido complejo escoger entre:
\begin{itemize}
    \item \textbf{Tecnologías valoradas:} UTP cat5e (Ver imagen~\ref{Img:CablesDatos}), UTP cat6 (Ver imagen~\ref{Img:CablesDatos}), Coaxial (ANSI/TIA/EIA 568.4D), ZigBee (IEEE 802.15.4), Bluetooth\cite{manual:IEEE802.11}. 
    \item \textbf{Tecnología elegida:} UTP cat5e (Ver imagen~\ref{Img:CablesDatos}).
\end{itemize}

Cabe destacar una particularidad de mi instalación: el router está albergado dentro del RTR ya que sobra espacio dentro del mismo y mi router está optimizado para no calentarse en espacios con poca ventilación. El canal que podemos ver entre el cuadro eléctrico general y el RTR es una toma eléctrica que, por normativa debe estar ahí y es la que alimenta algún dispositivo que deba estar dentro de esta situación.

Otro punto nuevo para mi fue el crimpado de los pines para poder trabajar con ellos. Al principio, no contaba con una crimpadora de estos pines JST-XH pero terminé haciéndome con una para que los cables quedaran correctamente fijados a los conectores.

\begin{figure}[h]
    \centering
    \includegraphics[width=0.9\textwidth]{img/fotos/caja-persiana.jpeg}
    \caption{Caja de derivación secundaria instalada.} \label{Img:CajaDerivacion}
\end{figure}
\begin{figure}[h]
    \centering
    \includegraphics[width=0.9\textwidth]{img/fotos/explicación caja instalada.jpg}
    \caption{Caja de derivación principal instalada.} \label{Img:CajaDerivacionPrincipal}
\end{figure}
\begin{figure}[h]
    \centering
    \includegraphics[width=0.9\textwidth]{img/fotos/RBP-instalada.jpeg}
    \caption{Instalación domótica y caja de derivación principal.} \label{Img:InstalacionDomotica}
\end{figure}
\begin{figure}[h]
    \centering
    \includegraphics[width=0.9\textwidth]{img/fotos/protoboard-instalada.jpeg}
    \caption{Protoboard conectada.} \label{Img:ProtoboardConecada}
\end{figure}

Tras valorar la información y la toma de decisiones hice la instalación física quedando del siguiente modo:
\begin{itemize}
    \item En la imagen~\ref{Img:CajaDerivacion} podemos ver una caja de derivación donde ya tenemos albergada la placa de relés. 
    \item En la imagen~\ref{Img:CajaDerivacionPrincipal} vemos la caja de derivación principal con la instalación terminada indicando algunos elementos.
    \item En la imagen~\ref{Img:InstalacionDomotica} vemos la situación final de la tirada de cable y conectorizado.
    \item En la imagen~\ref{Img:ProtoboardConecada} podemos diferenciar que existen dos tipos de conectores. Los redondos ya vienen conectorizados de fábrica pero los cuadrados los he fabricado según las necesidades. Estos conectores cuadrados son del tipo~\href{https://ae01.alicdn.com/kf/H4205e9c4ec4c4be4864e44b6925a22bdf/10-juegos-de-conector-de-Cable-de-2-54mm-XH2-54-conector-XH-macho-y-hembra.jpg_Q90.jpg_.webp}{JST-XH}.
\end{itemize}

\begin{figure}
    \centering
    \includegraphics[width=0.9\textwidth]{img/RbP_CompletaLineas.png}
    \caption{Conectorizado final de las líneas GPIO.} \label{Img:RbP_CompletaLineas}
\end{figure}

El conectorizado final de las salidas de los GPIO es el que se puede ver en la imagen~\ref{Img:RbP_CompletaLineas}

Podemos interactuar con los puertos GPIO desde Bash y desde Python~\cite{misc:Python}. Se ha decidido utilizar lenguaje Bash para interactuar con los GPIO, dejando la potencia de Python para la obtención y procesado de datos. En ocasiones se producen errores al lanzar desde Cron scripts Python, por lo que se lanzará todo desde scripts bash.


\subsection{Diagramas eléctricos}
Para mayor claridad he generado unos planos eléctricos para aclarar como se ha realizado la instalación.

\begin{figure}
    \centering
    \includegraphics[width=0.9\textwidth]{img/Diagramas/rele-pulsador-motor.png}
    \caption{Raspberry Pi, relé y pulsador.} \label{Img:Relé+Pulsador+Rbp_Fritzing}
\end{figure}
\begin{figure}
    \centering
    \includegraphics[width=0.5\textwidth]{img/Diagramas/relé-RBP.png}
    \caption{Raspberry y relé conectados.} \label{Img:Relé+Rbp_Fritzing}
\end{figure}

\begin{itemize}
    \item En la imagen~\ref{Img:Relé+Pulsador+Rbp_Fritzing} podemos ver que tenemos conectada una placa con dos relés y un pulsador en paralelo, con el motor de la persiana. La protoboard realmente no existe en este punto pero la he incluido para clarificar el circuito.
    \item En la imagen~\ref{Img:Relé+Rbp_Fritzing} podemos ver como se conecta el relé a una persiana identificando cada uno de los pines.
\end{itemize}

\begin{figure}
    \centering
    \includegraphics[width=0.6\textwidth]{img/Diagramas/PulsadorInterno.png}
    \caption{Funcionamiento interno de un pulsador para persianas.} \label{img:PulsadorInterno}
\end{figure}

Para explicar el funcionamiento de un pulsador de 3 posiciones para persianas podemos ver la imagen~\ref{img:PulsadorInterno}. En ella podemos ver como en la posición central no tiene ninguna salida, pero en la salida 1 el orden de salida de cabes es (+/-) y en la salida 2 el orden de salida de cables es (-/+). La lógica del cambio de orden en la salida de cables es porque según el sentido de la polaridad el motor girará en un sentido o en otro. Esto se respalda mediante la~\href{https://fisica.laguia2000.com/dinamica-clasica/fuerzas/ley-de-laplace-fuerza-ejercida-sobre-un-conductor}{segunda ley de Laplace} para cada espira del bobinado del motor quedando de la forma~\href{http://www.uco.es/grupos/giie/cirweb/teoria/tema_11/tema_11_01.pdf}{$M=N*(I*B*l+sen\theta)$}.

Siguiendo esta lógica, podemos ver que en la imagen~\ref{Img:Relé+Pulsador+Rbp_Fritzing} tenemos dos relés de forma que cada uno de ellos deja pasar la corriente en un sentido. En la realidad, el motor no tiene por qué tener dos entradas positivas y dos negativas pero en el dibujo queda mejor ilustrado que son entradas diferentes.

Resumiendo, la instalación mecánica y nuestra instalación mediante relés se instalarán en paralelo para poder activar una opción o la otra según la ocasión.

\subsection{Pruebas Físicas}

\begin{figure}
    \centering
    \includegraphics[width=0.6\textwidth]{img/Plano_Placa_Pruebas.png}
    \caption{Plano de tablero de pruebas.} \label{Img:Plano_Placa_Pruebas}
\end{figure}

La primera prueba, consistía en la instalación de la Raspberry Pi con un relé (ver imagen~\ref{Img:Rele1}) y una bombilla conectada a 220VAC\footnote{VAC:Voltage Alternating Current}, y se llevó a cabo con éxito. Podemos ver un diagrama de como se conectó en la imagen~\ref{Img:Plano_Placa_Pruebas}.

Llegados a este punto, era interesante el comprobar que se podía lanzar correctamente un script de este tipo desde CRON, encendiendo y apagando la bombilla, que también fue un éxito.

La tirada de cable y conectorizado en todo el domicilio, desde la primera caja de empalmes a las 5 persianas ya que es el punto del domicilio donde se recogen los cables eléctricos que llegan hasta las persianas, quedó como podemos ver en el diagrama~\ref{Img:diagramaBasico}. 

\subsection{Automatizado y puesta a producción}
Una vez se tienen las funcionalidades de los scripts corriendo correctamente, debemos asegurar las funciones y métodos de nuestra aplicación ante posibles fallos utilizando try/except. De esta manera siempre sabremos si se han actualizado correctamente los datos necesarios para que modifique nuestro Cron reportando un mensaje de error en caso contrario.

Para poder generar el proceso de que hablamos en la imagen~\ref{Img:Cron1} no es necesario salvar la cabecera del archivo de configuración de Cron, pero en nuestro caso, haremos una copia de la cabecera de forma que nuestra manipulación sea lo menos intrusiva posible. De esta manera, generaremos un Cron similar al que tenemos <<de serie>> pero con nuestras líneas de lanzado de scripts.
En ese punto he tenido problemas al intentar ejecutar scripts Python desde Cron impidiendo que se obtuvieran los datos.

Uno de los puntos de la automatización es la generación de un demonio para nuestro bot de forma que se ejecutará con el inicio del sistema y podremos trabajar con el una sencilla sentencia:

\begin{lstlisting}[language=sh,firstnumber=0]
sudo systemctl bot stop
\end{lstlisting}
\begin{lstlisting}[language=sh]
sudo systemctl bot start 
\end{lstlisting}
Para conseguirlo, he realizado los siguientes pasos:

He generado el archivo el archivo de nuestro demonio en lib/systemd/system/ con la extensión <<.service>>. Lo he escrito fácilmente con nano: 
\begin{lstlisting}[language=sh, firstnumber=0]
sudo nano /lib/systemd/system/bot.service
\end{lstlisting}
El contenido del archivo es:
\begin{lstlisting}[language=sh, caption={Modificaciones en el archivo /lib/systemd/system/bot.service.}, firstnumber=0]
[Unit]
Description=Lanza el bot de control domotico
After=network.target
StartLimitIntervalSec=0

[Service]
Type=simple
Restart=always
RestartSec=1
User=pi
WorkingDirectory=/home/pi/source/TFG/scripts/bot/
ExecStart=/usr/bin/env python3 /home/pi/source/TFG/scripts/bot/bot.py

[Install]
WantedBy=multi-user.target
\end{lstlisting}
Después debemos actualizar los demonios con: 
\begin{lstlisting}[language=sh, firstnumber=0]
systemctl daemon-reload
\end{lstlisting}
Iniciar el demonio: 
\begin{lstlisting}[language=sh, firstnumber=0]
sudo systemctl start bot
\end{lstlisting}
Parar el demonio: 
\begin{lstlisting}[language=sh, firstnumber=0]
sudo systemctl stop bot
\end{lstlisting}
Estado del demonio, con el que podremos conocer el pid~\footnote{Identificador del proceso}, que siempre es útil: 
\begin{lstlisting}[language=sh, firstnumber=0]
sudo systemctl status bot
\end{lstlisting}

Para incluirlo en el inicio de la máquina habría que moverlo a /etc/init.d y luego ejecutar: 
\begin{lstlisting}[language=sh, firstnumber=0]    
sudo update-rc.d bot defaults
sudo systemctl daemon-reload
sudo systemctl enable bot
sudo systemctl start bot
\end{lstlisting}  


\subsection{Telegram Bot}\label{5.TelegramBot}
En primer lugar, explicaré qué es un bot

El medio de comunicación con el sistema domótico es un bot de Telegram que he diseñado y programado para hacer completamente funcional el proyecto. Desde el Bot, podemos controlar las persianas de forma instantánea, modificar la configuración del sistema domótico y simulador de presencia, volver a generar recolección de datos y el grabado de los archivos. También podemos obtener información sobre la información recopilada, sobre nuestra Raspberry Pi y sobre el funcionamiento futuro del sistema domótico y simulador de presencia.

Afortunadamente estaba mejor documentado que el resto de la recursos que he consultado para el proyecto pero aún así no ha resultado tarea fácil entender la lógica del bot. 
Podemos ver la lógica de comunicaciones del bot en la imagen~\ref{Img:FuncionamientoBot}. 

\begin{figure}[h]
    \centering
    \includegraphics[width=1.0\textwidth]{img/Diagramas/FuncionamientoBot.png}
    \caption{Comunicación del bot. } \label{Img:FuncionamientoBot}
\end{figure}

Todos los mensajes de Telegram que llegan a nuestro bot, tienen la siguiente estructura json:

\begin{lstlisting}[language=json,firstnumber=0, basicstyle=\small, caption={Contenido de un mensaje de Telegram.}]
{
    'content_type':'text',
    'id':0000,
    'message_id':0000,
    'from_user':{
        'id':000000000,
        'is_bot':False,
        'first_name':'David',
        'username':'David',
        'last_name':'David',
        'language_code':'es',
        'can_join_groups':None,
        'can_read_all_group_messages':None,
        'supports_inline_queries':None
    },
    'date':1609105539,
    'chat':{
        'id':000000000,
        'type':'private',
        'title':None,
        'username':'David',
        'first_name':'David',
        'last_name':'David',
        'all_members_are_administrators':None,
        'photo':None,
        'description':None,
        'invite_link':None,
        'pinned_message':None,
        'permissions':None,
        'slow_mode_delay':None,
        'sticker_set_name':None,
        'can_set_sticker_set':None
    },
        'forward_from':None,
        'forward_from_chat':None,
        'forward_from_message_id':None,
        'forward_signature':None,
        'forward_sender_name':None,
        'forward_date':None,
        'reply_to_message':None,
        'edit_date':None,
        'media_group_id':None,
        'author_signature':None,
        'text':'a',
        'entities':None,
        'caption_entities':None,
        'audio':None,
        'document':None,
        'photo':None,
        'sticker':None,
        'video':None,
        'video_note':None,
        'voice':None,
        'caption':None,
        'contact':None,
        'location':None,
        'venue':None,
        'animation':None,
        'dice':None,
        'new_chat_member':None,
        'new_chat_members':None,
        'left_chat_member':None,
        'new_chat_title':None,
        'new_chat_photo':None,
        'delete_chat_photo':None,
        'group_chat_created':None,
        'supergroup_chat_created':None,
        'channel_chat_created':None,
        'migrate_to_chat_id':None,
        'migrate_from_chat_id':None,
        'pinned_message':None,
        'invoice':None,
        'successful_payment':None,
        'connected_website':None,
        'reply_markup':None,
        'json':{
        'message_id':0000,
        'from':{
        'id':000000000,
        'is_bot':False,
        'first_name':'David',
        'last_name':'David',
        'username':'David',
        'language_code':'es'
    },
    'chat':{
        'id':000000000,
        'first_name':'David',
        'last_name':'David',
            'username':'David',
        'type':'private'
    },
    'date':1609105539,
    'text':'a'
    }
}
\end{lstlisting}

De esta estructura utilizamos únicamente unos pocos y podríamos resumir en los siguientes:
\begin{lstlisting}[language=json,firstnumber=0, basicstyle=\small, caption={Resumen de los datos que utilizamos del mensaje de Telegram.}]
{
    'content_type':'text',
    'message_id':0000,
    'text':'a',
    'chat':{
        'id':000000000,
        'first_name':'David',
        'last_name':'David',
        'username':'David'}
}
\end{lstlisting}

Al comienzo del proyecto pensé en crear un grupo de usuarios para gestionar esta información pero he creído oportuno que cada usuario tenga su propia interfaz aunque realmente no habría problema en hacerlo de esta manera y hubiera facilitado el proyecto. Esto se ha solucionado incluyendo la lista de usuarios que pueden interactuar con el bot en el archivo de configuración. Para este paso, debemos recoger el id del usuario. Este id aparece varias veces en el json que recibimos en cada mensaje por lo que valdría cualquiera de ellos. Para el resumen he escogido el id que pertenece al objeto chat, junto al nombre y apellidos del propietario y el nombre de usuario.
El número de usuario es necesario para saber a quién le enviamos el mensaje. Por ejemplo, al comienzo del bot del proyecto tenemos las siguientes líneas:

\begin{lstlisting}[language=Python]
for usuario in usuarios:
    bot.send_message(usuario, "Iniciando Sistema a las "+str(hora))
\end{lstlisting}

Estas líneas envían un mensaje de inicio con la hora a la que se produce el evento a cada uno de los usuarios de la lista de admitidos.

Otro de los puntos necesarios es el de texto que envía el usuario. Gracias a éste obtendremos la orden que enviamos por Telegram y podremos actuar en consecuencia.
Hay un punto a tener en cuenta para comprender el funcionamiento del bot, éste radica en las líneas que hay al principio de cada uno de los métodos de que dispone el bot:

\begin{lstlisting}[language=Python]
@bot.message_handler(commands=['temp'])
@bot.message_handler(func=lambda message: message.text == "t")
@bot.message_handler(func=lambda message: message.text == "T")
def command_long_text(m):
    usuario = m.chat.id
    if (compruebaUsuario(m)):
        temperaturasManana.temperaturas(m , bot)
\end{lstlisting}

En ellas podemos ver, por orden, el comando temp, al que podemos acceder enviando el comando /temp. Posteriormente vemos dos funciones lambda, éstas significan que si introducimos únicamente esa letra también lanzará el método. He introducido estos comandos rápidos para facilitar la introducción de órdenes.
Observamos que si el usuario que escribe está en la lista de admitidos, lanzará el método temperaturas pasándole el mensaje enviado y el objeto bot, que contiene información sobre el usuario y el mensaje que envía.
Por último, el método al que se llama, recoge esta información y opera en consecuencia según lo que se le pida, bien con una función u otro método.
\begin{comment}
\capitulo{6}{Trabajos relacionados}
Este apartado sería parecido a un estado del arte de una tesis o tesina. En un trabajo final grado no parece obligada su presencia, aunque se puede dejar a juicio del tutor el incluir un pequeño resumen comentado de los trabajos y proyectos ya realizados en el campo del proyecto en curso. 
\end{comment}

\capitulo{6}{Trabajos relacionados}
Desde la aparición de elementos electrónicos accesibles y asequibles se ha intentado, con mayor o menor fortuna, generar sistemas que nos ayuden en nuestro día a día. A continuación se exponen proyectos con objetivos similares.

\section{Comparativa con otros proyectos}
En este apartado mostraré otras opciones de acometer un proyecto similar y compararé los principales puntos de éstos con el presente proyecto para poder tener una visión global entre estas opciones.

\subsection{Interacción remota para elementos de un hogar mediante una red de sensores actuadores}\label{Proy_Domo_1}
Se trata de un  se trata de un TFG/TFM desarrollado en la Universidad de Salamanca que trata sobre el control domótico basado en un conjunto de sensores instalados en el domicilio utilizando nodos intermedios para realizar el control de los dispositivos finales y la comunicación con los sensores.

Podemos ver su referencia en la bibliografía en la referencia~\cite{misc:TFG_Salamanca}.

En la columna `Domo\_1' de la tabla~\ref{tabla:comparativa-proyectos} podemos ver algunas características de este proyecto.

\subsection{Diseño e implementación de un sistema domótico basado en Raspberry Pi}\label{Proy_Domo_2}
Éste es un TFG/TFM desarrollado en la Universidad Carlos III, en Madrid que trata sobre el control domótico de un domicilio basándose únicamente en un conjunto de sensores instalados en el domicilio.

\begin{itemize}
    \item Url del trabajo: \url{https://e-archivo.uc3m.es/bitstream/handle/10016/26313/TFG_Hector_Santos_Senra.pdf?sequence=1&isAllowed=y}
\end{itemize}

En la columna `Domo\_2' de la tabla~\ref{tabla:comparativa-proyectos} podemos ver algunas características de este proyecto.

\section{Fortalezas y debilidades este proyecto}

Las principales fortalezas del proyecto que se presenta en esta memoria son:

\begin{itemize}
\item
    Éste es un proyecto que cualquiera puede implementar fácilmente en su domicilio haciendo una copia local del repositorio y haciendo unas sencillas configuraciones en el archivo de configuración.
    
\item
    En este proyecto se ha tenido en cuenta la eficiencia de recursos y de consumo de la propia máquina Raspberry Pi~\cite{misc:RbPWeb}, por lo que no se han incluido servicios extraordinarios como puede ser un servidor de bases de datos o web, como sucede en otros proyectos. Se ha optado por incluir un bot que `escucha' las órdenes y actúa conforme a un pequeño archivo de configuración y actúa contra la máquina utilizando los lenguajes nativos de la máquina. De esta manera, el gasto en cómputo es menor y también el utilizado en mantener los servicios en memoria, quedando recursos suficientes para continuar haciendo un uso normal de la máquina para cualquier otra tarea.
    Además, al no tener servidores levantados y con puertos de comunicaciones abiertos, aumentamos la seguridad de nuestra instalación.

\item
    Se ha procurado utilizar los lenguajes integrados de forma nativa en el Sistema Operativo para evitar posibles fallos de integración entre ellos y aumentando la fiabilidad del sistema. Estos son Python~\cite{misc:Python} y Bash~\cite{misc:Linux}. 
    
\item
    El sistema automatizado de nuestro proyecto se nutre de la información contrastable obtenida de API's reconocidas.
    
\begin{table}
\centering
\begin{tabular}{lccc}
\toprule
Características & SDI & Domo\_1 & Domo\_2  \\
\midrule
Proyecto libre                          & \cellcolor{green!25} {\checkmark} & \cellcolor{green!25} {\checkmark} & \cellcolor{red!25} {\xmark} \\
No precisa montar servicios             & \cellcolor{green!25} {\checkmark} & \cellcolor{red!25} {\xmark} & \cellcolor{red!25} {\xmark} \\
No requiere lenguajes no nativos en el SO  & \cellcolor{green!25} {\checkmark} & \cellcolor{red!25} {\xmark} & \cellcolor{red!25} {\xmark} \\
Obtiene información externa contrastada & \cellcolor{green!25} {\checkmark} & \cellcolor{red!25} {\xmark} & \cellcolor{red!25} {\xmark} \\
Interacción multiplataforma             & \cellcolor{green!25} {\checkmark} & \cellcolor{green!25} {\checkmark} & \cellcolor{green!25} {\checkmark} \\
No necesita nodos intermedios           & \cellcolor{green!25} {\checkmark} & \cellcolor{red!25} {\xmark} & \cellcolor{green!25} {\checkmark} \\
Cableado entre elementos                & \cellcolor{green!25} {\checkmark} & \cellcolor{red!25} {\xmark} & \cellcolor{green!25} {\checkmark} \\
WiFi entre elementos                    & \cellcolor{red!25} {\xmark} & \cellcolor{green!25} {\checkmark} & \cellcolor{red!25} {\xmark} \\
\bottomrule
\end{tabular}
\caption{Comparativa de las características de los proyectos.}
\label{tabla:comparativa-proyectos}
\end{table}

    
\item
    Se puede interactuar con el sistema domótico de varias formas. La más cómoda es mediante el bot de Telegram creado en exclusiva para controlar las funcionalidades principales de nuestro sistema domótico, aunque también se puede contectar de forma remota desde otra máquina mediante VNC~\footnote{VNC: <<Virtual Network Computing>>, programa de control remoto de equipos informáticos basado en una estructura <<cliente-servidor>>} siempre que se esté en la misma red. Se puede interactuar con el bot mediante aplicaciones Android o IOs, o desde navegadores de uso extendido como Chrome, Firefox o Edge, entre otros.
    
\item
    No se necesitan elementos intermedios de ningún tipo, únicamente tenemos nuestra Raspberry Pi~\cite{misc:RbPWeb} y los actuadores o relés~\ref{Img:Rele1}. De esta manera evitamos posibles fallos de los elementos intermedios. Podemos comprobarlo en la imagen~\ref{Img:diagramaBasico}.

    
\item
    Se ha preferido cablear la instalación para evitar que los elementos receptores puedan quedar en un estado no deseado así como evitar reconfigurar elementos ante cualquier posible cambio de alguno de los elementos como pueden ser el \textit{dongle} WiFi o el router, ver secciones~\ref{concepto:WIFI} y~\ref{4:Router} respectivamente.
    También, al dejar libres los canales WiFi, evitamos contribuir a la actual saturación de frecuencias en los espectros de 2.4GHz y 5GHz~\cite{manual:IEEE802.11}.

\end{itemize}


Podemos ver en la tabla~\ref{tabla:comparativa-proyectos} de comparativa de proyectos lo anteriormente expuesto.



Las principales debilidades del proyecto son:

\begin{itemize}
\tightlist
\item
    Hace falta cierta soltura a la hora de trabajar con el cableado UTP, ver sección~\ref{4:guiaPasacables}.
\item
    Es necesaria una línea de datos para que se actualicen los parámetros de los scripts.

\end{itemize}

\capitulo{7}{Conclusiones y Líneas de trabajo futuras}
\begin{comment}
Todo proyecto debe incluir las conclusiones que se derivan de su desarrollo. Éstas pueden ser de diferente índole, dependiendo de la tipología del proyecto, pero normalmente van a estar presentes un conjunto de conclusiones relacionadas con los resultados del proyecto y un conjunto de conclusiones técnicas. 
Además, resulta muy útil realizar un informe crítico indicando cómo se puede mejorar el proyecto, o cómo se puede continuar trabajando en la línea del proyecto realizado. 
\end{comment}

Con este apartado concluye la memoria de este proyecto, desglosando cada uno de los objetivos superados en el mismo. Posteriormente se sugerirán algunas posibles líneas de trabajo futuras que puedan dar continuidad al proyecto consiguiendo una mejora notable.

\section{Conclusiones}
Una vez finalizado el proyecto, podemos concluir con las siguientes afirmaciones:
\begin{itemize}
\item Se ha conseguido crear un Sistema de Simulación de Presencia que funciona según las horas de luz del día. En cada jornada, el Sistema genera una consulta de forma automática obteniendo los parámetros de ubicación, y posteriormente, los parámetros de salida y puesta de sol. Seguidamente procesa los datos para controlar los diferentes elementos externos.

\item Además, se ha ampliado el Sistema de Simulación de Presencia con el Sistema Domótico que controla la caldera, las persianas y las luces. De esta manera, también controlamos a que hora se suben las persianas por las mañanas y se controla el encendido de la caldera. Esta parte es parcialmente automática ya que tenemos la opción de subir las persianas, tras el amanecer, a la hora que queramos.

\item Se ha conseguido dotar al sistema de un control instantáneo de las persianas. Este apartado beneficia a la hora de querer controlar una persiana de forma individual y en un intervalo de tiempo corto.

\item También, se han incluido órdenes informativas que funcionan a partir de la información recopilada porque no cuesta nada poder contar con la información recopilada de una forma limpia e instantánea siempre que la queramos. Además, se dibujan diariamente una gráfica con las temperaturas del día siguiente que también están a nuestra disposición siempre que solicitemos el envío de alguna de ellas

\item El sistema es autónomo de la ubicación en la que está instalado ya que obtiene su ubicación en cada consulta lo cual es necesario para poder obtener las temperaturas del día siguiente.

\item Se ha generado un bot para automatizar todo el proceso de comunicación con el Sistema Domótico Inteligente de forma que podemos interactuar con él desde cualquier lugar de una manera segura y minimizando el cómputo, los servicios activos y también los puertos abiertos.

\item El sistema, finalmente, funciona con un mínimo de instrucciones y programas en background, lo cual contribuye a un mínimo consumo energético y pone a disposición del usuario los recursos de la máquina para realizar otras tareas que se precisen.

\item Estoy satisfecho con la instalación física realizada en el domicilio, me parece profesional ya que sigue las líneas de las normativas legales vigentes por las que se rigen este tipo de instalaciones.

\item Otro punto muy satisfactorio es el de la mínima inversión acometida en el proyecto junto a una funcionalidad completa hacen que sienta que el dinero ha sido muy bien invertido ya que ha aumentado notablemente la comodidad y confortabilidad del domicilio. Además se nota que, ahora en invierno, la temperatura de la vivienda desciende más despacio al bajarse las persianas cuando anochece.

\item La única pega que le podría al proyecto es que al utilizar la versión 2B de la Raspberry Pi, algunas peticiones tardan unos segundos más de lo que me gustaría. Esto sucede cuando se le ordena que modifique la hora de subida de las persianas ya que debe abrir y modificar varios archivos, pero se soluciona con una versión más nueva de Raspberry Pi.

\end{itemize}


\section{Líneas de trabajo futuras}
El proyecto ha quedado bastante <<redondo>>, es decir, se ha generado una funcionalidad completa y se ha dado solución a una necesidad dentro de los parámetros que se esperaba. No obstante, el proyecto es susceptible de crecer para aumentar el confort y mejorar el Sistema Simulador de Presencia y el Sistema Domótico.

\begin{itemize}
\item El proyecto se centra únicamente en valores externos como son el sol o la temperatura exterior. Se puede incluir un sensor de temperatura en la vivienda de forma que también se controle la calefacción siempre que la caldera esté encendida.

\item Al Sistema Simulador de Presencia podemos incluirle sonido. Una buena opción puede ser conseguir emitir un programa en streaming~\footnote{Retransmisión en directo.}, ya sea solo audio o también con vídeo.

\item Ya que hemos incluido Telegram y éste dispone de llamadas, podemos incluir un <<portero IP>> de forma que te llame a un usuario y puedas comunicarte con quien llame a la puerta. De esta manera no sólo simulas que estás en el domicilio sino que también sabes quién ha estado en tu puerta. Este portero puede ser sólo con voz o también con imagen.

\item Al mismo sistema, se le pueden incorporar unos sensores PIR para saber en que parte de la casa hay movimiento y disponer un control de luces o grabado de imágenes.

\item También se puede optar por crecer en la rama de seguridad añadiendo sensores de gas, agua y humo para y notifique las alertas.

\item Si se vive en un lugar con puerta de garaje individual, se puede programar para interactuar con ésta permitiendo el acceso con un mensaje.

\end{itemize}



\bibliographystyle{plain}
\bibliography{bibliografia}

\end{document}
